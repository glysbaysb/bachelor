%\maketitle 
%\thispagestyle{empty}
%\clearpage

\thispagestyle{empty}
\begin{center}
	\includegraphics[width=12cm]{fhlogo}\\
	\large
	\textbf{Fachbereich für}
	\textbf{Elektrotechnik und Informationstechnik}\\
	\large
	\textbf{Studiengang Technische Informatik}\\
	\vspace*{3cm}
	\LARGE
	\textbf{Benutzerhandbuch}\\
	\Huge
	\vspace*{1cm}
	\textsf{\textbf{XXXXXXXXXXXXXXXXXXXX}}\\
	\vspace*{3cm}
	
	\vfill
	\normalsize
	\newcolumntype{x}[1]{>{\raggedleft\arraybackslash\hspace{0pt}}p{#1}}
	\begin{tabular}{x{6cm}p{7.5cm}}
		\rule{0mm}{5ex}\textbAutor:} & Geert Martin Ijewski
	\end{tabular} 
\end{center}
\pagebreak

\section{Bedienungsanleitung}
Im folgenden wird die Bedienung der verschiedenen Teile des Systems erkl{\"{a}}rt. Hierbei wird zuerst auf das Simulationsprogramm und dann auf
den restlichen Code, inklusive Benutzerschnitstelle, eingegangen.

\section{Die Simulation}
Durch die Benutzt von Unity ist die Simulation auf vielen verschiedenen Plattformen lauff{\"{a}}hig\footnote{Getestet wurde hierbei die x86_64 Windows
und die x86 Linux Version}. Um {\"{A}}nderungen an ihr vorzunehmen, oder sie einfach nur f{\"{u}}r weitere Plattformen neu zu kompilieren.

\subsection{Kompilierung}
Hierf{\"{u}}r ist der Unity Editor 2017 n{\"{o}}tig; das Projekt ist nicht r{\"{u}}ckw{\"{a}}rtskomptabiel zu {\"{a}}lteren Versionen. Im Unity Editor
ist eine IDE enthalten mit der es m{\"{o}}glich ist den C\# Code zu kompilieren, bei der Installation wird allerdings die Installation von Visual Studio
2017 vorgeschlagen. Diese f{\"{u}}gt sich (mit den richtigen Plugins) besser in das Geschehen ein, so das es zum Beispiel m{\"{o}}glich ist im Code
Breakpointe zu setzen; falls dieser dann getroffen wird, wird das Spiel dann pausiert und kann debuggt werden (also k{\"{o}}nnen die Variablen eingesehen
und ver{\"{a}}ndert werden und es ist m{\"{o}}glich durch den Quellcode in Einzelschritten durchzugehen.

Da die \href[NNanomsg Bibliothek]{https://github.com/mhowlett/NNanomsg} genutzt wird um mit den Votern zu kommunizieren, muss diese auch der Simulation beigelegt werden. 
Dazu werden jeweils zwei Unternordner, einer namens "x86", einer namens "x64" angelegt in welche dann die jeweiligen vorkompilierten Dateien gelegt werden. F{\"{u}} die
Benutzung unter Windows sind dies .DLL Dateien, bei Linux sind es .so Dateien.


\subsection{Einstellungen}
\todo{Wie geht das mit den FI Config und den anderen Configs?}


\section{Der Code inklusive Interface}
