\clearpage
\section{Architektur}
\todo{warum beschreiben}

\subsection{Simulation}
Der wichtigste Teil dieser Bachelorarbeit ist die Simulation. Sie soll nicht nur ein physikalisch glaubw{\"{u}}rdiges Modell
f{\"{u}}r Roboterbewegungen darstellen, den Votern die ben{\"{o}}tigten Informationen zur Steuerung bereitstellen und die
Steuerkommandos auch empfangen und ausf{\"{u}}hren sondern dies auch graphisch darstellen. Dar{\"{u}}ber hinaus soll die
Simulation das Geschehen auch graphisch darstellen und dem Benutzer die M{\"{o}}glichkeit geben mit der Simulation zu
interagieren, zum einen durch Bewegen der Kamera, zum anderen durch eine Fernsteuerung eines Roboters, zwecks Fehlerinjektion.

Um all diese Aufgaben zu erf{\"{u}}llen gibt es mehrere Klassen. Diese lassen sich grob in zwei Kategorien einteilen: als
erstes die Klassen die f{\"{u}}r die Simulation und das Netzwerk zust{\"{a}}ndig sind, als zweites die Klassen die den Benutzer
mit dem System interagieren lassen.
\todo{Klassendiagramm neu malen}
\begin{figure}
	\centering
	\includevisio[width=\textwidth]{Arch}
	\caption{Klassendiagramm}
	\label{fig:arch}
\end{figure}

In der ersten Kategorie sind dabei die RPC Klasse die Daten serialisiert und dann {\"{u}}ber \textit{publish-suscribe} anderen
zur Verf{\"{u}}gung stellt. Sie nimmt auch Steuerkommandos an und leitet diese an den jeweiligen Roboter weiter. Dieser f{\"{u}}hr
das Steuerkommando dann aus -- allerdings unter Beachtung des Fehlermodells; es kann also sein das zu diesem Zeitpunkt zum
Beispiel ein Motor ausgefallen ist und daher sich nur der andere bewegen kann.

Um dem Benutzer die M{\"{o}}glichkeit zu geben einen anderen Teil der Simulation zu sehen, gibt es die Klasse "Camera Controller".
Sie ist da{\"{u}}r zust{\"{a}}ndig die Kamera zu bewegen. Abgesehen von den Objekten in der Welt wird, oben links in der Ecke,
auch angezeigt wie stark die Welt gekippt ist. Dies erledigt das "Radar" Skript, das drei farbige Kreise zeichnet welche die
Kippung symbolisieren und auf diesen einen schwarzen Kreis zeichnet der anzeigt wie stark gerade gekippt wurde.

\subsection{Kommunikation zwischen den Systemteilen}
Bei verteilten Systemen (wie diesem) ist einer der wichtigsten Punkte wie die Kommunikation zwischen den Systemteilen abl{\"{a}}uft.
Hierbei kann zwischen zwei Gruppierungen unterschieden werden: einmal die Kommunikation zwischen Simulation und Voter und andererseits
die Kommunikationen zwischen den Controllern und zwischen Controller und Voter. Da die letztere von den Studierenden geplant werden soll,
wird sie hier nicht weiter untersucht.

Durch die Simulationsidee  und Anforderungen der Fehlermodells sind einige Eckpunkte der Architektur bereits vorgegeben. Da alle Voter
in der gleichen Welt miteinander interagieren sollen, ist es am einfachsten wenn die Simulation ein Server ist, zu dem sich alle
verbinden. Dieser Server kann dann intern die Simulation vorantreiben und nur die Ergebnisse dieser weiterversenden. Die Alternative
w{\"{a}}re ein System ohne zentrale Instanz in dem die Simulation auf jedem Robter ausgef{\"{u}}hrt wird. In diesem Fall gibt es
allerdings das Problem das sich der Status der Simulationen voneinander desychnorisieren kann, zum Beispiel wenn ein Voter
eine schw{\"{a}}cheren CPU hat, die l{\"{a}}nger zur Berechnung eines Simulationstaktes braucht als die anderen Voter. Daher ist
das Modell mit zentrale Instanz vorzuziehen; es stellt automatisch sicher das alle Voter (im Normalfall, also ohne Fehler)
die gleichen Informationen kriegen und anhand dieser dann reagieren.

Auch soll die Simualation nicht rundenbasiert, sondern \todo{mhm kann ich hier echtzeit sagen?} sein. Gerade bei einer 3D-Ansicht
wurde es unangehnem auffallen wenn auch nur wenige Millisekunden nichts passiert. Viel wichtiger ist allerdings das dies eine
unrealistische Vereinfachung gegen{\"{u}}ber der realen Welt w{\"{a}}hre, die sich nun einmal konstant ver{\"{a}}ndert.

Daher wird das \hyperref[pubsub]{\textit{publish-subscripe pattern}} von \hyperref[nanomsg]{Nanomsg} eingesetzt. Mithilfe dieses sendet die Simulation / der Publisher
Informationen an die Suscriber / die Voter. Die Voter k{\"{o}}nnen diese Informationen dann an die Controller weitersenden und auf die
Steuerkommandos warten. Diese k{\"{o}}nnen dann ausgewertet und an die Simulation geschickt werden. Dieses Pattern f{\"{u}}hrt zu
einer zeitlichen Entkoppelung von Sender und Empf{\"{a}}nger, daher ist die Kommunikation event-gesteuert.

Desweiteren hat es auch den Vorteil das, wie in \hyperref[Anforderung]{Anforderung 2} gefordert, mehrere Instanzen gleichzeitig
laufen k{\"{o}}nnen. Mehrere Simulation k{\"{o}}nnen auf jeweils eigenem PC gestartet werden; die Voter verbinden sich dann mit "ihrer"
Simulation und kommen den anderen Instanzen nicht in die Quere.

Damit die Studenten sich nicht um die Kommunikation zur Simulation k{\"{u}}mmern m{\"{u}}ssen brauchen sie eine Bibliothek welche
diese Funktionalit{\"{a}}t bereitstellt. Sie kann benutzt werden um einen neuen Roboter zu erstellen und diesen zu steuern.
Im Fehlermodell gibt es auch eine Anforderung das einzelene Voter andere Informationen {\"{u}}ber den Status der Welt
bekommen sollen als andere. Daher ist die Bibliothek nicht nur f{\"{u}}r die reine Kommunikation zust{\"{a}}ndig, sondern wird auch
benutzt um diese Art von Fehler zu injizieren.

\subsection{Anzeige}\label{graphics}
F{\"{u}}r interessierte Sch{\"{u}}ler und auch f{\"{u}}r die Studenten, die diese Aufgabe l{\"{o}}sen
sollen muss es m{\"{o}}glich sein einzusch{\"{a}}tzen wie der momentane Status der Simulation ist.
Also: wo befinden sich Roboter, wie voll sind sie, wie stark kippt die Platte? All dies soll auch, wie
in \hyperref[anforderung]{Anforderung 5} festgelegt, visuell ansprechend sein.

Da die Unity Engine auch eine vollwertige Grafikengine ist (dies war ja schliesslich ein Kriterium),
kann diese Funktionalit{\"{a}}t genutzt werden. Das Simulationsprogramm ist nicht nur f{\"{u}}r die
reine Simulation sondern auch f{\"{u}}r die Anzeige verantwortlich.

Angezeigt wird eine kreisf{\"{o}}rmige Platte auf der sich die verschiedenen Roboter bewegen. Auch die
Ladestation wird angezeigt.

\paragraph{Kamera} Die Sicht auf die simulierte Welt geschieht durch das Unity Objekt "Camera". An diese
werden verschiedene Skripte angegeh{\"{a}}ngt, um z.B. dem Benutzer die M{\"{o}}glichkeit zu
geben die Kamera zu bewegen und damit einen anderen Teil der Simulation genauer zu betrachten.

Die Kamera kann {\"{u}}ber die Pfeiltasten bewegt werden, ein Kippen der Kamera ist nicht implementiert --
dies w{\"{u}}rde auch verwirren, da damit eventuell der Kippwinkel der Platte visuell ausgeglichen
werden k{\"{o}}nnte. Um immer einen Teil der Simulation im Blick haben zu k{\"{o}}nnen schwebt die Kamera
{\"{u}}ber der Platte und schaut in einem 20\textdegree Winkel schr{\"{a}}g nach unten.

\paragraph{Lichtquelle} Damit etwas sichtbar wird, muss es von Licht getroffen werden. In der Simulation
gibt es daher ein Spotlight. Dies befindet sich weit von der Platte entfernt und scheint in einem steilen Winkel
von links oben auf diese.

\paragraph{User Interface} Desweiteren wird dem Benutzer, in der Mitte, als Text angezeigt ob die
Fernsteuerungsfehlerinjektion an- oder ausgeschaltet ist. Das Radar oben links zeigt an wie stark die Platte
durch die Gewichte der Roboter und der Ladestation gekippt wurde und kann genutzt werden um einzusch{\"{a}}tzen
wie gut die Regelung ist.

\clearpage
\section{Fehlermodell}\label{fm}
Das Ziel der Aufgabe ist es, Studierenden etwas {\"{u}}ber ausfallsichere Systeme beizubringen. Daf{\"{u}}r
muss auch etwas ausfallen, denn nur dann wird getestet ob die Auswirkungen des Ausfalls abgemindert werden 
konnten.

Im Fehlermodell wird festgelegt, was alles -- und auf welche Weise -- ausfallen kann. Dabei wird f{\"{u}}r jeden Bestandteil der Simulation ein eigenes Fehlermodell aufgestellt. Die Bestandteile
der Simulation sind hier die simulierte Welt mit ihren Robotern (und den dazugeh{\"{o}}rigen Motoren/Aktoren
und Sensoren), die Kommunikationswege zwischen Voter und Controller und die Controller selbst.

\subsection{Roboter / Motoren}
Die simulierten Roboter {\"{a}}hneln dem Khepera Roboter und haben daher 2 Motoren. Diese beiden Motoren k{\"{o}}nnen unabh{\"{a}}ngig von einander fehlerhaft sein. Fehlerhaft
hei{\ss}t hier:
\begin{itemize}
	\item Die H{\"{a}}ufigkeit von Hardwarefehler wird oft mithilfe der Badewannenkurve modelliert. Sie besagt,
		dass Hardware entweder kurz nach Inbetriebnahme oder nach langer Betriebszeit mit hoher
		Wahrscheinlichkeit fehlerhaft ist, und in der Zwischenzeit nur mit niedriger Wahrscheinlichkeit.
		F{\"{u}}r die Motoren der Roboter wird ein vereinfachtes Modell genutzt, in dem die Wahrscheinlichkeit 
		f{\"{u}}r einen Ausfall bis zu einer bestimmten Betriebslaufzeit einstellbar ist und danach auf
		Null sinkt. Falls der Motor in dieser Zeit Fehler aufweist, bleibt diese Fehler auch.
	\item Desweiteren kann ein Motor tempor{\"{a}}r ein \textit{stuck-at}-Verhalten aufweisen. Dieses
		kann jederzeit passieren und entweder tempor{\"{a}}rer oder permament sein.
		Die Wahrscheinlichkeiten f{\"{u}}r das Auftreten eines permamenten oder tempor{\"{a}}ren
		\textit{stuck-at}-Fehlers sind unabh{\"{a}}ngig voneinander einstellbar.
\end{itemize}

Da jeder der \gls{N} Roboter 2 Motoren hat, k{\"{o}}nnten theoretisch auch 2 Motoren kaputt gehen. Ein Roboter mit einem fehlerhaften Motor kann allerdings immernoch sinnvoll genutzt werden,
bei einem Roboter mit 2 fehlerhaften Motoren ist dies nicht der Fall. Es soll m{\"{o}}glich sein verschiedene 
Schwierigkeitsstufen der Simulation einzustellen. Daher ist es m{\"{o}}glich
dass nur ein Motor pro Roboter gleichzeitig Fehler aufweisen kann.

Aus dem gleichen Grund ist es parametresierbar, wie viele Roboter gleichzeitig irgendeine Art von Fehler aufweisen k{\"{o}}nnen.

Zus{\"{a}}tzlich zu diesen Fehler soll es auch m{\"{o}}glich sein, einen Roboter fernzusteuern. W{\"{a}}hrend
der Fernsteuerung darf dieser Roboter keine Steuerbefehle seines Voters ausf{\"{u}}hrend sondern wird rein
vom Nuter gesteuert. Diese Art der Fehlerinjektion w{\"{u}}rde in Richtung byzantinische Fehler gehen,
denn dies w{\"{u}}rde es erm{\"{o}}glichen, gezielt die Platte zu destabilisieren oder Roboter zu
blockieren.

\subsection{Weltstatusinformationen}
Die Roboter haben ihre Aktoren, um sich in der Welt zu bewegen. Damit sie sich sinnvoll bewegen und die
Platte balanciert k{\"{o}}nnen, brauchen sie Informationen {\"{u}}ber den Status der Welt und ihre
(und die der anderen Roboter) Position in dieser. Diese Informationen sind direkt verf{\"{u}}gbar
und m{\"{u}}ssen nicht, zum Beispiel odeometrisch, erst herausgefunden werden. In der realen Welt sind
Sensoren, wie sie hier vielleicht h{\"{a}}tten simuliert werden k{\"{o}}nnen, nat{\"{u}}rlich manchmal 
fehlerbehaftet. Um dies nachzubilden k{\"{o}}nnen, die Informationen {\"{u}}ber den Weltstatus auf
verschiedene Arten fehlerhaft an den Roboter weitergeleitet werden.

Als erstes ist es m{\"{o}}glich, dass der Roboter einfach keinerlei Informationen mehr bekommt, also ein \textit{omission failure} vorliegt. Ein \textit{stuck-at} Fehler liegt
vor wenn veralterte Informationen wieder an den Roboter weitergegeben werden, so als w{\"{u}}rde ein Sensor nur noch seine letzen g{\"{u}}ltigen Daten liefern. Der dritte
Fall liegt vor, wenn komplett falsche Daten geliefert werden. Komplett falsch hei{\ss}t, dass die Anzahl und die Positionen, Gewichte etc. der Roboter potenziell fehlerhaft an
den Roboter weitergegeben werden. Die Wahrscheinlichkeiten f{\"{u}}r die jeweiligen F{\"{a}}lle lassen sich unabh{\"{a}}ngig von einander einstellen.

\subsection{Netwerk}
Da die simulierten Roboter den Khepera Robotern {\"{a}}hneln sollen, sollen auch ihre 
Kommunikationsm{\"{o}}glichkeiten denen der Khepera Roboter {\"{a}}hneln. Bei beweglichen
Objekten sind Ethernetkabel suboptimal, weswegen diese Funkverbindungen -- im Falle der Khepera
Roboter ist dies WLAN -- bevorzugen. Funkverbindungen haben aber, im Gegensatz zu gut abgeschirmten
Ethernetkabeln, Probleme mit Interferenzen durch andere Funk{\"{u}}bertragungen.

Dies f{\"{u}}hrt dazu das w{\"{a}}hrend der {\"{U}}bertragung Bits, oder sogar ganze Bytes, verf{\"{a}}lscht werden. Auch die Wahrscheinlichkeit f{\"{u}}r diesen Fall
soll einstellbar sein. Vereinfachend wird hier angenommen das jeweils nur ein Byte pro Paket verf{\"{a}}lscht wird, es gibt also keine \textit{burst errors}.
Es wird angenommen das die meiste Kommunikation zwischen Robotern "in Sichtweite" geschieht, ohne Zwischenstationen und komplexes Routing. In diesem Fall ist
es ist un{\"{u}}blich das Pakete verloren gehen oder zeitverz{\"{o}}gert weitergeleitet werden, daher ist dies nicht Teil des Fehlermodells.

\subsection{Controller}
Es wird davon ausgegangen, dass die Controllerprogramme auf unzuverl{\"{a}}ssiger Hardware, unzuverl{\"{a}}ssigen 
Betriebssystem laufen und fehlerhaft programmiert sind. Daher muss davon ausgegangen werden, dass die
Programme jederzeit abst{\"{u}}rzen k{\"{o}}nnen. Desweiteren kann auch nicht davon ausgegangen werden,
dass die Controllerprogramme das einzige Programm ist, das l{\"{a}}uft; es kann also auch zu Problemen bei
der Resourceverteilung (RAM, CPU Zeit) kommen.

\subsection{Auswirkungen}
Nachdem Fehlermodelle f{\"{u}}r die verschiedenen Elemente des Systems entwickelt wurden, kann
betrachtet werden, welche Auswirkungen die verschiedenen Fehler haben k{\"{o}}nnen. Hierbei werden nur
die Auswirkungen in Bezug auf die Platte betrachtet.

\paragraph{Unkontrollierte Bewegung} Falls sich ein Roboter unkontrolliert bewegt (also ein oder
mehrere Motoren einen \textit{stuck-at}-Fehler und die verbliebenden Motoren gar nicht mehr funktionieren)
kann dieser Roboter nicht mehr genutzt werden um die Platte auszubalancieren.
\begin{figure}
	\centering
	\includevisio[width=\textwidth]{fm_robot}
	\caption{Fehlerbaum: Unkontrollierte Bewegung eines Roboters}
	\label{fig:fault-tree-robot}
\end{figure}
\clearpage

\paragraph{Inkorrekte Bewegungen} Ein Roboter bewegt sich "falsch", wenn seine Bewegung weder die Platte 
ausbalanciert noch dazu gedacht, ist den Energiespeicher aufzuf{\"{u}}llen. Dies kann verschiedene Gr{\"{u}}nde 
haben: Entweder bewegt der Roboter sich unkontrolliert oder er wird falsch angesteuert. Beide F{\"{a}}lle
passieren falls die Controller falsche Steuerkommandos senden, oder theoretisch richtige Steuerkommandos
falsch ausgewertet werden.
\begin{figure}
	\centering
	\includevisio[width=\textwidth]{fm_robot2}
	\caption{Fehlerbaum: Inkorrekte Bewegung eines Roboter}
	\label{fig:fault-tree-robot2}
\end{figure}
\clearpage

\clearpage
\section{Fehlerinjektion}
Da es in der Simulation, anders in einer realen Welt, keine normalen Fehlerquellen gibt, wird eine
Fehlerinjektion benutzt, um die Fehlertoleranz zu testen. Die hier simulierten Fehlerarten 
sind im \hyperref[fm]{Fehlermodell} beschrieben.

In diesem Fall wird simuliert, dass das Netzwerk Daten fehlerhaft weiterleitet
(wie es z.B. ein Funknetzwerk tuen w{\"{u}}rde) und das die Controllerprogramme unvorhersehbar
ausfallen - was nachbildet, dass z.B. der Controller seine Batterie entladen hat. Auf der Roboterseite
kann die Funktionsf{\"{a}}higkeit der Motoren eingeschr{\"{a}}nkt werden oder sie k{\"{o}}nnen
falsche Informationen {\"{u}}ber den Status der Welt empfangen.

Wie in \hyperref[anforderung]{Anforderung 3} spezifiert, sind alle Fehlerwahrscheinlichkeiten einstellbar.

\subsection{Netzwerk}
Die Kommunikation zwischen Voter und Controller ist nur {\"{u}}ber ein IP Netzwerk m{\"{o}}glich;
f{\"{u}}r dieses Modul kann ein eigenes Netzwerk aufgebaut werden, in das eine Fehlerinjektionskomponente 
intergiert werden kann. Diese Komponente soll nur die Kommunikation, die von den Studentenprogrammen ausgeht, 
verf{\"{a}}lschen und alle andere normal behandeln. Die St{\"{a}}rke der Fehlerinjektion soll parametresierbar 
sein, zum Beispiel wie wahrscheinlich es ist, das, einzelne Pakete verf{\"{a}}lscht werden oder wie stark
die Verf{\"{a}}lschung sein soll.

Es wird vorgeschrieben, dass die ganze Kommunikation der Studentenprogramme {\"{u}}ber UDP laufen muss. Dadurch reicht es, wenn der Fehlerinjektor nur UDP verf{\"{a}}lscht, was dazu
f{\"{u}}hrt, dass die normalen Administrationstat{\"{a}}tigkeiten nicht beeintr{\"{a}}chtigt werden. Der Fehlerinjektor muss in der Lage sein, mit einstellbaren Wahrscheinlichkeiten
Pakete zu verf{\"{a}}lschen: bei den Arten der Verf{\"{a}}lschung ist es ausreichend, wenn einzelne Bytes gekippt werden.

Daf{\"{u}}r zust{\"{a}}ndig ist Net Inject\cite{kubertzki}. Dabei werden auf den fraglichen Rechnern
Standartrouten definiert, die den Netzwerkverkehr {\"{u}}ber andere Rechner leiten, auf denen Net Inject
installiert ist. Dort werden die Pakete vom Modul NetMod angenommen und es wird {\"{u}}berpr{\"{u}}ft ob
f{\"{u}}r diese Kommunikation (spezifiziert durch Protkoll, Quell- und Zielport) Verf{\"{a}}lschungsregeln
existieren. Anhand dieser Regeln wird das Paket dann ausgwertet und bei Bedarf verf{\"{a}}lscht.

Standardm{\"{a}}ssig ist dieser Fehlerinjektor so eingestellt dass durchschnittlich in $ ^1/_{20} $ aller Pakete ein Byte verf{\"{a}}lscht wird,
ohne dass das Betriebssystem diese Modifikation erkennt. Diese Erkennung muss also von den Studenten selbst implementiert werden.

\subsection{Controller}
Im Fehlermodell wurde spezifiziert, dass die Controller zu beliebigen Zeitpunkten einen
\textit{crash failure} erleiden k{\"{o}}nnen.

Um dies zu erm{\"{o}}glichen, gibt es ein Shellskript, das die Controllerprogramme zu zuf{\"{a}}ligen Zeitpunkten 
beendet und wieder startet - dabei wird das Programm einfach mit SIGKILL beendet und nicht vorgewarnt.
Mit einer Wahrscheinlichkeit von 33\% werden dem Programm auch nur eingeschr{\"{a}}nkte Ressourcen zugeteilt,
z.B. darf es nur eine bestimmte Anzahl an Dateideskriptoren gleichzeitig offen haben oder nur eine 
eingeschr{\"{a}}nkte Menge Speicher benutzen.
\lstinputlisting[language=Bash]{../fault_injector.sh}

\subsection{Simulation}
Auch innerhalb der Simulation soll es Fehler geben, zum Beispiel an den Robotermotoren. Desweiteren soll,
laut Fehlermodell, auch die Fernsteuerung eines Roboters m{\"{o}}glich sein.

Laut \hyperref[fm]{Fehlermodell} k{\"{o}}nnen die Robotermotoren auf verschiedene Arten fehlerbehaftet
sein und die Voter bekommen eventuell fehlerhafte, oder sogar gar keine, Informationen {\"{u}}ber
den Status der Welt. Die Wahrscheinlichkeiten f{\"{u}}r diese verschiedenen F{\"{a}}lle
werden in einer JSON Konfigurationsdatei angegeben. Die Simulation liest diese beim Start aus und 
l{\"{a}}sst dann w{\"{a}}hrend des laufenden Betriebes die Robotermotoren ausfallen.
Ein Teil der Informationen muss an den Voter weitergegeben werden, daher fragt dieser beim Start
automatisch die ben{\"{o}}tigten Daten ab.


\subsubsection{Weltstatus}
Die echten \hyperref[khepera]{Kheperaroboter} haben verschiedene Sensoren (Infrarot, Ultraschall,
odeometrische Motoren, eine Kamera, etc). In der echten Welt k{\"{o}}nnen auch diese Senoren fehlerhaft
sein. Allerdings werden in der Simulation diese Sensoren nicht simuliert, die Roboter kriegen alle
ben{\"{o}}tigten Informationen als Weltstatuspaket. Um die fehlerhafte Sensorik nachzuahmen, soll diese
Informationen abge{\"{a}}ndert werden.

Der Status der Welt der von der Simulation zum Voter geschickt wird kann, wie im \hyperref[fm]{Fehlermodell} angegeben, verf{\"{a}}scht werden, kann zeitweise gar
nicht ankommen oder es kann ein alter Status erneut versendet werden. Die Wahrscheinlichkeiten f{\"{u}}r diese drei F{\"{a}}lle werden direkt angegeben.
\begin{lstlisting}[frame=single, language=json] 
{
	"network" : {
		"dropWorldStatus": 0.001,
		"fakeWorldStatus": 0.001,
		"dupWorldStatus": 0.001
	}
}
\end{lstlisting}

\paragraph{dropWorldStatus} Der Status der Welt kann, aus Sicht des Roboters, einfach nicht ankommen. Dies
	ist ein \textit{omission fault}.
\paragraph{dupWorldStatus} Des weiteren kann ein veralteter Weltstatus ein weiteres Mal empfangen werden.
    Hierbei handelt es sich allerdings immer nur um den jeweils letzten und nicht um noch weiter in
	der Vergangenheit liegende.
\paragraph{fakeWorldStatus} Es ist auch m{\"{o}}glich, dass Daten zur richtigen Zeit ankommen, aber
    verf{\"{a}}lscht wurden. Verf{\"{a}}lscht werden kann hierbei:
	\begin{itemize}
		\item X- und Y-Kippwinkel. Dabei wird einer der Werte um maximal $\pm 2$\textdegree verf{\"{a}}lscht.
			Diese Art der Verf{\"{a}}lschung geschieht in jeweils 40\% der F{\"{a}}lle.
		\item Mit einer Wahrscheinlichkeit von 10\% werden die Daten maximal eines Roboters verf{\"{a}}lscht.
			Namentlich kann hierbei eine der folgenden Daten ver{\"{a}}ndert werden:
			\begin{itemize}
				\item X- oder Y-Koordinaten, um maximal $\pm 5$ Einheiten
				\item Die Rotation, um maximal $\pm 7$\textdegree
				\item Das Gewicht um maximal $\pm 5$ Einheiten
				\item Der Ladezustand um maximal $\pm 50$ Einheiten
			\end{itemize}
			Jede dieser Verf{\"{a}}lschungsarten ist gleich wahrscheinlich, sie liegen also bei 20\%.
		\item Die Daten der Ladestation, nach den gleichen Regeln wie f{\"{u}}r den Roboter
	\end{itemize}

\subsubsection{Motor}
Die zweite Kategorie von Fehlern sind die Ausf{\"{a}}lle der Motoren. Direkt angegeben werden die maximale
Anzahl an fehlerhaften Robotern und die maximale Anzahl an fehlerhaften Motoren pro Roboter.
Die einzelnen Motoren k{\"{o}}nnen entweder tempor{\"{a}}r oder permament fehlerbehaftet (\textit{stuck-at}
oder keinerlei Leitung) sein. Die Wahrscheinlichkeiten f{\"{u}}r diese beiden Fehlerarten werden getrennt
angegeben und k{\"{o}}nnen entweder einer zeitunabh{\"{a}}nigen festen Wahrscheinlichkeit oder der
Badewannenkurve folgen. Bei der Badewannenkurve wird zus{\"{a}}tzlich zur Ausfallwahrscheinlichkeit auch
noch die Dauer (in s) des ersten Bereiches angegeben.
\begin{lstlisting}[frame=single, language=json] 
{
	"robot" : {
		"breakEngineA" : {
			"perm": {"after": 120, "chance" : 0.01}
		},
		"stuckAtEngineA" : {
			"temp": 0.00001
		},
		"maxEnginesBroken": 3,
		"maxEnginesBrokenPerRobot": 1
	}
}
\end{lstlisting}

\todo{temporaer sind jetzt nur quasi einen frame lang...}

\subsubsection{Fernsteuerung}
Wie im \hyperref[fm]{Fehlermodell} festgelegt, soll der Benutzer einen Roboter fernsteuern k{\"{o}}nnen
und damit direkten Einfluss auf die simulierte Welt haben. W{\"{a}}hrend ein Roboter ferngesteuert wird, 
f{\"{u}}hrt dieser eine Roboter keine Befehle seines Voters mehr aus. Abgesehen davon verh{\"{a}}lt sich der
Roboter normal, es wird also die gleiche Menge Energie verbraucht, es gelten die gleichen
Einschr{\"{a}}nkungen bei Geschwindigkeit und Rotation, und so weiter.

Dies wird durch ein Script in Unity erm{\"{o}}glicht. Im deaktivierten Zustand zeigt es einen Hilfetext
auf dem Bildschirm an. Mit der Escapetaste wird die Fernsteuerung aktiviert; um dies zu signalisieren wird
der Hilfetext ge{\"{a}}ndert. Dann kann ein Roboter {\"{u}}ber die Pfeiltasten gesteuert werden
(Vorw{\"{a}}rts beschleunigt, Links und Rechts drehen den Roboter). Um anzuzeigen, welcher Roboter
gerade gesteuert wird wird dieser von oben angeleuchtet. Mit der Tabulatortaste kann zwischen den
Robotern umgeschaltet werden. Bei einem weiteren Druck auf die Escapetaste wird die Fernsteuerung deaktiviert.


\clearpage
\section{Die Simulation}
Die simulierte Welt besteht aus den Robotern, die von den Studierenden gesteuert werden sollen, einer Ladestation, an der die Robter Energie tanken k{\"{o}}nnen, und
der Welt, einer kreisf{\"{o}}rmigen, kippbaren Platten, auf der diese Objekte platziert werden und sich bewegen k{\"{o}}nnen.

Simuliert wird die Welt mit der Unity \textit{game engine}. Diese erm{\"{o}}glicht es plattformunabh{\"{a}}nige 
Spiele oder, in diesem Fall, Simulationen zu schreiben. Dabei stellt sie, unter anderem eine Physikengine,
eine Grafikengine und eine Schnitstelle zum Scripten dieser bereit. Desweiteren gibt es auch M{\"{o}}glichkeiten Benutzereingaben
abzufragen, um dem Benutzer die M{\"{o}}glichkeit zu geben mit der Simulation zu interagieren.

Hier werden die M{\"{o}}glichkeiten der objektorierntierten Programmierung genutzt. Jeder Bestandteil der Simulation wird durch
ein Skript abgebildet, diese Skripte k{\"{o}}nnen auch miteinander interagieren. Zum Beispiel benutzt die Robotersteuerungsklasse
die RPC Klasse, so wie dies auch die Fehlerinjektion tut. Genauso wurde die Vererbung genutzt, denn die Roboter und Ladestation haben
einige Eigenschaften gemeinsam (zum Beispiel Position und Gewicht), die eine Vererbungshierachie erm{\"{o}}glichen.

\subsection{Die Roboter}\label{robot}
In der simulierten Welt k{\"{o}}nnen sich bis zu \gls{N} Roboter bewegen. Diese bewegen sich aber nicht selbstst{\"{a}}ndig, sondern werden von den Controllern ferngesteurt.
Wie sie in der Simulation dargestellt werden, wird durch das grafische Modell bestimmt. Anhand dessen bestimmen sich auch die Dimensionen, diese werden f{\"{u}}r die Kollisionerkennung
gebraucht. Die Dimensionen, zusammen mit der Masse, ergeben das physische Modell; dieses hat Auswirkungen auf die Simulation.

\paragraph{Grafisches Modell} Mithilfe von Blender, einem 3D Designprogramm, wurde ein Robotermodell designt, das dem Kepheraroboter\ref{khepera} entspricht. Die Grundform des Roboters ist
eine S{\"{a}}ule. Auf dieser befindet sich eine Lampe die den Energielevel angezeigt; daf{\"{u}} {\"{a}}ndert sich ihre Farbe von Gr{\"{u}}n (voll), {\"{u}}ber Gelb bis Rot (leer) .
\todo{Bild}


\paragraph{Physikalisches Modell}
Ein Roboter \gls{Ni} wird dabei beschrieben durch seine Position und Gewicht
$ N_i = \bigl(\begin{smallmatrix} x(i) \\ y(i) \\ w(i) \end{smallmatrix}\bigr)$, eine
Geschwindigkeit $ V_i = \Delta v $ und den momentanen Drehwinkel
$ R_i = r_y(i)$. \todo{Ausdehnung}

Das Gewicht des Roboters ist abh{\"{a}}ngig vom Grundgewicht des Roboters und seinem momentanen F{\"{u}}llstatus: 
\begin{equation}\label{eq:w}
 w(N_i) = 1 + e(N_i) * 0.03
\end{equation}

Die Roboter haben einen Energiespeicher, der mit maximal 1000 Energieeinheiten
aufgeladen werden kann, und verbrauchen diese Energie, ob beim Fahren oder
Stillstand. Dabei verbrauchen sie pro Runde ohne Bewegung immer eine Energieeinheit und zus{\"{a}}tzlich, abh{\"{a}}ngig von der Geschwindigkeit, Energie f{\"{u}}r die Bewegung:
\begin{equation}\label{eq:entladen}
	e(N_i, t + 1) = e(N_i, t) - 1 - |V_i|
\end{equation}
Falls nicht gen{\"{u}}gend Energie f{\"{u}}r Bewegung und Rotation vorhanden ist, bewegt sich der Roboter nicht.

Die Bewegung des Roboters wird vorgegeben durch $V_l(t)$ und $ V_r(t)$. Diese werden vom Controller, {\"{u}}ber den Voter an die Simulation weitergegeben und dann von Unity verarbeitet.
Die neue Position und der neue Rotationswinkel des Roboters werden errechnet und, falls gen{\"{u}}gen Energie
vorhanden, auch eingenommen - dazu wird der \textit{rigidbody} des Roboters manipuliert.
Um die Bewegung fl{\"{u}}{\ss}iger darzustellen wird zwischen den momentanen und gew{\"{u}}nschten Positionen / Winkel interpoliert.

\subsection{Die Ladestation}\label{fuelstation}
Innerhalb der Welt muss eine Ladestation platziert werden, um den Roboter die M{\"{o}}glichkeit zu geben sich aufzuladen. Auch diese wird durch ihren Vektor $ F = \bigl(\begin{smallmatrix} x \\ y \\ w \end{smallmatrix}\bigr)$ beschrieben. Eine Ladestation hat dabei ein festes Gewicht: $ w(F) = 5 $.

Diese wird vor Simulationsbeginn platziert und bewegt sich im weiteren Verlauf nicht.
Falls sich ein Roboter an die Ladestation heranbewegt, also gilt: 
\begin{equation}\label{eq:dist}
 |\bigl(\begin{smallmatrix} x(i) \\ y(i) \end{smallmatrix}\bigr) - \bigl(\begin{smallmatrix} x(F) \\ y(F) \end{smallmatrix}\bigr)| \leq |\bigl(\begin{smallmatrix} 1 \\ 1 \end{smallmatrix}\bigr)|
\end{equation}
wird dieser Roboter aufgeladen. Die Ladefunktion ~\ref{eq:laden} ist hier eine einfache Gerade:
\begin{equation}
    \label{eq:laden}
	e(N_i, t + 1) = max((e(N_i, t) + 10, 1000) 
\end{equation}

\subsection{Die Platte}\label{plate}
Die simulierte Welt besteht aus einer 100 Einheiten gro{\ss}en kreisf{\"{o}}rmigen Platte, die, basierend auf den Gewichten welche sich auf ihr befinden kippt.
XXXX \todo{} text, iwie auch auf das da oben beziehen

\subsection{Benutzerinterface}

\subsection{Netzwerkschnittstelle}

\clearpage
\section{Interface f{\"{u}}r die Studenten}\label{interface}
Damit die Studenten sich auf die Implementierung der Fehlertoleranz konzentrieren k{\"{o}}nnen, gibt es Schnittstellen.
Im ganzen System gibt es zwei Schnittstellen:
\begin{itemize}
\item Die Schnittstelle zwischen Controller und Voter
\item Die Schnittstelle zwischen Voter und Roboter/Simulation
\end{itemize}

\paragraph{Die Schnittstelle zwischen Controller und Voter} Den Studenten wird nicht vorgegeben wie die Kommunikation zwischen den Controllern und Votern aussehen soll - denn gerade hier soll ja die Fehlertoleranz implementiert werden.

\paragraph{Die Schnittstelle zwischen Voter und Roboter} Diese Schnittstelle besteht aus den Funktionen:
\begin{lstlisting}[frame=single, language=c] 
void* connectToWorld();
void detachFromWorld(void* ctx);
int createRobot(void* ctx);
typedef void (*TypeGetWorldStatusCallback)(WorldStatus ws, void* optional);
int startProcessingWorldEvents(void* ctx, TypeGetWorldStatusCallback cb, void* optional);
void moveRobot(void* ctx, int id, float vL, float vR);
\end{lstlisting}

Diese wird den Studierenden als kompilierte Library mit einem detailliert kommentierten Headerfile zur Verf{\"{u}}gung gestellt und kann
dann vom Studentencode einfach aufgerufen werden.


\clearpage
\section{Beispielimplementation}
Um das Prinzip dieser Simulation, ob den Studierenden des Moduls Ausfallsichere Systeme oder Besuchern, zu verdeutlichen, ist ein Teil der Bachelorarbeit die Implementierung einer
Beispielimplementation.

\paragraph{Ablauf} Jede Viertelsekunde sendet die Welt Statusinformationen aus. Diese Informationen werden von den Votern empfangen, k{\"{o}}nnen
aber auf dem {\"{U}}bertragungsweg zum Controller potenziel verf{\"{a}}lscht worden sein. Also wird f{\"{u}}r jedes Objekt der Welt der Konsensalgorithmus ausgef{\"{u}}hrt. Wenn ein Konsens {\"{u}ber den Status der Welt hergestellt wurde, berechnet 
jeder Controller f{\"{u}}r jeden Roboter die n{\"{a}}chste Bewegung. Diese wird dann an den Voter geschickt, der ein einfaches Mehrheitsvotum durchf{\"{u}}hrt und diese Bewegung an die Welt weitergibt.

\subsection{Fehlermodell} \label{error-model}
Bei der Planung eines ausfallsicheren Systems ist es besonders wichtig zu definieren, welche Art von Fehlern
{\"{u}}berhaupt korrigiert / abgefangen werden soll. F{\"{u}}r die Beispielimplementation ist es nicht
n{\"{o}}tig eine Regelung zur Ausbalancierung zu entwickelen. Daher werden nur f{\"{u}}r bestimmte Fehlerklassen
die m{\"{o}}glichen Fehler aufgelistet und beschrieben, ob und wie sie gel{\"{o}}st werden.

\paragraph{Crash failure} Die Controller k{\"{o}}nnen jederzeit ausfallen. Im Extremfall k{\"{o}}nen alle Controller ausfallen, die Voter sind, per Definition, gegen diese Art von Fehlern unempfindlich.

\paragraph{Value error} Das Netzwerk kann Pakete verf{\"{a}}lschen, es wird angenommen, dass bei bis zu
$^1/_{20}$ aller Pakete Verf{\"{a}}lschungen geben kann, diese sich allerdings auf
\textit{Single Byte Errors} beschr{\"{a}}nkt. Dar{\"{u}}ber hinausgehende Verf{\"{a}}lschungen werden nicht
erkannt und f{\"{u}}hren zu einem \textit{silent failure}; Dadurch wird der Roboter fehlerhaft angesteuert,
dies ist allerdings nicht mehr Teil des Fehlermodells.

\subsubsection{R{\"{a}}umliche Redudanz}
Es wird davon ausgegangen, dass die Controller sehr fehleranf{\"{a}}llig sind und leicht ausfallen --
daraus folgt das eine gro{\ss}e Anzahl ($Y$) an Controllern f{\"{u}}r jeden Roboter kontrollieren muss;
nur falls weniger als $X$ Controller dieses Roboter noch aktiv sind kann eine ordnungsgem{\"{a}}{\ss}e
Steuerung nicht mehr garantiert werden. Zu bestimmen wie gro{\ss} $X$ sein muss ist nicht Teil der
Beispielimplementation.

\subsubsection{Netzwerkkommunikation}
Da durch die Fehlerinjektion das Netzwerk UDP Pakete verf{\"{a}}lscht, m{\"{u}}ssen alle Daten mit einer 
Kanalkodierung versehen werden. Hier wird ein (255, 240) Reed-Solomon Code benutzt, also ein Code, der 15 parity 
bits pro 240 Datenbits benutzt. Durch die Benutzung dieser Kodierung k{\"{o}}nnen alle Einzelfehler und 
Doppelfehler erkannt und korrigiert werden. Erst ab 7 Fehlern ist eine Korrektur nicht mehr m{\"{o}}glich, ab 15 
Fehlern versagt auch eine Fehlererkennung.

Diese Kodierungsart wurde aus zwei Gr{\"{u}}nden gew{\"{a}}hlt:
\begin{itemize}
\item Durch expermientelle Verifikation wurde klar, dass im Netzwerk nur 1 Byte pro Paket verf{\"{a}}lscht wird.
	Da das durchschnittliche Paket um mindestens Faktor 100 gr{\"{o}}{\ss}er ist, ist nicht erforderlich,
	eine sehr kompakte Kodierung zu benutzen, es ist ausreichend, die Redudanz zu reduzieren. 
\item Es ist ein systematischer Code, der es erlaubt, w{\"{a}}hrend des laufenden Betriebes die Pakete 
	mitzuschneiden und sich die Daten anzugucken. Dies vereinfacht die Fehlersuche -- beispielsweise bei
	einem Viterbicode w{\"{a}}re dies nicht m{\"{o}}glich, dort sind Nutz- und Kodierungsdaten nicht klar 
	unterscheidbar.
\end{itemize}

\subsection{Voter}
\label{voter}
Der Roboter kann nur direkt durch den Voter gesteuert werden. Der Voter ist dabei nur daf{\"{u}}r
zust{\"{a}}ndig, aus den vielen Steuerkommandos, die ankommen, die Mehrheit zu bilden (\textit{N-modular reduancy}).
Da nicht klar ist, wie viele Controller momentan {\"{u}}berhaupt Steuerkommandos senden k{\"{o}}nnten,
kann nicht gewartet werden bis eine bestimmte Anzahl von Kommandos empfangen wurde. Deswegen wird beim Empfangen
eines neuen Weltstatus (also zu periodisch wiederkehrenden Zeitpunkten) das Steuerergebniss gebildet,
weggesendet und der Vorgang wird neu gestartet.

Um das Steuerergebniss zu finden werden die empfangenen Steuerkommandos sortiert und der Medianwert genommen 
(\textit{Mid-Value Selection}). Anstatt des Mittelwerts wird der Medians genutzt weil dadurch weniger "extreme"
Bewegungen bevorzugt werden.
\noindent\begin{minipage}{.30\textwidth}
	\begin{lstlisting}[caption=Sammeln, frame=tlrb, language={[11]C++}]
votes.res[id].push_back(Vector{x, y});
\end{lstlisting}
\end{minipage}\hfill
\begin{minipage}{.60\textwidth}
\begin{lstlisting}[caption=Auswahl, frame=tlrb, language={[11]C++}]
/* go through all robots */
for(auto&& votes : votes.res) {
  /* vote */
  std::sort(std::begin(votes.second), std::end(votes.second));
  auto x = votes.second[votes.second.size() / 2];

  /* send */
  int r = 0;
  if((r = moveRobot(info->worldCtx, votes.first, x.x_, x.y_)) < 0) {
  	fprintf(stderr, "can't move robot: %d", r);
  }
}
\end{lstlisting}
\end{minipage}


\begin{figure}
	\centering
	\includevisio[width=\textwidth]{seqvoter}
	\caption{Ablaufdiagramm Voter}
	\label{fig:sequence-voter}
\end{figure}
\clearpage % make sure the table is, at least, in the right section

\subsection{Controller}\label{controller}
Die Roboter m{\"{u}}ssen sich so bewegen, dass die Platte m{\"{o}}glichst gut ausbalanciert
ist und gleichzeitig die Roboter nicht ihre ganze Tankf{\"{u}}llung verbrauchen. Gleichzeitig sollten
die Roboter auch nicht miteinander kollidieren.

\paragraph{Algorithmus zur Bestimmung der Bewegung} Der implementierte Algorithmus unterschiedet zwischen 3 Situation: ist der Zustand des Roboters kritisch,
unkritisch oder dazwischen? Dabei ist das Kriterium die momentane Tankf{\"{u}}llung geteilt durch die Entfernung zur Ladestation. Roboter mit fast leerem Tank
versuchen also sich aufzuf{\"{u}}llen, Roboter mit fast vollem Tank versuchen auszubalancieren.

Jeder Controller berechnet f{\"{u}}r jeden Roboter wie dieser sich bewegen soll. Dabei wird mit dem Roboter
im kritischtesten Zustand anfangen, denn dieser muss ja unbedingt Richtung Ladestation gesteuert werden 
und die weniger kritischen k{\"{o}}nnen/m{\"{u}}ssen diese Bewegung dann ausgleichen. Daf{\"{u}}r werden nach
jeder berechneten Roboterbewegung die Weltkippwinkel neu berechnet, so dass die vollsten Roboter die Bewegungen
der vorher berechneten mit in ihre Entscheidung einbeziehen.

\paragraph{Bewegung zu einem Punkt} Um zu einem bestimmten Punkt hinzufahren (z.B. zur Ladestation),
wird zu erst die Rotation des Roboters in einen Bewegungsvektor umgerechnet. Die Rotation wird immer in Bezug
auf eine feste Richtung angegeben, also kann der Bewegungsvektor durch trigonometrische Funktionen bestimmt werden:
\begin{equation}
\label{eq:M}
	M_i = \bigl(\begin{smallmatrix} sin(R_i) \\ cos(R_i) \end{smallmatrix}\bigr)
\end{equation}
Die n{\"{a}}chste Bewegung gerade nach vorne wird dann anhand dieser Werte "aufgeteilt". Falls der Roboter
in einem 90\textdegree Winkel nach rechts "guckt" ($M_i = \bigl(\begin{smallmatrix} 0 \\ 1 \end{smallmatrix}\bigr)$)
w{\"{u}}rde er sich beim nach vorne fahren auf der Y-Achse nach rechts bewegen.

Die Distanz zum Ziel l{\"{a}}sst sich trivial durch Vektorsubtraktion berechnen: $ D_i = \gls{F} - P_i $

Der Winkel zwischen diesen beiden Vektoren gibt an, wie weit der Roboter sich drehen muss um gerade zu dem Ziel 
fahren zu k{\"{o}}nnen. Dabei muss beachtet werden, dass nur der Wert des Winkels berechnet wird, das
Vorzeichen allerdings nicht. Der Roboter w{\"{u}}rde also immer (egal ob das Ziel links oder rechts liegt) nach 
rechts drehen, damit den Winkel vergr{\"{o}}{\ss}ern und noch st{\"{a}}rker in die falsche Richtung drehen.
Das Vorzeichen kann allerdings berechnet werden, in dem man die Bewegungsvektoren in drei Dimensionen betrachtet 
und das Keuzprodukt bildet. Dabei gibt das Vorzeichen der Z Kompenente des resultierenden Vektors das Vorzeichen 
des Winkels an.

Um nun zu einem Punkt zu fahren wird sich anhand des Drehwinkels gedreht und entsprechend des Distanzvektores 
bewegt. Da Rotation und Translation gleichzeitig ausgef{\"{u}}hrt werden, f{\"{a}}hrt der Roboter eine Kurve.
\footnote{Es w{\"{a}}hre auch m{\"{o}}glich erst die Rotation, dann die Translation durchzuf{\"{u}}hren}
Da sich der Winkel stetig verkleinert, dreht sich der Roboter immer weniger. Die Geschwindigkeit wird nicht
angepasst.

\paragraph{Ausbalancieren} Auch das ausbalancieren kann als Bewegung zu einem Punkt aufgefasst werden.
Die Platte ist ja durch die Gewichte der Roboter aus dem Gleichgewicht gebracht worden, um sie wieder ins
Gleichgewicht zu bringen k{\"{o}}nnten sich die einzelnen Roboter ja an den jeweiligen Punkt bewegen der
die Platte ins Gleichgewicht bringen w{\"{u}}rde.

Dazu wird zuerst der Kippwinkel der Platte ohne den momentan betrachteten Roboter berechnet. Anhand dessen
kann bestimmt werden an welcher Position der momentan betrachtete Roboter sein m{\"{u}}sste um die Platte
ins Lot zu bringen. Dorthin bewegt der Roboter sich nun, stoppt allerdings am Rande der Platte um nicht
herunterzufallen -- schlie{\ss}lich kann es sein das dieser Roboter sich von der Platte bewegen m{\"{u}}sste
um die Platte auszubalancieren.

\clearpage
\section{Evaluation}
nochmal die anforderungen durchgehen
eigentliches ziel ist lehrmittel, weil studenten erst demnaechst anfangen selbst implementieren


