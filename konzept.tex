\clearpage
\section{Simulationsidee}
Es wurde sich entschieden die Grundidee der Simulation beizubehalten, diese allerdings neu zu implementieren um die Kritikpunkte am
bestehenden System auszumerzen. Im folgenden wird die Grundidee noch einmal konkretisiert:

Die simulierte Welt besteht aus einer kreisf{\"{o}}rmigen Platte. Auf dieser k{\"{o}}nnen bis zu \gls{N}
Roboter und $1$ F{\"{u}}llstation (\gls{F}) platziert werden. Die Summe alle Gewichte kippt die Platte;
sind die Gewichte zu ungleichm{\"{a}}{\ss}ig verteilt, kippt die Platte um und die Simulation endet.

Die Roboter k{\"{o}}nnen sich frei auf der Platte bewegen, verbrauchen dabei aber Energie und m{\"{u}}ssen diese dann irgendwann
an der F{\"{u}}llstation auff{\"{u}}llen. Auch wenn die Roboter sich nicht bewegen, verbrauchen sie Energie (allerdings weniger
schnell).

Gesteuert werden die Roboter von bis zu \gls{M} $(M \geq N)$ Controllern. Einer oder mehr Controller stimmen sich ab,
so dass:
\begin{itemize}
\item die Roboter ihre Energie immer rechtzeitig auff{\"{u}}llen
\item die Roboter nicht kollidieren
\item die Platte nicht kippt
\end{itemize}

Dieser Abstimmungsvorgang sollte fehlertolerant implementiert werden, so dass auch beim Ausfall eines (oder mehrerer) Controller die Roboter sich immer noch koordiniert bewegen.
(Abh{\"{a}}ngig vom Verh{\"{a}}ltnis der Anzahl der Controller zur Anzahl der Roboter k{\"{o}}nnen verschiedene Stufen der Fehlertoleranz erreicht werden.\cite[s.149]{Werner00})
Daraus folgt das die Studierenden in ihrer L{\"{o}}sung eine Konsensbildung implementieren m{\"{u}}ssen.

Die Voter und Controller laufen auf Virtuellen Maschinen, jeweils eine VM pro Voter- oder Controllerinstanz.
Diese VMs befinden sich in einem Netzwerk, das durch eine Fehlerinjektion UDP Pakete verf{\"{a}}lscht --
dies simuliert ein fehlertr{\"{a}}chtiges Netzwerk, wie es zum Beispiel bei Funknetzwerken der Fall ist.
Es wird vorgegeben, dass die gesamte Kommunikation der Studentenprogramme {\"{u}}ber UDP stattfinden muss.
Dies f{\"{u}}hrt dazu das die Studierenden Gegenmassnahmen gegen die Verf{\"{a}}lschung ergreifen m{\"{u}}ssen, denkbar
w{\"{a}}hre hier zum Beispiel eine Kanalkodierung zur Vorw{\"{a}}rtsfehlerkorrektur.
Die Kommunikation von Simulation zu den Votern ist nicht betroffen, sie soll nicht verf{\"{a}}lscht werden. (Statdessen
kann die Schnittstelle gezieltere Verf{\"{a}}lschungen vornehmen.)

\begin{figure}
	\centering
	\includevisio[width=\textwidth]{Netzwerk}
	\caption{Aufteilung der Netzwerkteilnehmer}
	\label{fig:network}
\end{figure}

Zusammenfassend besteht das ganze System aus diesen Teilen:
\begin{itemize}
	\item Die Simulation berechnet die Bewegungen und implementiert das physikalische Modell. Sie {\"{u}}bernimmt auch die Visualisierung. Per Definition ist sie ausfallsicher.
	\item Der Voter sammelt die Steuerkommandos von den Controllern, bildet daraus das Steuerergebniss und sendet dies an die Simulation. Er ist per Definition ausfallsicher gegen \textit{crash failures}.
	\item Die Controller generieren Steuerkommandos. Sie k{\"{o}}nnen jederzeit ausfallen.
\end{itemize}

\paragraph{Nutzung durch die Studenten} Von den Studierenden muss nicht die gesamte Simulation erstellt werden;
die {\"{U}}bungsaufgabe bezieht sich nur auf die Abstimmung zwischen den Controllern und die Ansteuerung der Roboter.
Daher ist es notwendig, ein Interface ins "Innere" der Simulation bereitzustellen, das dann von den Studenten genutzt wird.

\clearpage
\section{Architekturkonzept}
Nach dem ausarbeiten der Anforderungen an das Simulationswerkzeug und die Entscheidung {\"{u}}ber den Simulationsablauf (inklusive
der Trennung zwischen f{\"{u}}r die Studierenden vorgefertigten Teil und den von diesen selbst zu implementierenden) ist es n{\"{o}}tig
die Systembestandteile aufzulisten. Dabei werden auch die Beziehungen zwischen ihnen aufgef{\"{u}}hrt, da diese Einfluss auf die
Architektur haben.

\begin{figure}
	\centering
	\includevisio[width=\textwidth]{Komponenten1}
	\caption{Ben{\"{o}}tigte Systembestandteile und ihre Beziehungen}
	\label{fig:arch_pre}
\end{figure}

Der Hauptbestandteil der Simulation ist die Physiksimulation. Sie ist daf{\"{u}}r zust{\"{a}}ndig
das sich die Roboter bewegen, betankt werden k{\"{o}}nnen, muss Informationen {\"{u}}ber den
Status der Simulation (z.B. Roboterpositionen) bereitstellen und auch die Roboter basierend
auf den Steuerbefehlen der Voter bewegen (unter Beachtung aller physikalischen Eigenschaften
des Roboters und der Fehlerinjektion).

Um die Roboter zu steuern muss a) der Status der Roboter bekannt sein b) es f{\"{u}}r den
Voter m{\"{o}}glich sein die Steuerbefehle an die Simulation weiter zu reichen. Beides
erm{\"{o}}glicht das Interface.

F{\"{u}}r die Fehlerinjektion ist der Fehlerinjektionssmanager zus{\"{a}}ndig. Die Physiksimulation
benutzt diesen um Roboterspezifische Fehler\ref{fm-robot} zu injezieren, w{\"{a}}hrend das
Interface basierend auf dem Fehlerinjektionsmanager die Informationen {\"{u}}ber den
Status der Welt verf{\"{a}}lscht.

Um den Simulationsstatus verfolgen zu k{\"{o}}nnen ist eine visuelle Anzeige hilfreich. Daher
ist ein Bestandteil der Simulation ein Anzeigeprogramm.

Jede Voterinstanz ist daf{\"{u}}r zust{\"{a}}ndig ihren Roboter in der Simulation zu steuern.
Daf{\"{u}}r m{\"{u}}ssen die Steuerkommandos an die Physiksimulation weitergeleitet werden,
dies {\"{u}}bernimmt das Interface. Der Controller ist komplett den Studenten {\"{u}}berlassen ---
Es gibt nur die Einschr{\"{a}}nkung das alle Kommunikation zur Simulation hin, {\"{u}}ber den Voter erfolgen muss.

\clearpage
\section{Kommunikation zwischen den Systemteilen}
Bei verteilten Systemen (wie diesem) ist einer der wichtigsten Punkte wie die Kommunikation zwischen den Systemteilen abl{\"{a}}uft.
Hierbei kann zwischen zwei Gruppierungen unterschieden werden: einmal die Kommunikation zwischen Simulation und Voter und andererseits
die Kommunikationen zwischen den Controllern und zwischen Controller und Voter. Da die letztere von den Studierenden geplant werden soll,
wird sie hier nicht weiter untersucht.

Da alle Voter in der gleichen Welt miteinander interagieren sollen, ist es am einfachsten wenn die Simulation ein Server ist, zu dem sich alle
verbinden. Dieser Server kann dann intern die Simulation vorantreiben und nur die Ergebnisse dieser weiterversenden. Die Alternative
w{\"{a}}re ein System ohne zentrale Instanz in dem die Simulation auf jedem Robter ausgef{\"{u}}hrt wird. In diesem Fall gibt es
allerdings das Problem das sich der Status der Simulationen voneinander desychnorisieren kann, zum Beispiel wenn ein Voter
eine schw{\"{a}}cheren CPU hat, die l{\"{a}}nger zur Berechnung eines Simulationstaktes braucht als die anderen Voter. Daher ist
das Modell mit zentrale Instanz vorzuziehen; es stellt automatisch sicher das alle Voter (im Normalfall, also ohne Fehler)
die gleichen Informationen kriegen und anhand dieser dann reagieren.

Durch die Verwendung einer zentralen Instanz ist die eine H{\"{a}}lfte der Isolation zwischen verschiedenen Instanzen realisiert:
Die Voter jeder Instanz verbinden sich mit ihrer Instanz, diese sendet ihre Informationen nur an die verbundenen Voter.
\footnote{Der andere Teil --- die Isolation zwischen verschiedenen Instanzen in Bezug auf Controller und Voter --- kann entweder
durch die Benutzung verschiedener Ports oder den Gebrauch mehrere \textit{subnets} erreicht werden.}

Ein Nachteil dieser L{\"{o}}sung ist das sie einen \textit{single point of failure} erschafft. Fehler in dem Simulationsprogramm
wirken sich direkt auf das ganze System aus; daher wird definiert das dieses keine Fehler enth{\"{a}}hlt.

F{\"{u}}r den zeitlichen Ablauf in der Simulation sind zwei M{\"{o}}glichkeiten denkbar: eine rundenbasierte Abfolge oder eine
Abfolge in Echtzeit. Bei der rundenbasieren Simulationstrategie wird der Status der Simulation immer nach festgelegten Intervallen
ver{\"{a}}ndert. In diesem Fall w{\"{a}}re es denkbar das die Controller und Voter 250ms Zeit haben ihre Bewegungsvorschl{\"{a}}ge
zu berechnen und diese an die Simulation schicken. Diese sammelt sie, wendet sie alle auf einmal an und versendet den neuen Status der
Welt. Damit beginnt die n{\"{a}}chste Runde. Die Alternative ist es das Bewegungsvorschl{\"{a}}ge direkt angewendet werden, falls
der Abstimmungsvorgang zwischen den Controllern also schneller vonstatten geht bewegen sich die Roboter auch {\"{o}}fter.

Beide M{\"{o}}glichkeiten w{\"{a}}hren realisierbar, allerdings wird die Echtzeitmethode vorgezogen. Einerseits da, gerade bei einer
3D-Ansicht, es unangehnem auff{\"{a}}llt wenn auch nur wenige Millisekunden nichts passiert --- es wirkt als w{\"{u}}rde die
Simulation stocken. Desweiteren ist es auch eine unrealistische Vereinfachung gegen{\"{u}}ber der realen Welt w{\"{a}}hre,
die sich nun einmal kontinuierlich ver{\"{a}}ndert.

Dies hei{\ss} zwangsl{\"{a}}ufig das eine Event-gesteuerten Kommunikation eingesetzt werden muss. Sender und Empf{\"{a}}nger
sollen zeitlich entkoppelt sein, haupts{\"{a}}chlich ihre eigenen Aufgaben durchf{\"{u}}hren und erst zu "passenden" Zeitpunkten
mit den jeweils anderen Kommunikationspartnern interagieren. 

Damit die Studenten sich nicht um die Kommunikation zur Simulation k{\"{u}}mmern m{\"{u}}ssen brauchen sie eine Bibliothek welche
diese Funktionalit{\"{a}}t bereitstellt. Sie kann benutzt werden um einen neuen Roboter zu erstellen und diesen zu steuern.
Im Fehlermodell gibt es auch eine Anforderung das einzelene Voter andere Informationen {\"{u}}ber den Status der Welt
bekommen sollen als andere. Daher ist die Bibliothek nicht nur f{\"{u}}r die reine Kommunikation zust{\"{a}}ndig, sondern wird auch
benutzt um diese Art von Fehler zu injizieren.

\clearpage
\section{Das zu simulierende Fehlermodell}\label{fm}
Das Ziel der Aufgabe ist es, Studierenden etwas {\"{u}}ber ausfallsichere Systeme beizubringen. Daf{\"{u}}r
muss auch etwas ausfallen, denn nur dann wird getestet ob die Auswirkungen des Ausfalls abgemindert werden 
konnten.

Im Fehlermodell wird festgelegt, was alles -- und auf welche Weise -- ausfallen kann. Dabei wird f{\"{u}}r jeden Bestandteil der Simulation ein eigenes Fehlermodell aufgestellt. Die Bestandteile
der Simulation sind hier die simulierte Welt mit ihren Robotern (und den dazugeh{\"{o}}rigen Motoren/Aktoren
und Sensoren), die Kommunikationswege zwischen Voter und Controller und die Controller selbst.

\subsection{Roboter / Motoren}\label{fm-robot}
Die simulierten Roboter {\"{a}}hneln dem Khepera Roboter und haben daher 2 Motoren. Diese beiden Motoren k{\"{o}}nnen unabh{\"{a}}ngig von einander fehlerhaft sein. Fehlerhaft
hei{\ss}t hier:
\begin{itemize}
	\item Die H{\"{a}}ufigkeit von Hardwarefehler wird oft mithilfe der Badewannenkurve modelliert. Sie besagt,
		dass Hardware entweder kurz nach Inbetriebnahme oder nach langer Betriebszeit mit hoher
		Wahrscheinlichkeit fehlerhaft ist, und in der Zwischenzeit nur mit niedriger Wahrscheinlichkeit.
		F{\"{u}}r die Motoren der Roboter wird ein vereinfachtes Modell genutzt, in dem die Wahrscheinlichkeit 
		f{\"{u}}r einen Ausfall bis zu einer bestimmten Betriebslaufzeit einstellbar ist und danach auf
		Null sinkt. Falls der Motor in dieser Zeit Fehler aufweist, bleibt diese Fehler auch.
	\item Desweiteren kann ein Motor tempor{\"{a}}r ein \textit{stuck-at}-Verhalten aufweisen. Dieses
		kann jederzeit passieren und entweder tempor{\"{a}}rer oder permament sein.
		Die Wahrscheinlichkeiten f{\"{u}}r das Auftreten eines permamenten oder tempor{\"{a}}ren
		\textit{stuck-at}-Fehlers sind unabh{\"{a}}ngig voneinander einstellbar.
\end{itemize}

Da jeder der \gls{N} Roboter 2 Motoren hat, k{\"{o}}nnten theoretisch auch 2 Motoren kaputt gehen. Ein Roboter mit einem fehlerhaften Motor kann allerdings immernoch sinnvoll genutzt werden,
bei einem Roboter mit 2 fehlerhaften Motoren ist dies nicht der Fall. Es soll m{\"{o}}glich sein verschiedene 
Schwierigkeitsstufen der Simulation einzustellen. Daher ist es m{\"{o}}glich
dass nur ein Motor pro Roboter gleichzeitig Fehler aufweisen kann.

Aus dem gleichen Grund ist es parametresierbar, wie viele Roboter gleichzeitig irgendeine Art von Fehler aufweisen k{\"{o}}nnen.

Zus{\"{a}}tzlich zu diesen Fehler soll es auch m{\"{o}}glich sein, einen Roboter fernzusteuern. W{\"{a}}hrend
der Fernsteuerung darf dieser Roboter keine Steuerbefehle seines Voters ausf{\"{u}}hrend sondern wird rein
vom Nuter gesteuert. Diese Art der Fehlerinjektion w{\"{u}}rde in Richtung byzantinische Fehler gehen,
denn dies w{\"{u}}rde es erm{\"{o}}glichen, gezielt die Platte zu destabilisieren oder Roboter zu
blockieren.

\subsection{Weltstatusinformationen}
Die Roboter haben ihre Aktoren, um sich in der Welt zu bewegen. Damit sie sich sinnvoll bewegen und die
Platte balanciert k{\"{o}}nnen, brauchen sie Informationen {\"{u}}ber den Status der Welt und ihre
(und die der anderen Roboter) Position in dieser. Diese Informationen sind direkt verf{\"{u}}gbar
und m{\"{u}}ssen nicht, zum Beispiel odeometrisch, erst herausgefunden werden. In der realen Welt sind
Sensoren, wie sie hier vielleicht h{\"{a}}tten simuliert werden k{\"{o}}nnen, nat{\"{u}}rlich manchmal 
fehlerbehaftet. Um dies nachzubilden k{\"{o}}nnen, die Informationen {\"{u}}ber den Weltstatus auf
verschiedene Arten fehlerhaft an den Roboter weitergeleitet werden.

Als erstes ist es m{\"{o}}glich, dass der Roboter einfach keinerlei Informationen mehr bekommt, also ein \textit{omission failure} vorliegt. Ein \textit{stuck-at} Fehler liegt
vor wenn veralterte Informationen wieder an den Roboter weitergegeben werden, so als w{\"{u}}rde ein Sensor nur noch seine letzen g{\"{u}}ltigen Daten liefern. Der dritte
Fall liegt vor, wenn komplett falsche Daten geliefert werden. Komplett falsch hei{\ss}t, dass die Anzahl und die Positionen, Gewichte etc. der Roboter potenziell fehlerhaft an
den Roboter weitergegeben werden. Die Wahrscheinlichkeiten f{\"{u}}r die jeweiligen F{\"{a}}lle lassen sich unabh{\"{a}}ngig von einander einstellen.

\subsection{Netwerk}
Da die simulierten Roboter den Khepera Robotern {\"{a}}hneln sollen, sollen auch ihre 
Kommunikationsm{\"{o}}glichkeiten denen der Khepera Roboter {\"{a}}hneln. Bei beweglichen
Objekten sind Ethernetkabel suboptimal, weswegen diese Funkverbindungen -- im Falle der Khepera
Roboter ist dies WLAN -- bevorzugen. Funkverbindungen haben aber, im Gegensatz zu gut abgeschirmten
Ethernetkabeln, Probleme mit Interferenzen durch andere Funk{\"{u}}bertragungen.

Dies f{\"{u}}hrt dazu das w{\"{a}}hrend der {\"{U}}bertragung Bits, oder sogar ganze Bytes, verf{\"{a}}lscht werden. Auch die Wahrscheinlichkeit f{\"{u}}r diesen Fall
soll einstellbar sein. Vereinfachend wird hier angenommen das jeweils nur ein Byte pro Paket verf{\"{a}}lscht wird, es gibt also keine \textit{burst errors}.
Es wird angenommen das die meiste Kommunikation zwischen Robotern "in Sichtweite" geschieht, ohne Zwischenstationen und komplexes Routing. In diesem Fall ist
es ist un{\"{u}}blich das Pakete verloren gehen oder zeitverz{\"{o}}gert weitergeleitet werden, daher ist dies nicht Teil des Fehlermodells.

\subsection{Controller}
Es wird davon ausgegangen, dass die Controllerprogramme auf unzuverl{\"{a}}ssiger Hardware, unzuverl{\"{a}}ssigen 
Betriebssystem laufen und fehlerhaft programmiert sind. Daher muss davon ausgegangen werden, dass die
Programme jederzeit abst{\"{u}}rzen k{\"{o}}nnen. Desweiteren kann auch nicht davon ausgegangen werden,
dass die Controllerprogramme das einzige Programm ist, das l{\"{a}}uft; es kann also auch zu Problemen bei
der Resourceverteilung (RAM, CPU Zeit) kommen.

\subsection{Auswirkungen}
Nachdem Fehlermodelle f{\"{u}}r die verschiedenen Elemente des Systems entwickelt wurden, kann
betrachtet werden, welche Auswirkungen die verschiedenen Fehler haben k{\"{o}}nnen. Hierbei werden nur
die Auswirkungen in Bezug auf die Platte betrachtet.

\paragraph{Unkontrollierte Bewegung} Falls sich ein Roboter unkontrolliert bewegt (also ein oder
mehrere Motoren einen \textit{stuck-at}-Fehler und die verbliebenden Motoren gar nicht mehr funktionieren)
kann dieser Roboter nicht mehr genutzt werden um die Platte auszubalancieren.
\begin{figure}
	\centering
	\includevisio[width=\textwidth]{fm_robot}
	\caption{Fehlerbaum: Unkontrollierte Bewegung eines Roboters}
	\label{fig:fault-tree-robot}
\end{figure}
\clearpage

\paragraph{Inkorrekte Bewegungen} Ein Roboter bewegt sich "falsch", wenn seine Bewegung weder die Platte 
ausbalanciert noch dazu gedacht, ist den Energiespeicher aufzuf{\"{u}}llen. Dies kann verschiedene Gr{\"{u}}nde 
haben: Entweder bewegt der Roboter sich unkontrolliert oder er wird falsch angesteuert. Beide F{\"{a}}lle
passieren falls die Controller falsche Steuerkommandos senden, oder theoretisch richtige Steuerkommandos
falsch ausgewertet werden.

Die tats{\"{a}}chliche Auswirkung dieses Zustandes h{\"{a}}ngt von der momentanen Stellung der Roboter (und ihrer
Stati) auf der Platte ab und kann nicht allgemein bestimmt werden. Generell l{\"{a}}sst sich nur sagen das dieser
Roboter nicht mehr genutzt werden kann um die Platete auszubalancieren und die verbleibenden Roboter nun auch noch
die Bewegungen dieses Roboters ausgleichen m{\"{u}}ssen.

\begin{figure}
	\centering
	\includevisio[width=\textwidth]{fm_robot2}
	\caption{Fehlerbaum: Inkorrekte Bewegung eines Roboter}
	\label{fig:fault-tree-robot2}
\end{figure}


