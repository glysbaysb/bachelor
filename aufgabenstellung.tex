\section{Einf{\"{u}}hrung}
\begin{figure}
	\centering
	\includevisio[width=\textwidth]{simulation}
	\caption{Screenshot der Simulation}
	\label{fig:simulation}
\end{figure}

\subsection{Motivation} Seit der Einf{\"{u}}hrung der Computer und vor allem seit dem rasanten Wachstums des
Internets, werden Computer in immer mehr Bereichen eingesetzt. Es gibt nur noch wenige Berufe oder 
Freizeitaktivit{\"{a}}ten die komplett ohne auskommen; Die Buchhaltung geschieht komplett elektronisch und
das Geld wird elektronisch {\"{u}}berwiesen, anstatt in einer Lohnt{\"{u}}te ausgeh{\"{a}}ndigt zu werden.
Und um herauszufinden ob der Lohn angekommen ist benutzt man Onlinebanking, wobei eine
unm{\"{o}}glich absch{\"{a}}tzbare Anzahl an vernetzten Computern beteiligt ist. Auch komplett ohne Technik
durchf{\"{u}}hrbare Freizeitaktivit{\"{a}}ten wie das Kegeln sind heute ohne Computerunterst{\"{u}}tzung
kaum noch vorstellbar --- in den Kegelbahnen z{\"{a}}hlt ein Computer f{\"{u}}r jeden Spieler mit wie viele
Punkte erreicht wurden und stellt die umgeworfenen Kegel automatisch wieder auf.

Dies bietet enorme, jeden Tag ersichtlich werdene Vorteile, birgt jedoch auch Risiken. Die Komplexit{\"{a}}t,
oder den Aufwand, die der "Nutzer" bew{\"{a}}ltigen muss, sinkt zwar (meistens), allerdings wird das
System fehleranf{\"{a}}lliger. Es gibt mehr Komponenten, also kann mehr kaputt gehen.

Zu einem Problem wird dies falls es um lebenswichtige (oder NOCH SCHLIMEMR: Geld) Bereiche wie die
Strom- und Wasserversorgung, Fahrassistenz und {\"{a}}hnliches geht. Auch kleine Fehler dort k{\"{o}}nnen
enorme Konsequen haben. 

bla bla menschen achten nicht drauf bla bla

\subsection{Aufgabenstellung} In dem Modul "Ausfallsichere Systeme" werden die verschiedenen Techniken und ihre Vor- und Nachteile
erl{\"{a}}utert und verglichen. Um das neu erworbenes Wissen auch praktisch anwenden (und vertiefen) zu k{\"{o}}nnen gibt es w{\"{a}}hrend des Semesters eine {\"{U}}bungsaufgabe.
Die Aufgabe dieser Bachelorarbeit ist die Konzeption der selbigen. Zus{\"{a}}tzlich soll das fertige Produkt auch im Rahmen von Informationsveranstaltungen als Demonstrationsprojekt
nutzbar sein und gerade auch Menschen ohne Fachkentnisse einen Einblick in die Komplexit{\"{a}}t geben.

Dies f{\"{u}}hrte zur Idee eine Computersimulation zu entwickeln. Diese k{\"{o}}nen visuell ansprechend sein und bieten vielf{\"{a}}tige Interationsm{\"{o}}glichkeiten.

\subsection{Gliederung} Als erstes wird die Simulation in groben Z{\"{u}}gen beschrieben und mit vorhandenen XXX\todo{} vergliechen. Danach werden die ben{\"{o}}tigten
Grundlagen zur Umsetzung erkl{\"{a}}rt. Dabei geht es einerseits um mechanische und physische Konzepte wie das
\textit{differential steering}\ref{diffs} und die mathematische Grundlagen der Vektorrechnung, aber auch Informationstechnologische Aspekte wie die Serialisierung von Daten oder
verschiedene Netzwerkkommunikationsarten werden betrachtet. Darauf aufbauend wird dann die Implementierung beschrieben, mit besonderem Fokus auf dem Fehlermodell und der
Fehlerinjektion.

Zum Schluss wird die Simulation anhand der Kriterien evaluiert und ein Fazit gezogen.

\clearpage
\section{Die ausfallsichere Heizung}
Eine weitere {\"{U}}bungsaufgabe wurde im Sommersemester 2016 an der FH S{\"{u}}dwestfalen entwickelt. Das Ziel ist es, mithilfe zwei Heizk{\"{o}}rpern, zweier
L{\"{u}}fter und 3 Temperatursensoren (jeweils einer an dem L{\"{u}}fter und Heizk{\"{o}}rper, einer im Ausgangsluftstrom) eine Temperatur zu regeln.
Das Gesamtsystem ist Y-f{\"{o}}rmig aufgebaut; an den Armen des Ypsilon befindet sich jeweils ein Heizk{\"{o}}rper und ein L{\"{u}}fter, an dem Punkt, an dem sich
die Arme treffen ist eine Klappe die den Luftstrom eines Arms ganz oder teilweise blockieren kann (diese Luft wird dann nach oben abgeleitet).

\paragraph{Systemaufbau} Die verschiedenen Komponenten h{\"{a}}ngen an drei verschiedenen Rechnern\footnote{jeweils mit eigenem Temperatursensor}:
\begin{itemize}
\item einem Rechner, der die Heizk{\"{o}}rper steuert
\item einem Rechner, der die L{\"{u}}fter steuert
\item einem Rechner, der die Klappe steuert
\end{itemize}

Diese Rechner k{\"{o}}nnen {\"{u}}ber UDP und I\textsuperscript{2}C miteinander kommunizieren. Diese Kommunikationswege, oder sogar ganze Rechner, k{\"{o}}nnen
allerdings zu jedem Zeitpunkt ausfallen; Auch die Sensoren und Aktoren (die Heizungen, die L{\"{u}}fter oder die Klappe) k{\"{o}}nnen ausfallen --- es ist also notwendig das System ausfallsicher zu planen.

\todo{Mehr? Weiss nicht, ich will ja nicht unsere Implementation beschreiben}

\paragraph{Evaluation} Dieses System hat allerdings physikalische Beschr{\"{a}}nkungen, die es sehr
tr{\"{a}}ge, und damit als Vorf{\"{u}}hrobjekt zum Beispiel f{\"{u}}r Schulklassen,
ungeeignet machen. Es gibt auch wenig Interaktionsm{\"{o}}glichkeiten mit dem System, es ist also schwer etwas auszuprobieren und
mit dem System zu spielen.
gibt nur eine.
warum die doof war, mal andere studenten fragen

\clearpage
\section{Anforderungen}
nummerierte liste
- mehrere instanzen gleichzeitig
- komplett anders
- fehlerinjektion parametrisierbar

\clearpage
\section{Die balancierenden Roboter}
"Ja... das ganze hatte damals ein ganz anderes Ziel, schon das macht es
hier eher unpassend. Es muss entsprechend dargestellt werden."

Um die M{\"{o}}glichkeiten des in \cite{Werner00} neu entwickelten CORE Systems zu testen wurde an der Humboldt Universit{\"{a}}t Berlin in Zusammenarbeit
mit Daimler Benz, eine Simulation {\"{a}}hnlich dieser entwickelt. Zu dieser wurde dann sp{\"{a}}ter ein echter physikalischer Roboter hinzugef{\"{u}}gt.

Im Gegensatz zu diesem System musste es, zus{\"{a}}tzlich zur nicht-funktionalen Eigenschaft Ausfallsicherheit, auch echtzeitf{\"{a}}hig sein.

\paragraph{Systemaufbau} Die Simulation und die Controller laufen auf dem NeXTStep Betriebssystem und kommunizieren {\"{u}}ber CORBA miteinander\cite{predictablecorba}, w{\"{a}}hrend
das Anzeigeprogramm Java basiert ist und z.B. auf einem Windowsrechner laufen kann.

\todo{mehr}

\paragraph{Evaluation} Trotz des sehr {\"{a}}hnlichen Konzeptes ist es nicht sinnvoll, diese bestehende Arbeit weiterzuverwenden.
Auch als reine {\"{U}}bungsaufgabe ist es nicht gut geeignet, da die Einarbeitung in COBRA vom tats{\"{a}}chlichen Lernstoff ablenken und das ganze unn{\"{o}}tig verkomplizieren w{\"{u}}rde.
bla bla gibt ga´r keine NextStep rechner mehr.
links, rechts bewegung, keine dynamik



