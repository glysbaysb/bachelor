\documentclass[
    12pt,
    bibliography=totoc,
    ngerman
]{scrartcl}

\usepackage[utf8]{inputenc}
\usepackage[T1]{fontenc}
\usepackage{lmodern}
\usepackage[ngerman]{babel}
%\usepackage[decimalsymbol=comma]{siunitx}

%\usepackage[style=ieee]{biblatex}
\usepackage{setspace}
\PassOptionsToPackage{hyphens}{url}
\usepackage[hidelinks,linktoc=all,pdfusetitle]{hyperref}
\usepackage[ampersand]{easylist}
\usepackage{graphicx}
\usepackage{longtable}
\usepackage{tabu}
\usepackage{booktabs}
\usepackage{mathtools}
\usepackage{csquotes}
\usepackage{lscape}
\usepackage{textcomp}
\usepackage{listings}
\usepackage{fancyhdr}
\usepackage{rotating}
\usepackage{subcaption}
\usepackage{amssymb} % for \checkmark
\usepackage{alltt}
\usepackage[nottoc,numbib]{tocbibind}
\usepackage{todonotes}

\setstretch{1.433} %entspricht 1,5 in Word 
\graphicspath{{images/}}
\MakeOuterQuote{"}
\pagestyle{fancy}
\lstset{frame=single,breaklines=true}
%\setcounter{biburllcpenalty}{7000}
%\setcounter{biburlucpenalty}{8000}

\newcommand{\includevisio}[2][]{\includegraphics[clip, trim=1cm 1cm 1cm 1cm, #1]{#2}}

\titlehead{{\large Fachhochschule S{\"{u}}dwestfalen} \\ Fachbereich f{\"{u}}r Elektrotechnik und Informationstechnik \\ Studiengang Technische Informatik}
\subject{\vspace{2cm} Bachelorarbeit}
\title{XXXXXXX}
\author{Geert Martin Ijewski}
\publishers{\vfill Betreuer: Prof. Dr.-Ing. habil. Jan Richling}

\begin{document}

\maketitle
\thispagestyle{empty}

\clearpage
\tableofcontents
\listoffigures
%\addcontentsline{toc}{section}{\listfigurename}
\listoftables
\listoftodos

\clearpage
\section{Einf{\"{u}}hrung}
\todo{Renderbild, Bild von der Ausgabe?}

\paragraph{Motivation} \todo{...}

\paragraph{Simulation} Die simulierte Welt besteht aus einer kreisf{\"{o}}rmigen Platte. Auf dieser k{\"{o}}nnen bis zu $N$ Roboter und $1$ F{\"{u}}llstation
platziert werden. Die Summer ihrer Gewichte kippt die Platte - sind die Gewichte zu ungleichm{\"{a}}{\ss}ig verteilt, kippt
die Platte und die Simulation endet.

Die Roboter k{\"{o}}nnen sich frei auf der Platte bewegen, verbrauchen dabei aber Energie und m{\"{u}}ssen diese dann irgendwann
an der F{\"{u}}llstation auff{\"{u}}llen. Auch wenn die Roboter sich nicht bewegen verbrauchen sie Energie (allerdings weniger
schnell).

Kontrolliert werden die Roboter von bis zu $M$ $(M >= N)$ Kontrollern. Einer oder mehr Kontroller stimmen sich ab,
so dass:
\begin{itemize}
\item die Roboter ihre Energie immer rechtzeitig auff{\"{u}}llen
\item die Roboter nicht kollidieren
\item die Platte nicht kippt
\end{itemize}

Dieser Abstimmungsvorgang sollte fehlertolerant implementiert werden. (Abh{\"{a}}ngig vom Verh{\"{a}}ltniss der Anzahl der Kontroller zur
Anzahl der Roboter k{\"{o}}nnen verschiedene Stufen der Fehlertoleran erreicht werden.\cite[s.149]{Werner00}) Teil dieser Bachelorarbeit ist eine
Beispielimplementation, die auch mithilfe eines Fehlerinjektors getestet wurde.

\todo{anforderungen?}

\paragraph{Nutzung durch die Studenten} \todo{Einleitung}
\todo{interface}

\paragraph{Evaluation}

\clearpage
\section{Grundlagen}
\todo{einleitung}

\subsection{Vektorrechnung}

\todo{mehr physikalisches? Was denn Sinus?}
\subsection{Fehlerinjektion}

\clearpage
\section{Die Welt}
\todo{Einleitung}
\subsection{Die Platte}
\subsection{Die Roboter}
\subsection{Die F{\"{u}}llstation}

\clearpage
\section{Netzwerk}
\todo{Einleitung}
wer, in welchem drin ist. \todo{aufteilung im netzwerk als Bild}

\clearpage
\section{Interface f{\"{u}}r die Studenten}
Damit die Studenten sich auf die Implementierung der Fehlertoleranz konzentrieren k{\"{o}}nnen gibt es Interfaces. \todo{umformulieren} 
Im ganzen System gibt es drei Schnittstellen:
\begin{itemize}
\item Die Schnittstelle zwischen Controller und Voter
\item Die Schnittstelle zwischen Voter und Roboter
\item Die Schnittstelle zwischen Controller und simulierter Welt \todo{oder Voter und simulierter Welt und dann an Controller senden?}
\end{itemize}

\paragraph{Die Schnittstelle zwischen Controller und Voter} ist komplett den Studenten {\"{u}}berlassen.

\paragraph{Die Schnittstelle zwischen Voter und Roboter} wird {\"{u}}ber die Welt geroutet, damit die Welt kontrollieren kann das ein
Roboter sich pro Runde nur einmal bewegt oder dreht. Die Schnittstelle selbst besteht aus der Funktionen:
\begin{lstlisting}
int rotate(RobotID r, Degree leftWheel, Degree rightWheel);
int move(RobotID r, bool move); // true = moves, false = stops
\end{lstlisting}


\clearpage
\section{Fehlerinjektion}

\clearpage
\section{Referenzimplementation}
\subsection{Voter}
\subsection{Consensus algorithm}
\paragraph{Schutzmassnahmen gegen omission failure}
\paragraph{bla bla bla}


\clearpage
\section{Evaluation}


\clearpage
%\nocite{*} % alle unbenutzen Literatureintr{\"{a}}ge trotzdem anzeigen
\begin{thebibliography}{9}

\bibitem{lamport94}
  Leslie Lamport,
  \emph{\LaTeX: a document preparation system},
  Addison Wesley, Massachusetts,
  2nd edition,
  1994.
  
\bibitem{Werner00}
  Matthias Werner,
  \emph{Responsivit{\"{a}}t: ein konsensbasierter Ansatz},
  Humboldt University of Berlin, Germany,
  978-3-89811-924-5,
  2000.
   
\end{thebibliography}

\end{document}
