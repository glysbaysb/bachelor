\documentclass[
    12pt,
    bibliography=totoc,
    ngerman,
    enabledeprecatedfontcommands
]{scrartcl}

\usepackage[utf8]{inputenc}
\usepackage[T1]{fontenc}
\usepackage{lmodern}
\usepackage[ngerman]{babel}
%\usepackage[decimalsymbol=comma]{siunitx}

\usepackage[style=ieee, backend=biber]{biblatex}
\nocite{*} % alle unbenutzen Literatureintraege trotzdem anzeigen
\addbibresource{../lit.bib}
\usepackage{setspace}
\PassOptionsToPackage{hyphens}{url}
\usepackage[hidelinks,linktoc=all,pdfusetitle]{hyperref}
\usepackage[ampersand]{easylist}
\usepackage{graphicx}
\usepackage{longtable}
\usepackage{tabu}
\usepackage{booktabs}
\usepackage{mathtools}
\usepackage{csquotes}
\usepackage{lscape}
\usepackage{textcomp}
\usepackage{listings}
\usepackage{fancyhdr}
\usepackage{rotating}
\usepackage{subcaption}
\usepackage{amssymb} % for \
\usepackage{alltt}
\usepackage[nottoc,numbib]{tocbibind}
\usepackage{todonotes}
\usepackage[toc]{glossaries}

\setstretch{1.433} %entspricht 1,5 in Word 
\graphicspath{{../images/}}
\MakeOuterQuote{"}
\pagestyle{fancy}
\lstset{frame=single,breaklines=true}
\setcounter{biburllcpenalty}{7000}
\setcounter{biburlucpenalty}{8000}

\newcommand{\xmark}{\text{\sffamily X}} % gegensymbol zu \checkmark

\newcommand{\includevisio}[2][]{\includegraphics[#1]{#2}}

\titlehead{{\large Fachhochschule S{\"{u}}dwestfalen} \\ Fachbereich f{\"{u}}r Elektrotechnik und Informationstechnik \\ Studiengang Technische Informatik}
\subject{\vspace{2cm} Bachelorarbeit}
\title{XXXXXXX}
\author{Geert Martin Ijewski}
\publishers{\vfill Betreuer: Prof. Dr.-Ing. habil. Jan Richling}

\loadglsentries{../glossary}
\makeglossaries
\glsaddall

\begin{document}

\maketitle
\todo{Bilder: vom Klassendiagram?, Ablaufdiagram für Algo?, Konsens?, Voter?}
\todo{Formatierung durchgehen. Bilder, Tabellen, Codestuecke an der richtigen Stelle. Gleichungen sehen gut aus. Alle Verweise (aufs Glossar oder Sektionen)}
\thispagestyle{empty}

\clearpage
\tableofcontents
\listoffigures
\listoftables
\clearpage
\printglossaries

\clearpage
\section{Einf{\"{u}}hrung}
\begin{figure}
	\centering
	\includevisio[width=\textwidth]{simulation}
	\caption{Screenshot der Simulation}
	\label{fig:simulation}
\end{figure}

Da fehlt noch sehr, sehr viel. Erzählen Sie eine "Geschichte": Beginnen
Sie mit einer Welt, die jeder kennt, und beschreiben diese immer weiter,
bis Sie beim Thema angekommen sind (das wäre hier die Ausbildung in
Ausfallsicheren Systemen, die Welt davor sind die ausfallsicheren
Systeme selbst). Damit haben Sie eine umfassende Motivation für das
Problem aufgebaut, das dann kurz und knackig erklärt wird: Eine weitere
Aufgabe. Alles andere kommt dann später..

\paragraph{Motivation} Im Rahmen des Modules Ausfallsichere Systeme sollen die Studenten ihr neu erworbenes Wissen auch praktisch anwenden k{\"{o}}nnen. Dies geschieht im Rahmen einer {\"{U}}bungsaufgabe. In dieser sollen die gelernten Konzepte angewandt werden, wodurch das frisch erworbene Wissen vertieft und die M{\"{o}}glichkeiten und Grenzen dessen verdeutlicht werden.

Zus{\"{a}}tzlich soll das fertige Produkt auch im Rahmen von Informationsveranstaltungen als Demonstrationsprojekt nutzbar sein.
Dies f{\"{u}}hrte zur Idee, eine Computersimulation zu entwickeln, da diese visuell ansprechend ist und vielf{\"{a}}tige
Interationsm{\"{o}}glichkeiten bieten k{\"{o}}nnen.
\todo{Aufabau dieser Arbeit?}

\clearpage
\section{Vorherige Arbeiten}
Da das Ziel dieser Arbeit die Entwicklung einer Aufgabe zur Verdeutlichung der Prinzipen Ausfallsicherer Systeme ist, wird ein kurzer {\"{U}}berblick {\"{u}}ber bereits bestehende Systeme dieser Art gegeben.

\subsection{Die balancierenden Roboter}
Um die M{\"{o}}glichkeiten des in \cite{Werner00} neu entwickelten CORE Systems zu testen wurde an der Humboldt Universit{\"{a}}t Berlin in Zusammenarbeit
mit Daimler Benz, eine Simulation {\"{a}}hnlich dieser entwickelt. Zu dieser wurde dann sp{\"{a}}ter ein echter physikalischer Roboter hinzugef{\"{u}}gt.

Im Gegensatz zu diesem System musste es, zus{\"{a}}tzlich zur nicht-funktionalen Eigenschaft Ausfallsicherheit, auch echtzeitf{\"{a}}hig sein.

\paragraph{Systemaufbau} Die Simulation und die Controller laufen auf dem NeXTStep Betriebssystem und kommunizieren {\"{u}}ber CORBA miteinander\cite{predictablecorba}, w{\"{a}}hrend
das Anzeigeprogramm Java basiert ist und z.B. auf einem Windowsrechner laufen kann.

\todo{mehr}

\paragraph{Evaluation} Trotz des sehr {\"{a}}hnlichen Konzeptes ist es nicht sinnvoll, diese bestehende Arbeit weiterzuverwenden.
Die damals noch akzeptable graphische Darstellung wirkt heute antiquiert, was gerade bei einem Vorf{\"{u}}hrobjekt hinderlich
ist. Auch als reine {\"{U}}bungsaufgabe ist es nicht gut geeignet, da die Einarbeitung in COBRA und die Implementierung der
Echtzeitanforderungen vom tats{\"{a}}chlichen Lernstoff ablenken und das ganze unn{\"{o}}tig verkomplizieren w{\"{u}}rde.

\subsection{Die ausfallsichere Heizung}
Eine weitere {\"{U}}bungsaufgabe wurde im Sommersemester 2016 an der FH S{\"{u}}dwestfalen entwickelt. Das Ziel ist es, mithilfe zwei Heizk{\"{o}}rpern, zweier
L{\"{u}}fter und 3 Temperatursensoren (jeweils einer an dem L{\"{u}}fter und Heizk{\"{o}}rper, einer im Ausgangsluftstrom) eine Temperatur zu regeln.
Das Gesamtsystem ist Y-f{\"{o}}rmig aufgebaut; an den Armen des Ypsilon befindet sich jeweils ein Heizk{\"{o}}rper und ein L{\"{u}}fter, an dem Punkt, an dem sich
die Arme treffen ist eine Klappe die den Luftstrom eines Arms ganz oder teilweise blockieren kann (diese Luft wird dann nach oben abgeleitet).

\paragraph{Systemaufbau} Die verschiedenen Komponenten h{\"{a}}ngen an drei verschiedenen Rechnern\footnote{jeweils mit eigenem Temperatursensor}:
\begin{itemize}
\item einem Rechner, der die Heizk{\"{o}}rper steuert
\item einem Rechner, der die L{\"{u}}fter steuert
\item einem Rechner, der die Klappe steuert
\end{itemize}

Diese Rechner k{\"{o}}nnen {\"{u}}ber UDP und I\textsuperscript{2}C miteinander kommunizieren. Diese Kommunikationswege, oder sogar ganze Rechner, k{\"{o}}nnen
allerdings zu jedem Zeitpunkt ausfallen; Auch die Sensoren und Aktoren (die Heizungen, die L{\"{u}}fter oder die Klappe) k{\"{o}}nnen ausfallen --- es ist also notwendig das System ausfallsicher zu planen.

\todo{Mehr? Weiss nicht, ich will ja nicht unsere Implementation beschreiben}

\paragraph{Evaluation} Dieses System hat allerdings physikalische Beschr{\"{a}}nkungen, die es sehr
tr{\"{a}}ge, und damit als Vorf{\"{u}}hrobjekt zum Beispiel f{\"{u}}r Schulklassen,
ungeeignet machen. Es gibt auch wenig Interaktionsm{\"{o}}glichkeiten mit dem System, es ist also schwer etwas auszuprobieren und
mit dem System zu spielen.

\clearpage
\section{Konzept}
bla bla\todo{}

Die simulierte Welt besteht aus einer kreisf{\"{o}}rmigen Platte. Auf dieser k{\"{o}}nnen bis zu $N$ Roboter und $1$ F{\"{u}}llstation
platziert werden. Die Summe alle Gewichte kippt die Platte; sind die Gewichte zu ungleichm{\"{a}}{\ss}ig verteilt, kippt
die Platte und die Simulation endet.

Die Roboter k{\"{o}}nnen sich frei auf der Platte bewegen, verbrauchen dabei aber Energie und m{\"{u}}ssen diese dann irgendwann
an der F{\"{u}}llstation auff{\"{u}}llen. Auch wenn die Roboter sich nicht bewegen, verbrauchen sie Energie (allerdings weniger
schnell).

Gesteuert werden die Roboter von bis zu $M$ $(M \geq N)$ Controllern. Einer oder mehr Controller stimmen sich ab,
so dass:
\begin{itemize}
\item die Roboter ihre Energie immer rechtzeitig auff{\"{u}}llen
\item die Roboter nicht kollidieren
\item die Platte nicht kippt
\end{itemize}

Dieser Abstimmungsvorgang sollte fehlertolerant implementiert werden, so dass auch beim Ausfall eines (oder mehrerer) Controller die Roboter sich immer noch koordiniert bewegen. (Abh{\"{a}}ngig vom Verh{\"{a}}ltniss der Anzahl der Controller zur
Anzahl der Roboter k{\"{o}}nnen verschiedene Stufen der Fehlertoleranz erreicht werden.\cite[s.149]{Werner00})
Teil dieser Bachelorarbeit ist eine
Beispielimplementation, die die oben ausgef{\"{u}}hrten Anforderungen erf{\"{u}}llt. Dies wurde auch mithilfe eines Fehlerinjektors verifiziert.

\paragraph{Nutzung durch die Studenten} Von den Studierenden muss nicht die gesamte Simulation erstellt werden; die {\"{U}}bungsaufgabe bezieht sich nur auf die Abstimmung zwischen den Controllern und die Ansteuerung der Roboter.
Daher ist es notwendig, ein Interface ins "innere" der Simulation bereitzustellen, das dann von den Studenten genutzt wird.

\begin{itemize}
\item Simulation berechnet bewegungen und so. per definition ausfallsicher gegen ALLES! (... als ob). grafische ausgabe
\item Voter sammelt steuerkommandos. per definition ausfallsicher ggü crash failure
\item Controller generiert steuerkommandos, kann jederzeit ausfallen
\end{itemize}

Die Voter und Controller laufen auf Virtuellen Maschinen , jeweils eine VM pro Voter- oder Controllerinstanz. Diese VMs befinden sich in einem Netzwerk das durch eine Fehlerinjektion UDP Pakete verf{\"{a}}lscht --- dies simuliert ein fehlertr{\"{a}}chtiges Netzwerk wie es zum Beispiel bei Funknetzwerken der Fall ist. Es wird vorgegeben dass die gesamte Kommunikation {\"{u}}ber UDP stattfinden muss.
Die Kommunikation von Simulation zu den Votern ist davon nicht betroffen.

\clearpage
\section{3D Engine}
\subsection{Anforderungen}
Die Simulation, vor allem die Bewegung der Roboter, kann besonders gut visuell dargestellt werden. Auch wenn sich das Geschehen rein auf einer zweidimensionalen Ebene stattfindet, ist eine dreidimensionale
Darstellung ansprechender und wird deshalb vorgezogen.

Diese Darstellung geschieht mithilfe einer \textit{3D Engine}; sie bietet die M{\"{o}}glichkeit Objekte und oft auch Lichtquellen und Schattenspiel darzustellen.
Anstatt solch eine Engine selbst zu implementieren, wird eine vorhandene genommen.

Um die sehr gro{\ss}e Auswahl einzuschr{\"{a}}nken wurden die nachfolgenden Kriterien benutzt.

\paragraph{Kosten} Die 3D Engine muss f{\"{u}}r den nicht-kommerziellen edukativen Gebrauch kostenlos sein; dies schliesst, zum Beispiel, die CryEngine\cite{cryengine} automatisch aus.

\paragraph{Plattformunabh{\"{a}}ngigkeit} Die mit der \textit{game engine} entwickelte Simulation muss auf ohne gro{\ss}en Aufwand auf mehrere System (hier Windows und Linux) portierbar sein.

\paragraph{Funktionalit{\"{a}}t} Eine Untermenge der 3D Engines sind \textit{game engines}, diese beeinhalten, zus{\"{a}}tzlich zur reinen 3D Engine, auch eine Physikengine und eine Audioengine.
Durch die Art der Simulation ist ersichtlich das eine Physikengine viel Arbeit ersparen k{\"{o}}nnte, gerade im Hinblick auf Beschleunigung, Bewegung oder Kollisionserkennung. Daher werden \textit{game engines} reinen 3D Engines (wie Blender) vorgezogen.

\paragraph{\todo{namen finden}} Das Hauptkriterium ist die einfache Benutzung. Dies beeinhaltet zum einen die
tats{\"{a}}chliche Nutzung (also: gibt es einen graphischen Leveleditor? Wie kann die Interaktionen zwischen
Objekten gesteuert werden?), zum anderen die Qualit{\"{a}}t der Dokumentation und schließlich wie verbreitet diese
\textit{engine} ist, und damit auch wie einfach es ist bei Problemen Hilfestellungen zu bekommen.

Gerade in diesem Punkt stach Unity hervor und wurde deshalb als Basis f{\"{u}}r diese Arbeit ausgew{\"{a}}hlt.

\subsection{Unity}
Unity ist eine von Unity Technologies entwickelte \textit{game engine}, die 2005 f{\"{u}}r Macintosh entwickelt wurde und in der Zwischenzeit auf 27 Plattformen portiert wurde.

In dieser Arbeit wird es benutzt, um die Welt und die Interaktionen zwischen den Objekten zu simulieren und diese auch anzuzeigen. Mithilfe einer Scriptingschnittstelle kann das Geschehen modifiziert werden, z.B. kann die Welt
abh{\"{a}}ngig von der Position und den Gewichten der Objekt gekippt werden oder die Kamera durch Tastatureingaben bewegt werden.

\subsection{Unitys Grafikengine}
Die in Unity enthaltene Grafikengine ist eine vollwertige 3D-Grafikengine, die es erm{\"{o}}glicht professionelle Videospiele zu programmieren. In dieser Arbeit wird nur
ein kleiner Teil der Funktionalit{\"{a}}t benutzt; komplexere Features wie die Anzeige von Rauch oder realistischem Wasser sind nicht in der Simulation enthalten.

Notwendig ist es aber, dass die verschiedenen Objekte in der simulierten Welt in 3D zu sehen sind und sich die Kamera bewegen l{\"{a}}sst, um das simulierte Geschehen zu
verfolgen. Dies geh{\"{o}}rt zur Grundfunktionalit{\"{a}}t und muss nicht extra implementiert werden.

\subsection{Unitys Physikengine}
Ein integraler Teil moderner Videospiele ist eine realistische physikalische Umgebung. Dies wird durch die Physikengine von Unity erm{\"{o}}glicht, die in dieser Arbeit wichtigen Funktionalit{\"{a}}ten sind:

\begin{itemize}
\item die Kollisionserkennung, einerseits um zu bestrafen wenn Roboter in einander fahren und auch zur Erkennung des Kontaktes mit der F{\"{u}}llstation
\item die Bewegung der Objekte in der Welt zu simulieren, abh{\"{a}}ngig von ihrem Gewicht und ihrer Geschwindigkeit
\end{itemize}

\subsection{Scripting bei Unity}
Alle Elemente der Unity Engine k{\"{o}}nnen vom Programmierer gesteuert werden, unterst{\"{u}}tzt werden dabei sie Skriptsprachen C\#, JavaScript und Boo.\cite{wiki:unity} Um dies zu erm{\"{o}}glichen ist es aus dem Skript heraus m{\"{o}}glich alle Eigenschaften
der Objekte \footnote{Dies beinhaltet nicht nur die Roboter und F{\"{u}}llstationen sondern auch die Kamera oder statische Objekte wie Mauern.} auszulesen und zu manipulieren.

Beispielsweise wurde der F{\"{u}}llstation ein Skript zugeordnet das solange ein Objekt mit dieser kollidiert testet ob dies Objekt ein Roboter ist, und falls dies der Fall ist,
den Roboter auftankt. Ein anderes Skript bewegt die Kamera abh{\"{a}}ngig von den Tastatureingaben des Benutzers.

\clearpage
\section{Publish-Suscribe}
Da die Simulation mit einer nicht vorab bestimmbaren Anzahl von Votern/Robotern (die nicht unbedingt im gleichen \textit{subnet} sein m{\"{u}}ssen) arbeitet, ist das Publish-Suscribe Kommunikationspattern\cite{pubsub}
der angemessen Weg, um Voter und Simulation zu verbinden.
Desweiteren erlaubt dieses Pattern auch eine zeitliche Trennung und asynchrone Kommunikation, dies entspricht dem Verhalten der Controller --- Kommunikationspartner die zu jedem Zeitpunkt ausfallen oder
sich verbinden k{\"{o}}nnen. Eine n{\"{a}}here Erl{\"{a}}uterung, speziell der Implementation in nanomsg, befindet sich im jeweiligen Kapitel.

Oft gibt es ein Kommunikationsverhalten in dem eine Person (im folgenden \textit{Publisher} genannt), ob periodisch oder zu nicht vorhersagbaren Zeiten, etwas sagt,
dass f{\"{u}}r mehrere andere (im folgenden \textit{Suscriber} genannt) wichtig ist. Falls bestimmte Informationen nur f{\"{u}}r bestimmte Suscriber wichtig sind, kann die Filterung entweder auf Seiten des Publisher oder des Suscribers vorgenommen werden.
Eine M{\"{o}}glichkeit diese Kommunikation zu implementieren ist es die M{\"{o}}glichkeiten des UDP broadcasts zu nutzen --- allerdings funktioniert dies nur wenn sich alle Netzwerkteilnehmer im gleichen
\textit{subnet} befinden.

Eine Alternative ist es sich bei dem Publisher, entweder nur f{\"{u}}r eine bestimmte Art von Informationen (genannt \textit{topic}) oder f{\"{u}}r alle, anzumelden. Solange eine Verbindung besteht,
wird der Publisher allen Suscribern die Informationen mitteilen; durch die Benutzung von nanomsg wird eine zeitliche Trennung und asynchrone Kommunikation zwischen die Netzwerkteilnehmer aufgebaut,
dies f{\"{u}}hrt dazu das, falls ein Suscriber ausf{\"{a}}llt (\textit{crash error}) oder neu hinzukommt, die Kommunikation mit den anderen Teilnehmern nicht beeinflusst wird (auch nicht im Zeitverhalten).
\begin{figure}
	\centering
	\includevisio{pubsub}
	\caption{Publish-Suscribe}
	\label{fig:pubsub}
\end{figure}
\todo{sieht kacke aus. kleiner und andere Auflösung?}


\clearpage
\section{Netzwerklibrary}
Um die Kommunikation zwischen den einzelnen Netzwerkteilnehmern nicht komplett selbst zu implementieren, k{\"{o}}nnen bereits bestehende Netzwerklibraries genutzt werden.
Diese bieten, zum Beispiel, die M{\"{o}}glichkeit an einem entfernten Netzwerkteilnehmer Funktionen aufzurufen oder sich f{\"{u}}r Multicasts anzumelden, auch wenn man in einem anderen
\textit{subnet} ist.

\subsection{Anforderungen}
Um verteilte Systeme zu programmieren, bietet sich das RPC Modell an, mit dem auf einem anderen Rechner Funktionen aufrufen kann.

Es gibt allerdings keine Netzwerklibrary die dieses Pattern und RPC anbietet; da es einfacher ist auf Publish-Suscribe ein RPC Modell aufzubauen, als dieses Pattern, zum Beispiel, grpc beizubringen, wird
diese Funktionalit{\"{a}}t vorgezogen.

Netzwerkbibliotheken, die in C++ geschrieben wurden, k{\"{o}}nnen zwar (mit einem Interface) auch unter C benutzt werden, f{\"{u}}gen dem Programm
dann aber "heimlich" die C++ Runtime hinzu. Eine rein in C geschriebene Bibliothek ist daher vorzuziehen.

\clearpage % make sure the table is, at least, in the right section
\begin{table}[h]
\centering
\begin{tabu}{c | c | c | c | p{5cm}}
	\toprule
	Name & RPC & Publish-Suscribe & C++ Runtimeabh{\"{a}}ngigkeit & andere Sprachen \\
	\midrule
	SunRPC & \checkmark & \xmark & \checkmark & C\#, Java, ... \\
	grpc\cite{grpc} & \checkmark & \xmark & \xmark & C\#, Python, Java, ... \\
	ZeroMQ\cite{zeromq} & \xmark & \checkmark & \xmark & C\#, Python, Java, ... \enquote{40+ languages} \\
	nanomsg\cite{nanomsg} & \xmark & \checkmark & \checkmark & C\#, Python, Java, ... (25 Sprachen) \\
	\bottomrule
\end{tabu}
\caption{Anforderungen an die Netwerkbibliothek}
\end{table}

Aus dieser vergleichenden Darstellung wird klar, dass nanomsg am besten geeignet ist; auf dessen speziellen Eigenschaften wird nun n{\"{a}}her eingegangen.

\subsection{nanomsg}
nanomsg ist eine komplett in C geschrieben Netzwerkbibliothek, mit einer API die der POSIX Socket API gleicht (es wird sich also, zum Beispiel mit nn\_connect() verbunden oder mit nn\_send() gesendet).
Der gro{\ss}e Vorteil von nanomsg sind die implementierten {\"{U}}bertragungsarten. Je nach gew{\"{u}}nschter {\"{U}}bertragungsart wird ein unterschiedlicher Socket erstellt, dieser erlaubt die
Interaktion mit den anderen Netzwerkteilnehmern die auch diese {\"{U}}bertragunsart benutzen.

Beispielhaft wird nun auf einige der genutzten Arten eingegangen.

\paragraph{{\"{U}}bertragungsarten}
\subparagraph{Request-Reply} Die intuitiv verst{\"{a}}ndlichste Art zu kommunizieren ist es eine Frage zu stellen, die dann vom Gegen{\"{u}}ber beantwortet wird. In nanomsg ist dies die request-reply
{\"{U}}bertragungsart.

\clearpage
\section{Serialisierung}
Durch die Benutzung der Unity 3D Engine ist die Benutzung von C\# vorgeschrieben, der Rest der Arbeit soll in C geschreiben werden. Die Verwendung verschiedener Sprachen erschwert die Benutzung eines
Bin{\"{a}}rprotokoll --- aus diesem Grund werden die Daten serialisiert.

\subsection{Anforderungen}
Das wichtigste Auswahlkriterium der Serialsierungslibrary ist das es eine C (f{\"{u}}r das Studenteninterface) und eine C\# Version (f{\"{u}}r Unity) gibt.
Desweiteren sollen die serialisierten Nachrichten m{\"{o}}glichst klein sein, um auch nach der Anwendung
eines fehlerkorrigierenden Codes, so wenig Redudanz wie m{\"{o}}glich aufzuweisen und damit nur wenig f{\"{u}}r Paketverf{\"{a}}lschungen anf{\"{a}}llig zu sein.

\begin{table}[h]
\centering
\begin{tabu}{c | c | c | c | p{5cm}}
	\toprule
	Name & Wenig Redudanz? & C Version & C\# Version \\
	\midrule
	JSON & \xmark & \checkmark & \checkmark \\
	XML & \xmark & \checkmark & \checkmark \\
	ProtoBuf & \checkmark & \xmark\footnotemark & \checkmark \\
	Msgpack\cite{msgpack} & \checkmark & \checkmark & \checkmark \\
	BSON & \checkmark & \checkmark & \checkmark \\
	\bottomrule
\end{tabu}
\caption{Anforderungen an die Serialisierungsbibliothek}
\end{table}
\footnotetext{Es existieren unoffizielle, nicht mehr weiterentwickelte Implementation} 

Bekannte Serialisierungsformate wie JSON und XML enthalten zu viel Redudanzen und k{\"{o}}nnen ausgeschio{\ss}en werden. Zur Auswahl bleiben damit beispielsweise BSON und MsgPack.
Bei BSON werden die Daten an sich effizient serialisiert, allerdings wird immer der Feldname mitgesendet. Dies erleichtert zwar die Versionierbarkeit, vergr{\"{o}}{\ss}ert
die Nachricht allerdings. Aus diesem Grund wird MsgPack als Serialisierungsformat genutzt.

\subsection{MessagePack}
MessagePack ist ein in {\"{u}}ber 50 Sprachen verf{\"{u}}gbares Serialisierungsframework, dessen Nachrichten besonders wenig Redudanzen enthalten sollen. Beispielhaft wird hier auf die Serialisierung von
Integern und Zeichenketten mit fester L{\"{a}}nge eingegangen.

Die Grundpidee von MessagePack ist es Datentypen nicht mit fester Breite sondern nur mit der minmal n{\"{o}}tigen Breite (oder wenigstens einer Ann{\"{a}}hrung an diese) abzuspeichern.
\paragraph{Integer} Da Datens{\"{a}}tze oftmals aus vielen kleinen Integerwerte bestehen, macht es Sinn gerade die Serialsierung dieser zu optimieren. Im Gegensatz zur "normalen" Verarbeitung als immer gleich breite (zum Beispiel 64bit) Zahlen, werden vorzeichenlose Integerwerte kleiner als 128 in einem Byte abgespeichert; der Wert des Bytes entspricht dabei dem Wert des Integers. Dahingegen signalisiert ein gesetztes MSB das es kein (kleiner) Integerwert sondern ein anderer Datentyp ist.
\paragraph{Fixed Size String} Um kleinere Zeichenketten zu encodieren gibt es erst ein Signalzeichen \todo{besseres wort} in dem die obersten 3 bits gesetzt sind. Die restlichen 5bits geben die L{\"{a}}nge der Zeichenkette an --- also maximal $2^5 - 1 = 31$ Zeichen. Dies f{\"{u}}hrt zu einem durchschnittlichen Overhead von $ \frac{1}{\frac{2^5-1}{2}} = 6\% $.

L{\"{a}}ngere Zeichenketten werden mit anderen Signalzeichen kodiert; beispielsweise 0xDB f{\"{u}}r eine Zeichenketten mit der Maximall{\"{a}}nge von $(2^{32})-1$ Zeichen.
In diesem Fall folgt die L{\"{a}}ng als 32bit big endian
\footnote{L{\"{a}}ngen werden immer in big-endian kodiert um den Datenaustausch auch zwischen Rechnern unterschiedlicher Architekturen gew{\"{a}}rleisten zu k{\"{o}}nen.}
Wert mit fester Breite. Also ein verschwindend geringer durchschnittlicher Overhead von $ \frac{1}{\frac{2^{32}-1}{2}} = 0.00000004\% $.

\clearpage
\section{RPC?}

\clearpage
\section{Quaternions? Vektorrechnung?}

\clearpage
\section{Bewegungen}
Die simulierten Roboter sind an den Kepheraroboter angelehnt, einen gerade im unitversit{\"{a}}ren Bereich weit verbreiteten Roboter mit 2 Motoren und 2 starren R{\"{a}}dern.
Dies gibt auch das Bewegungsmodell vor: \textit{differential steering}, auch \textit{differential drive} genannt. Ausf{\"{u}}hrlicher wird dies im zugeh{\"{o}}rigen Unterkapitel erkl{\"{a}}rt.

\subsection{Khepera}\label{kephera}
\todo{wie gebe ich Quellen fuer einen ganzen Absatz / Sektion an?}
Die Kheperaroboterserie existiert seit Mitte der 90ger Jahre, das neuste Modell ist der Khepera IV. Dieser Roboter hat eine zylindrische Grundform mit einem Durchmesser von 140mm und einer
H{\"{o}}he von 58mm. Der Roboter wiegt 540g und kann zus{\"{a}}tzlich bis zu 2kg Nutzlast tragen.

Auf der Unterseite sind 2 starre R{\"{a}}der, angetrieben von jeweils einem elektrischen Servomotor. Damit erreicht der Roboter Geschwindigkeiten von bis zu einem Meter pro Sekunde.
Um die Position des Roboters odeometrisch bestimmen zu k{\"{o}}nnen geben die Motoren auch aus, wie oft sich das jeweilige Rad schon gedreht hat. Zus{\"{a}}tzlich dazu hat der Roboter
auch Infrarot-, Ultraschall- und Lichtsensoren, mit denen die Umgebung wahrgenommen werden kann. Dies kann z.B. genutzt werden um Kollisionen zu vermeiden.
Mithilfe der integrieten Beschleunigungs- und Gyrosensoren kann der Roboter auch seine Geschwindigkeit und seine Neigungswinkel feststellen.

Gesteuert wird der Khepera IV von einer ARM CPU auf der ein vollwertiges Linux l{\"{a}}uft. Des weiteren gibt es auch ein WLAN Modul, eine Kamera und Lautsprecher plus Mikrophon.

\begin{figure}
	\centering
	\includevisio{k4-detail1.png}
	\caption{Der Khepera IV Roboter}
	\label{fig:kheperaiv}
\end{figure}

\subsection{Differential Steering}
Beim \textit{differential steering} k{\"{o}}nnen die R{\"{a}}der nicht, wie z.B. beim Auto, geschwenkt werden --- sie sind fest. Es kann zwischen drei Bewegungsarten unterschieden werden:
\begin{itemize}
\item Falls die R{\"{a}}der gleich schnell drehen, f{\"{a}}hrt der Roboter gerade nach vorne oder hinten.
\item Falls die R{\"{a}}der unterschiedlich schnell, aber in die gleiche Richtung, rotieren wird eine Kurve gefahren. Dabei dreht sich der Roboter in Richtung des langsamer rotierenden Rades.
\item Falls die R{\"{a}}der in unterschiedliche Richtungen rotieren, wendet der Roboter auf der Stelle. Dabei dreht sich der Roboter auch wieder in Richtung des langsamer rotierenden Rades.
\end{itemize}

Diese Bewegungen k{\"{o}}nnen auch mathematisch ausgedr{\"{u}}ckt werden. Dabei gibt $V_r(t)$ die Rotationsgeschwindigkeit des rechten Motors zum Zeitpunkt t angibt, w{\"{a}}hrend $V_l(t)$ den
linken Motor beschreibt. $R$ ist der Radius eines Rades, und $L$ die Distanz zwischen den R{\"{a}}dern.
\begin{subequations}\label{eq:diffs}
\begin{align}
	\Theta(t) &= \Theta(0) + \frac{R}{L} \int_0^t \mathrm{v_R(t) - v_L(t)}\mathrm{d}t \label{eq:diffs-theta}\\
\intertext{Mithilfe der Orientierung zum Zeitpunkt $t$ ist es m{\"{o}}glich die "Fahrtrichtung" zum
	Zeitpunkt $t$ zu bestimmen. Eine Multiplikation dieser mit den Geschwindigkeitsvektoren ergibt den
	Bewegungsvektor zum Zeitpunkt $t$ - daraus l{\"{a}}sst sich die Position ermitteln:
}
	x(t) &= x(0) + \frac{1}{2} \int_0^t \mathrm{\big(v_R(t) + v_L(t)\big) * \cos\big(\Theta(t)\big)}\mathrm{d}t \label{eq:diffs-x}\\
	y(t) &= y(0) + \frac{1}{2} \int_0^t \mathrm{\big(v_R(t) - v_L(t)\big) * \cos\big(\Theta(t)\big)}\mathrm{d}t \label{eq:diffs-y}
\end{align}
\end{subequations}

\clearpage
\section{Die Gleichgewichte?}

\clearpage
\section{Die Welt}
Die simulierte Welt besteht aus den Robotern, die von den Studierenden gesteuert werden sollen, einer F{\"{u}}llstation, an der die Robter Energie tanken k{\"{o}}nnen, und
der Welt, einer kreisf{\"{o}}rmigen, kippbaren Platten, auf der diese Objekte platziert werden und sich bewegen k{\"{o}}nnen.

Simuliert wird die Welt mit der Unity \textit{game engine}. Diese erm{\"{o}}glicht es plattformunabh{\"{a}}nige Spiele oder, in diesem Fall, Simulationen zu schreiben.
Dabei stellt sie, unter anderem, eine Physikengine, eine Grafikengine und eine Schnitstelle zum scripten dieser bereit. 

\subsection{Die Roboter}\label{robot}
In der simulierten Welt k{\"{o}}nnen sich bis zu $N$ Roboter bewegen. Diese bewegen sich aber nicht selbstst{\"{a}}ndig, sondern werden von den Controllern ferngesteurt.
Wie sie in der Simulation dargestellt werden, wird durch das grafische Modell bestimmt. Anhand dessen bestimmen sich auch die Dimensionen, diese werden f{\"{u}}r die Kollisionerkennung
gebraucht. Die Dimensionen, zusammen mit der Masse, ergeben das physische Modell, dieses hat Auswirkungen auf die Simulation.

\paragraph{Grafisches Modell} Mithilfe von Blender, einem 3D Designprogramm, wurde ein Robotermodell designt, das dem Kepheraroboter entspricht. Die Grundform des Roboters ist
ein Kegel, an dessen DINGSBUMS\todo{} befinden sich Ausbuchtungen, diese geben eine visuelle Indikation {\"{u}}ber die Position der R{\"{a}}der. Der F{\"{u}}llstand wird {\"{u}}ber eine Lampe angezeigt, die ihre
Farbe von Gr{\"{u}}n (voll), {\"{u}}ber Gelb bis Rot (leer) {\"{a}}ndert.
\todo{Bild}


\paragraph{Physikalisches Modell}
\todo{physisch?}
Ein Roboter $i$ wird dabei beschrieben durch seine Position und Gewicht
$ N_i = \bigl(\begin{smallmatrix} x(i) \\ y(i) \\ w(i) \end{smallmatrix}\bigr)$, eine
Geschwindigkeit $ V_i = \Delta v $ und den momentanen Drehwinkel
$ R_i = r_y(i)$. \todo{Ausdehnung}

Das Gewicht des Roboters ist abh{\"{a}}ngig vom Grundgewicht des Roboters und seinem momentanen F{\"{u}}llstatus: 
\begin{equation}
    \label{eq:w}
 w(N_i) = 1 + e(N_i) * 0.03
\end{equation}

Die Roboter haben einen Energiespeicher, der mit maximal 1000 Energieeinheiten
aufgeladen werden kann, und verbrauchen diese Energie, ob beim Fahren oder
Stillstand. Dabei verbrauchen sie pro Runde ohne Bewegung immer eine Energieeinheit und zus{\"{a}}tzlich, abh{\"{a}}ngig von der Geschwindigkeit, Energie f{\"{u}}r die Bewegung:
\begin{equation}
    \label{eq:entladen}
	e(N_i, t + 1) = e(N_i, t) - 1 - |V_i|
\end{equation}
Falls nicht gen{\"{u}}gend Energie f{\"{u}}r Bewegung und Rotation vorhanden ist, bewegt sich der Roboter nicht.

Die Bewegung des Roboters wird vorgegeben durch $V_l(t)$ und $ V_r(t)$. Diese werden vom Controller, {\"{u}}ber den Voter an die Simulation weitergegeben und dann von Unity verarbeitet.
Die neue Position und der neue Rotationswinkel des Roboters werden errechnet und, falls gen{\"{u}}gen Energie
vorhanden, auch eingenommen - dazu wird der \textit{rigidbody} des Roboters manipuliert.
Um die Bewegung fl{\"{u}}{\ss}iger darzustellen wird zwischen den momentanen und gew{\"{u}}nschten Positionen / Winkel interpoliert.

\subsection{Die F{\"{u}}llstation}\label{fuelstation}
Innerhalb der Welt muss eine F{\"{u}}llstation platziert werden, um den Roboter die M{\"{o}}glichkeit zu geben sich aufzuladen. Auch diese wird durch ihren Vektor $ F = \bigl(\begin{smallmatrix} x \\ y \\ w \end{smallmatrix}\bigr)$ beschrieben. Eine F{\"{u}}llstation hat dabei ein festes Gewicht: $ w(F) = 5 $.

Diese wird vor Simulationsbeginn platziert und bewegt sich im weiteren Verlauf nicht.
Falls sich ein Roboter an die F{\"{u}}llstation heranbewegt, also gilt: 
\begin{equation}
    \label{eq:dist}
 |\bigl(\begin{smallmatrix} x(i) \\ y(i) \end{smallmatrix}\bigr) - \bigl(\begin{smallmatrix} x(F) \\ y(F) \end{smallmatrix}\bigr)| \leq |\bigl(\begin{smallmatrix} 1 \\ 1 \end{smallmatrix}\bigr)|
\end{equation}
wird dieser Roboter aufgeladen. Die Ladefunktion ~\ref{eq:laden} ist hier eine einfache Gerade:
\begin{equation}
    \label{eq:laden}
	e(N_i, t + 1) = max((e(N_i, t) + 10, 1000) 
\end{equation}

\subsection{Die Platte}\label{plate}
Die simulierte Welt besteht aus einer 100 Einheiten gro{\ss}en kreisf{\"{o}}rmigen Platte, die, basierend auf den Gewichten welche sich auf ihr befinden, kippt.
Die Platte soll sich im Gleichgewicht befinden, also sich nicht bewegen und nicht gekippt sein. Dies ist nur der Fall wenn alle Kr{\"{a}}fte/Gewichte (der Objekte, also der F{\"{u}}llstation und aller Roboter) die auf die Platte wirken, sich zu Null aufaddieren. Dabei werden die Objekte als Vektoren bestehend aus Gewicht und Abstand zum Mittelpunkt betrachtet.

Um das Kippen der Platte auf der Links-Rechts und vorne-hinten Achse getrennt angeben zu k{\"{o}}nnen werden diese getrennt voneinander ausgerechnet. Dabei wird das Hebelgesetz im eindimensionalen Fall (wie auf einer Wippe) angewandt, also Distanz mal Masse ergibt die Kraft in dieser Richtung\footnote{Da die Positionsbestandteile (x und y) einzeln betrachtet werden, muss der Winkel nicht in die Rechnung einflie{\ss}en}.

F{\"{u}}r die in Links-Rechts wirkende Kr{\"{a}}fte gilt Gleichung ~\ref{eq:forces:x}, f{\"{u}}r die Kr{\"{a}}fte die beschreiben wie die Platte nach vorne-hinten gekippt gilt Gleichung ~\ref{eq:forces:y}.
\begin{subequations}\label{eq:forces}
\begin{align}
	V(x) = x(F) * w(F) + \sum_{i=0}^{|N|} ( x(N_i) * w(N_i) ) \label{eq:forces:x}\\
	V(y) = y(F) * w(F) + \sum_{i=0}^{|N|} ( y(N_i) * w(N_i) ) \label{eq:forces:y}\\
\end{align}
\end{subequations}

Das Ziel ist es die Platte auszubalancieren, also soll gelten:
\begin{equation}\label{eq:gleichgewicht}
	V(x) = V(y) = 0
\end{equation}

\todo{bild, wie im code die dolle ascii art}
Um aus den Kr{\"{a}}ften den Kippwinkel zu errechnen muss, wie im Bild dargestellt, der Arcussinus genutzt werden. Dabei ist die "Hyptothenuse" das Plattengewicht ($ N_Platte = 1000$). \todo{das wirkt total komisch. Ich teile Kraft/Gewicht? Und sowieso kann man asinus fuer solche Sachen ueberhaupt benutzen?}
$$ asin(\frac{V(x)}{ N_Platte}) = \beta = W_{x}, asin(\frac{V(y)}{N_Platte}) = \alpha = W_{y} $$

\clearpage
\section{Systemaufbau}
\todo{alles neu}
Das ganze System besteht aus verschiedenen Komponenten, die teilweise sogar in unterschiedlichen Netzwerken h{\"{a}}ngen. 

\subsection{Netzwerk}\label{network}
Die Fehlertoleranz dieses Systems wird durch eine r{\"{a}}umliche Redudanz erreicht. Die verschiedenen Teile des Systems sind {\"{u}}ber ein Ethernetnetzwerk miteinander verbunden und benutzen dieses
f{\"{u}}r die Konsensbildung und um Informationen {\"{u}}ber den momentanen Status der Simulation zu bekommen.

Allerdings ist ein Teil der Systemkomponenten in einem Netzwerk das durch eine Fehlerinjektion Pakete verf{\"{a}}lscht, verschwinden l{\"{a}}sst oder mehrfach sendet. 
\todo{aufteilung im netzwerk als Bild}

Der Voter jedes simulierten Roboters ist dabei, wie auch die Simulation und die Anzeige, im nichtverf{\"{a}}lschenden Netzwerk. Dies entspricht der Realit{\"{a}}t, in der die physikalischen Bewegungen
des Roboters auch einfach umgesetzt werden. Die Controller allerdings k{\"{o}}nnen sich nicht darauf verlassen, dass ihre Daten nicht verf{\"{a}}lscht wurden, sie m{\"{u}}ssen also von den Studenten so
programmiert werden dass sie mit \textit{value faults/message corruption}\cite{richling-ass} umgehen k{\"{o}}nnen.

\begin{figure}
	\centering
	\includevisio[width=\textwidth]{Netzwerk}
	\caption{Aufteilung der Netzwerkteilnehmer}
	\label{fig:network}
\end{figure}

\subsection{VMs}

\subsection{Anzeige}\label{graphics}
Um den Status der Simulation zu verfolgen, wird die simulierte Welt und die in ihr enthaltenen Objekte angezeigt. Dies geschieht mit dem gleichen Programm das die Simulation auch tats{\"{a}}chlich
durchf{\"{u}}hrt, die Unity engine stellt die ben{\"{o}}tigte Funktionalit{\"{a}}t f{\"{u}}r beide Aufgaben bereit.

\paragraph{Kamera} Die Sicht auf die simulierte Welt geschieht durch das Unity Objekt "Camera". An diese werden verschiedene Skripte angegeh{\"{a}}ngt um z.B. dem Benutzer die M{\"{o}}glichkeit zu
geben die Kamera zu bewegen oder \todo{halt noch irgendwas}.

Die Kamera kann {\"{u}}ber die Pfeiltasten bewegt werden, ein Kippen der Kamera ist nicht implementiert - dies w{\"{u}}rde auch verwirren, da damit eventuell der Kippwinkel der
Platte visuell ausgeglichen werden k{\"{o}}nnte.

\paragraph{Lichtquelle} Um die Szene auszuleuchten, gibt es ein Spotlight. Dies befindet sich an der Stelle \todo{nachgucken} und scheint mit einem XXXX\textdegree Winkel nach unten.


\clearpage
\section{Interface f{\"{u}}r die Studenten}\label{interface}
Damit die Studenten sich auf die Implementierung der Fehlertoleranz konzentrieren k{\"{o}}nnen, gibt es Schnittstellen.
Im ganzen System gibt es zwei Schnittstellen:
\begin{itemize}
\item Die Schnittstelle zwischen Controller und Voter
\item Die Schnittstelle zwischen Voter und Roboter/Simulation
\end{itemize}

\paragraph{Die Schnittstelle zwischen Controller und Voter} Den Studenten wird nicht vorgegeben wie die Kommunikation zwischen den Controllern und Votern aussehen soll - denn gerade hier soll ja die Fehlertoleranz implementiert werden.

\paragraph{Die Schnittstelle zwischen Voter und Roboter} Diese Schnittstelle besteht aus den Funktionen:
\begin{lstlisting}[frame=single, language=c] 
void* connectToWorld();
void detachFromWorld(void* ctx);
int createRobot(void* ctx);
typedef void (*TypeGetWorldStatusCallback)(WorldStatus ws, void* optional);
int startProcessingWorldEvents(void* ctx, TypeGetWorldStatusCallback cb, void* optional);
void moveRobot(void* ctx, int id, float speed, float angle);
\end{lstlisting}

Diese wird den Studierenden als kompilierte Library mit einem detailliert kommentierten Headerfile zur Verf{\"{u}}gung gestellt und kann
dann vom Studentencode einfach aufgerufen werden.

\clearpage
\section{Fehlerinjektion}
Da es in der Simulation, anders in einer realen Welt, keine normalen Fehlerquellen gibt, wird eine Fehlerinjektion benutzt, um die Fehlertoleranz zu testen.
In diesem Fall wird simuliert, dass das Netzwerk Daten fehlerhaft weiterleitet (wie es z.B. ein Funknetzwerk tuen w{\"{u}}rde) und das die Controllerprogramme
unvorhersehbar ausfallen - was nachbildet, dass z.B. der Controller seine Batterie entladen hat.

\todo{bild? wovon wollte ich hier ein bid haben?}

\subsection{Netzwerk}
Alle UDP Pakete k{\"{o}}nnen verf{\"{a}}lscht werden. Die einzelnen Pakete werden dabei so manipuliert, dass durchschnittlich in $ ^1/_{20} $ aller Pakete
ein Byte verf{\"{a}}lscht wird, ohne dass das Betriebssystem diese Modifikation erkennt. Diese Erkennung muss also von den Studenten selbst implementiert werden.

\subsection{Controller}
Die Controllerprogramme werden zu zuf{\"{a}}ligen Zeitpunkten beendet und gestartet - dabei wird das Programm einfach mit SIGKILL beendet und nicht vorgewarnt. Dies simuliert einen \textit{crash failure}. Mit einer Wahrscheinlichkeit von 33\% werden dem Programm auch nur eingeschr{\"{a}}nkte Ressourcen zugeteilt, z.B. darf es nur eine bestimmte Anzahl an Dateideskriptoren gleichzeitig
offen haben oder nur eine eingeschr{\"{a}}nkte Menge Speicher benutzen.
\lstinputlisting{../fault_injector.sh}

\subsection{Simulation}
Der Benutzer kann einen Roboter fernsteuern und hat damit direkten Einfluss auf die simulierte Welt. W{\"{a}}hrend ein Roboter
ferngesteuert wird, f{\"{u}}hrt dieser eine Roboter keine Befehle seines Voters mehr aus. Abgesehen davon verh{\"{a}}hlt sich der Roboter normal, es wird also die gleiche Menge Energie verbraucht, es gelten die gleichen Einschr{\"{a}}nkungen bei
Geschwindigkeit und Rotation, und so weiter.

Die Roboter werden {\"{u}}ber die Pfeiltasten gesteuert (Vorw{\"{a}}rts beschleunigt, Links und Rechts drehen den Roboter). Um
anzuzeigen welcher Roboter gerade gesteuert wird wird dieser von oben angeleuchtet. Mit der Tabulatortaste kann zwischen den
Robotern umgeschaltet werden, mit der Escapetaste wird die Steuerung der Roboter aufgehoben; falls die Fernsteuerung gerade nicht aktiv ist, wird dies in der Simulation angezeigt.

\clearpage
\section{Beispielimplementation}
Um das Prinzip dieser Simulation, ob den Studierenden des Moduls Ausfallsichere Systeme oder Besuchern, zu verdeutlichen ist ein Teil der Bachelorarbeit die Implementierung einer
Beispielimplementation.

\paragraph{Ablauf} Jede Viertelsekunde sendet die Welt Statusinformationen aus. \todo{aeh damit ist es dann doch iwie wieder rundenbasiert?} Diese Informationen werden von den Votern empfangen, k{\"{o}}nnen
aber auf dem {\"{U}}bertragungsweg zum Controller potenziel verf{\"{a}}lscht worden sein. Also wird f{\"{u}}r jedes Objekt der Welt der Konsensalgorithmus ausgef{\"{u}}hrt. Wenn ein Konsens {\"{u}ber den Status der Welt hergestellt wurde, berechnet 
jeder Controller f{\"{u}}r jeden Roboter die n{\"{a}}chste Bewegung. Diese wird dann an den Voter geschickt, der ein einfaches Mehrheitsvotum durchf{\"{u}}hrt und diese Bewegung an die Welt weitergibt.

\subsection{Fehlermodell} \label{error-model}
Bei der Planung eines ausfallsicheren Systems ist es besonders wichtig zu definieren, welche Art von Fehlern
{\"{u}}berhaupt korrigiert / abgefangen werden soll. F{\"{u}}r dieses Projekt werden f{\"{u}}r jede Fehlerklasse
die m{\"{o}}glichen Fehler aufgelistet und beschrieben, ob und wie sie gel{\"{o}}st werden.

\paragraph{Crash failure} Die Controller k{\"{o}}nnen jederzeit ausfallen. Im Extremfall k{\"{o}}nen alle Controller ausfallen, die Voter sind, per Definition, gegen diese Art von Fehlern unempfindlich.

\paragraph{Value error} Das Netzwerk kann Pakete verf{\"{a}}lschen, es wird angenommen, dass bei bis zu $^1/_{20}$ aller Pakete Verf{\"{a}}lschungen geben kann, diese sich allerdings auf \textit{Single Byte Errors} beschr{\"{a}}nkt.
Dar{\"{u}}ber hinausgehende Verf{\"{a}}lschungen werden nicht erkannt und f{\"{u}}hren zu einem \textit{silent failure}; \todo{auswirkungen}

\subsubsection{R{\"{a}}umliche Redudanz}
Es wird davon ausgegangen, dass die Controller sehr fehleranf{\"{a}}lig sind und leicht ausfallen - daraus folgt das eine gro{\ss}e
Anzahl an Controllern f{\"{u}}r jeden Roboter kontrollieren muss; nur falls weniger als $X$ Controller dieses Roboter noch aktiv
sind kann eine ordnungsgem{\"{a}}{\ss}e Steuerung nicht mehr garantiert werden.

Wenn ein Roboter erstellt wird, werden auch $Y$ Controller gestartet. Die Ausf{\"{u}}hrung der Controller wird durch ein Skript
{\"{u}}berwacht das ausgefallene Controller neu startet. \todo{ist das sinnvoll? dann muss man ja auch das skript killen um dann
irgendwann mal zu wenig controller zu haben}

\subsubsection{Netzwerkkommunikation}
Da durch die Fehlerinjektion das Netzwerk UDP Pakete verf{\"{a}}lscht m{\"{u}}ssen alle Daten mit einer Kanalkodierung versehen werden. Hier wird ein (255, 240) Reed-Solomon Code benutzt, also ein Code, der 15 parity bits pro 240 Datenbits benutzt.
Durch die Benutzung dieser Kodierung k{\"{o}}nnen alle Einzelfehler und Doppelfehler erkannt und korrigiert werden. Erst ab 7 Fehlern ist eine Korrektur nicht mehr m{\"{o}}glich, ab 15 Fehlern versagt auch eine Fehlererkennung.
Diese Kodierungsart wurde aus zwei Gr{\"{u}}nden gew{\"{a}}hlt:
\begin{itemize}
\item Durch expermientelle Verifikation wurde klar, dass im Netzwerk nur 1 Byte pro Paket verf{\"{a}}lscht wird. Da das durchschnittliche Paket um mindestens Faktor
	100 gr{\"{o}}{\ss}er ist, ist nicht erforderlich, eine sehr kompakte Kodierung zu benutzen, es ist ausreichend, die Redudanz zu reduzieren. 
\item Es ist ein systematischer Code, der es erlaubt, w{\"{a}}hrend des laufenden Betriebes die Pakete mitzuschneiden
	und sich die Daten anzugucken. Dies vereinfacht die Fehlersuche - beispielsweise bei einem Viterbicode w{\"{a}}re dies nicht m{\"{o}}glich, dort
	sind Nutz- und Kodierungsdaten nicht klar unterscheidbar.
\end{itemize}

\subsection{Voter}
\label{voter}
Der Roboter kann nur direkt durch den Voter gesteuert werden. Der Voter ist dabei nur daf{\"{u}}r zust{\"{a}}ndig aus den vielen Steuerkommandos die ankommen die Mehrheit zu bilden (\textit{N-modular reduancy}).
Da nicht klar ist wie viele Controller momentan {\"{u}}berhaupt Steuerkommandos senden k{\"{o}}nnten, kann nicht gewartet werden bis eine bestimmte Anzahl von Kommandos empfangen wurde. Deswegen wird beim empfangen
eines neuen Weltstatus das Steuerergebniss gebildet, weggesendet und der Vorgang wird neu gestartet.

Um das Steuerergebniss zu finden werden die empfangenen Steuerkommandos sortiert und der Medianwert genommen (\textit{Mid-Value Selection}). Da die Steuerkommandos
Flie{\ss}kommazahlen enthalten ist keine exakte Votierung m{\"{o}}glich. Die Wahl des Medians f{\"{u}}hrt zus{\"{a}}tzlich dazu das weniger "extreme" Bewegungen bevorzugt werden.
\noindent\begin{minipage}{.30\textwidth}
\begin{lstlisting}[caption=Sammeln, frame=tlrb, language=c++]
votes.res[id].push_back(Vector{x, y});
\end{lstlisting}
\end{minipage}\hfill
\begin{minipage}{.60\textwidth}
\begin{lstlisting}[caption=Auswahl, frame=tlrb, language=c++]
/* go through all robots */
for(auto&& votes : votes.res) {
  /* vote */
  std::sort(std::begin(votes.second), std::end(votes.second));
  auto x = votes.second[votes.second.size() / 2];

  /* send */
  int r = 0;
  if((r = moveRobot(info->worldCtx, votes.first, x.x_, x.y_)) < 0) {
  	fprintf(stderr, "can't move robot: %d", r);
  }
}
\end{lstlisting}
\end{minipage}


\begin{figure}
	\centering
	\includevisio[width=\textwidth]{seqvoter}
	\caption{Ablaufdiagramm Voter}
	\label{fig:sequence-voter}
\end{figure}
\clearpage % make sure the table is, at least, in the right section

\subsection{Controller}\label{controller}
Die Roboter m{\"{u}}ssen sich so bewegen das die Platte m{\"{o}}glichst gut ausbalanciert
ist und gleichzeitig die Roboter nicht ihre ganze Tankf{\"{u}}llung verbrauchen. Gleichzeitig sollten die Roboter auch nicht miteinander
kollidieren.

\paragraph{Algorithmus zur Bestimmung der Bewegung} Der implementierte Algorithmus unterschiedet zwischen 3 Situation: ist der Zustand des Roboters kritisch,
unkritisch oder dazwischen? Dabei ist das Kriterium die momentane Tankf{\"{u}}llung geteilt durch die Entfernung zur F{\"{u}}llstation. Roboter mit fast leerem Tank
versuchen also sich aufzuf{\"{u}}llen, Roboter mit fast vollem Tank versuchen auszubalancieren.

Jeder Controller berechnet f{\"{u}}r jeden Roboter wie dieser sich bewegen soll. Dabei wird mit dem Roboter im kritischtesten Zustand anfangen, denn dieser muss ja 
unbedingt Richtung F{\"{u}}llstation und die weniger kritischen k{\"{o}}nnen/m{\"{u}}ssen dann ausgleichen. Nachdem eine Roboterbewegung berechnet wurde, werden die Weltkippwinkel
dementsprechend neu berechnet - so dass die letzten Roboter die Bewegungen der vorher berechneten mit in ihre Entscheidung einbeziehen.

\paragraph{Bewegung} Um zu einem bestimmten Punkt hinzufahren (z.B. zur F{\"{u}}llstation) wird zu erst die Rotation des Roboters in einen Bewegungsvektor umgerechnet. Die Rotation
wird immer in Bezug auf eine feste Richtung angegeben, also kann der Bewegungsvektor durch trigeometrische Funktionen bestimmt werden:
$$ M_i = \bigl(\begin{smallmatrix} sin(R_i) \\ cos(R_i) \end{smallmatrix}\bigr)$$
Desweiteren gibt es einen Distanzvektor zum Ziel:
$$ D_i = F - P_i $$

Der Winkel zwischen diesen beiden Vektoren gibt an wie weit der Roboter sich drehen muss um gerade zu dem Ziel fahren zu k{\"{o}}nnen. Dabei muss beachtet werden das nur der Wert des
Winkels berechnet wird, das Vorzeichen allerdings nicht. Der Roboter w{\"{u}}rde also immer nach rechts drehen, damit den Winkel vergr{\"{o}}{\ss}ern und noch st{\"{a}}rker in die
falsche Richtung drehen. Das Vorzeichen kann allerdings berechnet werden in dem man die Bewegungsvektoren in drei Dimensionen betrachtet und das Keuzprodukt bildet. In dem resultierenden
Vektor gibt das Vorzeichen der Z Kompenente das Vorzeichen des Winkels an.

\paragraph{Ausbalancieren} Um auszubalancieren wird 


\clearpage
\section{Evaluation}


\clearpage
\printbibliography

\end{document}
