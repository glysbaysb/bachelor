\section{Physikalische Grundlagen}
In diesem Kapitel werden die ben{\"{o}}tigten physikalischen Grundlagen erl{\"{a}}utert.

\subsection{Bewegungsmuster}
In der Simulation sollen sich Roboter bewegen, daf{\"{u}}r muss festgelegt werden, wie diese sich bewegen. Denkbar w{\"{a}}ren verschiedene Bewegungsmuster, zum
Beispiel der aufrechte bipedale Gang, eine Art "Hovercraft", ein zweiachsiges Auto, ein Rollstuhl und viele weitere.

Hier wird das \textit{differential steering} Modell vorgestellt und auch genutzt; es ist weit verbreitet, leicht verst{\"{a}}ndlich, hat aber trotzdem bestimmte
Limitierungen, die eine Regelung n{\"{o}}tig machen.

\subsubsection{Differential Steering}\label{diffs}
Beim \textit{differential steering} k{\"{o}}nnen die R{\"{a}}der nicht, wie z.B. beim Auto, geschwenkt werden -- sie sind fest. Eine Rotation wird durch eine unterschiedliche Rotationsgeschwindigkeit der R{\"{a}}der
erm{\"{o}}glicht; bekannt ist dieses Prinzip vom Rollstuhl. Es kann zwischen drei Bewegungsarten unterschieden werden:
\begin{itemize}
\item Falls die R{\"{a}}der gleich schnell drehen, f{\"{a}}hrt der Roboter gerade nach vorne oder hinten.
\item Falls die R{\"{a}}der unterschiedlich schnell rotieren wird eine Kurve gefahren. Dabei dreht sich der Roboter in Richtung des langsamer rotierenden Rades.
\item Falls die R{\"{a}}der gleich schnell, aber in unterschiedliche Richtung rotieren, wendet der Roboter auf der Stelle.
\end{itemize}

Diese Bewegungen k{\"{o}}nnen auch mathematisch ausgedr{\"{u}}ckt werden. Dabei gibt $V_r(t)$ die Rotationsgeschwindigkeit des rechten Rades zum Zeitpunkt t an, w{\"{a}}hrend $V_l(t)$ das
linke Rad beschreibt. \gls{R} ist der Radius eines Rades, und \gls{L} die Distanz zwischen den R{\"{a}}dern.
Folgende Gleichungen aus \cite{Dudek2010, Egerstedt}:
\begin{subequations}\label{eq:diffs}
\begin{align}
	\Theta(t) &= \Theta(0) + \frac{R}{L} \int_0^t \mathrm{v_R(t) - v_L(t)}\mathrm{d}t \label{eq:diffs-theta}\\
	\intertext{Mithilfe der Orientierung ($\Theta$) zum Zeitpunkt $t$ ist es m{\"{o}}glich die "Fahrtrichtung" zum
	Zeitpunkt $t$ zu bestimmen. Eine Multiplikation dieser mit den Geschwindigkeitsvektoren ergibt den
	Bewegungsvektor zum Zeitpunkt $t$ - daraus l{\"{a}}sst sich die Position ermitteln:
}
	x(t) &= x(0) + \frac{1}{2} \int_0^t \mathrm{\big(v_R(t) + v_L(t)\big) \times \cos\big(\Theta(t)\big)}\mathrm{d}t \label{eq:diffs-x}\\
	y(t) &= y(0) + \frac{1}{2} \int_0^t \mathrm{\big(v_R(t) - v_L(t)\big) \times \sin\big(\Theta(t)\big)}\mathrm{d}t \label{eq:diffs-y}
\end{align}
\end{subequations}


\subsubsection{Wegfindung}
Da ein \textit{differential steering} Roboter sich nicht-holonomisch (ohne Beschr{\"{a}}nkung auf allen Achsen) bewegen kann, wird die Wegfindung
erschwert. Zur L{\"{o}}sung dieses Problems (auch bekannt als inverses kinematisches Problem) k{\"{o}}nnte der Roboter sich erst auf der Stelle
drehen, bis er das Ziel "anguckt" und sich dann gradlinig auf dieses zubewegen.

Falls einer der beiden Robotermotoren allerdings nicht vollst{\"{a}}ndig funktionst{\"{u}}chtig ist, ist dies nicht m{\"{o}}glich. Eine m{\"{o}}gliche
Alternative w{\"{a}}hre es eine Kurve zu fahren, dies muss allerdings -- abh{\"{a}}ngig von der zur{\"{u}}ckzulegenden Strecke -- nicht immer m{\"{o}}glich
sein.

\paragraph{Bewegung zu einem Punkt} Um zu einem bestimmten Punkt hinzufahren (z.B. zur Ladestation),
wird zu erst die Rotation des Roboters in einen Bewegungsvektor umgerechnet. Die Rotation wird immer in Bezug
auf eine feste Richtung angegeben, also kann der Bewegungsvektor durch trigonometrische Funktionen bestimmt werden:
\begin{equation}
\label{eq:M}
	\vec{M_i} = \bigl(\begin{smallmatrix} cos(R_i) \\ sin(R_i) \end{smallmatrix}\bigr)
\end{equation}
Die n{\"{a}}chste Bewegung gerade nach vorne wird dann anhand dieser Werte "aufgeteilt". Falls der Roboter
in einem 90\textdegree Winkel nach rechts "guckt" ($M_i = \bigl(\begin{smallmatrix} 0 \\ 1 \end{smallmatrix}\bigr)$)
w{\"{u}}rde er sich beim nach vorne fahren auf der Y-Achse nach rechts bewegen.

Die Distanz zum Ziel l{\"{a}}sst sich trivial durch Vektorsubtraktion berechnen: $ \vec{D_i} = \vec{\gls{F}} - \vec{N_i} $

Der Winkel zwischen diesen beiden Vektoren gibt an, wie weit der Roboter sich drehen muss um gerade zu dem Ziel
fahren zu k{\"{o}}nnen. Dabei muss beachtet werden, dass nur der Wert des Winkels berechnet wird, das
Vorzeichen allerdings nicht. Der Roboter w{\"{u}}rde also immer (egal ob das Ziel links oder rechts liegt) nach
rechts drehen, damit den Winkel vergr{\"{o}}{\ss}ern und noch st{\"{a}}rker in die falsche Richtung drehen.
Das Vorzeichen kann allerdings berechnet werden, in dem man die Bewegungsvektoren in drei Dimensionen betrachtet
und das Keuzprodukt bildet. Dabei gibt das Vorzeichen der Z Kompenente des resultierenden Vektors das Vorzeichen
des Winkels an.

\subsection{Physiksimulation der Platte}
Das zu erreichende Ziel ist, die Platte m{\"{o}}glichst "still" in einem Gleichgewicht zu halten, also
die Kippwinkel und die {\"{A}}nderungsrate des Kippwinkels zu minimieren.

Dabei ist der Kippwinkel abh{\"{a}}ngig von der Verteilung der Gewichte auf der Platte, w{\"{a}}hrend die
{\"{A}}nderungsrate von den auf die Platte wirkenden Kr{\"{a}}ften abh{\"{a}}ngt. Diese beiden Unterscheidungen
werden in den nachfolgenden Kapiteln behandelt.

\subsubsection{Statisches Gleichgewicht}
Die Platte ist im statischen Gleichgewicht wenn alle Kr{\"{a}}fte/Gewichte (der Objekte, also der F{\"{u}}llstation und aller Roboter) die auf sie wirken, sich zu Null aufaddieren.
Dabei werden die Objekte als Vektoren, bestehend aus Gewicht und Abstand zum Mittelpunkt betrachtet.

Um das Kippen der Platte auf der links-rechts und vorne-hinten Achse getrennt angeben zu k{\"{o}}nnen, werden diese getrennt voneinander ausgerechnet. Dabei wird das Hebelgesetz
im eindimensionalen Fall (wie auf einer Wippe) angewandt, also Distanz mal Masse ergibt die Kraft in dieser Richtung
\footnote{Da die Positionsbestandteile (x und y) einzeln betrachtet werden, muss der Winkel nicht in die Rechnung einflie{\ss}en.}.

F{\"{u}}r die in Links-Rechts wirkenden Kr{\"{a}}fte gilt Gleichung ~\ref{eq:forces:x}, f{\"{u}}r die Kr{\"{a}}fte die beschreiben wie die Platte nach vorne-hinten gekippt
gilt Gleichung ~\ref{eq:forces:y}.
\begin{subequations}\label{eq:forces}
\begin{align}
	V(x) = x(F) \times w(F) + \sum_{i=0}^{|\gls{N}|} ( x(N_i) \times w(N_i) ) \label{eq:forces:x}\\
	V(y) = y(F) \times w(F) + \sum_{i=0}^{|N|} ( y(N_i) \times w(N_i) ) \label{eq:forces:y}
\end{align}
\end{subequations}

Das Ziel ist es, die Platte auszubalancieren, also soll gelten:
\begin{equation}\label{eq:gleichgewicht}
	V(x) = V(y) = 0
\end{equation}

\subsubsection{Dynamisches Gleichgewicht}
Eine Geschwindigkeits{\"{a}}nderung der Roboter erzeugt einen Impuls, auch dieser wirkt auf die Platte. Die Energie des Impulses ist proportional zur
{\"{A}}nderung und schwingt dann langsam ab. Da ein geschlossenes System simuliert wird, ist dem Impuls eine gleich gro{\ss}e Kraft entgegen gerichtet.
Diese beiden Kr{\"{a}}fte f{\"{u}}hren zu einer (abklingenden) Schwingung. Mathematisch ausgedr{\"{u}}ckt:
\begin{equation}\label{eq:schwingung}
	s(t) = e^\mathrm{-\delta t} \times sin(\omega_d t) \times s(0)
\end{equation}
\cite{wiki:schwingung}
\footnote{$\delta$ und $\omega_d$ sind in der Simulation frei einstellbar}

Hierbei ist $s(t)$ ein Vektor, in dem die Geschwindigkeits{\"{a}}nderung in X-, Y- und Z-Richtung die jeweiligen Elemente sind. Um die gesamte Schwingung der
Platte zu bestimmen, werden die Ergebnisvektoren der einzelnen Roboter aufaddiert.

\clearpage
\section{Game Engine}
\subsection{Anforderungen}
Die Simulation, vor allem die Bewegung der Roboter, kann besonders gut visuell dargestellt werden.
Auch wenn sich das Geschehen rein auf einer zweidimensionalen Ebene stattfindet, ist eine dreidimensionale
Darstellung ansprechender und wird deshalb vorgezogen.

Diese Darstellung geschieht mithilfe einer \textit{3D Engine}; sie bietet die M{\"{o}}glichkeit Objekte und oft 
auch Lichtquellen und Schattenspiel darzustellen. Anstatt solch eine Engine selbst zu implementieren, wird eine 
vorhandene genommen.

Ein weiterer Teil der Simulation ist die Physiksimulation an sich. Auch diese muss nicht selbst programmiert
werden, es gibt verschiedene fertige. \textit{Game engines} kombinieren \textit{3D Engine} und Physikengine;
aus diesem Grund wird eine \textit{game engine} benutzt um die Simulation zu erstellen. Um die sehr gro{\ss}e 
Auswahl einzuschr{\"{a}}nken, wurden die nachfolgenden Kriterien benutzt.

\paragraph{Kosten} Die 3D Engine muss f{\"{u}}r den nicht-kommerziellen edukativen Gebrauch kostenlos sein.

\paragraph{Plattformunabh{\"{a}}ngigkeit} Die mit der \textit{game engine} entwickelte Simulation muss auch
ohne gro{\ss}en Aufwand auf mehrere System (hier Windows und Linux) portierbar sein.

\paragraph{Einfache Benutzung} Das Hauptkriterium ist die einfache Benutzung. Dies beeinhaltet zum einen die
tats{\"{a}}chliche Nutzung (also: gibt es einen graphischen Leveleditor? Wie kann die Interaktionen zwischen
Objekten gesteuert werden?), zum anderen die Qualit{\"{a}}t der Dokumentation. Und schlie{\ss}lich wie verbreitet diese
\textit{engine} ist, und damit auch wie einfach es ist, bei Problemen Hilfestellungen zu bekommen.

Es gibt viele \textit{game engines}, genannt seien hier zum Beispiel Cryengine\cite{cryengine}, Unity, Blender, Source
oder Cube. Bis auf die Cryengine w{\"{a}}hren alle kostenlos; auch die Plattformunabh{\"{a}}ngigkeit ist bei
fast allen gegeben. Doch gerade beim letzten Kriterium stach Unity hervor. Nicht nur gibt es eine sehr gro{\ss}e
Community, welche bei Problemen helfen kann, eine gute Dokumentation, auch es ist auch m{\"{o}}glich in bekannten
(normalen) Programmiersprachen die Engine zu steuern. Beispielsweise wird die Unreal Engine mit UnrealScript
programmiert, w{\"{a}}hrend Unity in C\# oder JavaScript programmiert werden kann. Daher baut die Simulation auf
Unity auf.

\subsection{Unity}\label{unity}
Unity ist eine von Unity Technologies entwickelte \textit{game engine}, die 2005 f{\"{u}}r Macintosh entwickelt wurde und in der Zwischenzeit auf 27 Plattformen portiert wurde.

In dieser Arbeit wird es benutzt, um die Welt und die Interaktionen zwischen den Objekten zu simulieren und diese auch anzuzeigen. Mithilfe einer Scriptingschnittstelle kann das Geschehen modifiziert werden, z.B. kann die Welt
abh{\"{a}}ngig von der Position und den Gewichten der Objekt gekippt werden oder die Kamera durch Tastatureingaben bewegt werden.

\subsubsection{Grafikengine}
Die in Unity enthaltene Grafikengine ist eine vollwertige 3D-Grafikengine, die es erm{\"{o}}glicht professionelle Videospiele zu programmieren. In dieser Arbeit wird nur
ein kleiner Teil der Funktionalit{\"{a}}t benutzt; komplexere Features wie die Anzeige von Rauch oder realistischem Wasser sind nicht in der Simulation enthalten.

Notwendig ist es aber, dass die verschiedenen Objekte in der simulierten Welt in 3D zu sehen sind und sich die Kamera bewegen l{\"{a}}sst, um das simulierte Geschehen zu
verfolgen. Dies geh{\"{o}}rt zur Grundfunktionalit{\"{a}}t und muss nicht extra implementiert werden.

\subsubsection{Physikengine}
Ein integraler Teil moderner Videospiele ist eine realistische physikalische Umgebung. Dies wird durch die Physikengine von Unity erm{\"{o}}glicht, die in dieser Arbeit wichtigen Funktionalit{\"{a}}ten sind:

\begin{itemize}
\item die Kollisionserkennung, einerseits um zu bestrafen wenn Roboter in einander fahren,
	und andererseits zur Erkennung des Kontaktes mit der F{\"{u}}llstation
\item die Bewegung der Objekte in der Welt zu simulieren, abh{\"{a}}ngig von ihrem Gewicht und ihrer Geschwindigkeit
\end{itemize}

\subsubsection{Scripting bei Unity}
Alle Elemente der Unity Engine k{\"{o}}nnen vom Programmierer gesteuert werden, daf{\"{u}}r k{\"{o}}nnen
die Skriptsprachen C\#, JavaScript und Boo genutzt werden.\cite{wiki:unity}
Um dies zu erm{\"{o}}glichen, ist es aus dem Skript heraus m{\"{o}}glich, alle Eigenschaften
der Objekte \footnote{Dies beinhaltet nicht nur die Roboter und F{\"{u}}llstationen sondern auch die Kamera
oder statische Objekte wie Mauern.} auszulesen und zu manipulieren.

Beispielsweise wurde der F{\"{u}}llstation ein Skript zugeordnet, dass solange ein Objekt mit dieser kollidiert, testet, ob dies Objekt ein Roboter ist, und falls dies der Fall ist,
den Roboter auftankt. Ein anderes Skript bewegt die Kamera abh{\"{a}}ngig von den Tastatureingaben des Benutzers.

Um die Skripte in Unity zu integrieren, m{\"{u}}ssen sie bestimmte Eigenschaften haben. Alle Klassen\todo{funktionen in diesen}, die von UnityEngine.MonoBehaviour
abstammen und registriert sind, werden w{\"{a}}hrend der Berechnung jedes Frames aufgerufen. Der genau Zeitpunkt dieses Aufrufes h{\"{a}}ngt von
der Funktion ab\footnote{Vgl. \cite{unityexecorder}}, beispielsweise wird Update() genau einmal pro Frame aufgerufen nachdem die Physikengine fertig ist
und Inputevents behandelt wurden. Im Gegensatz dazu kann FixedUpdate() mehrmals oder sogar gar nicht w{\"{a}}hrend der Renderung eines Frames aufgerufen
werden, abh{\"{a}}ngig von der Framerate. Daf{\"{u}}r wird FixedUpdate() vor der Physikengine aufgerufen, {\"{A}}nderungen, die in dieser
Funktion vorgenommen werden, haben also schon Auswirkungen auf den momentanen Frame. Desweiteren muss der Klassenname innerhalb einer Datei so
hei{\ss}en wie die Datei selbst.

Da Konstruktor und Dekonstruktor nicht {\"{u}}berschrieben werden k{\"{o}}nnen, gibt es die virtual functions Start() und OnDestroy(). Diese
{\"{u}}bernehmen die Aufgaben des Konstruktor und Dekonstruktors.

Es gibt auch weitere {\"{u}}berschreibbare Funktionen, mit denen es m{\"{o}}glich ist, sich in verschiedene Zwischenschritte des Renderprozesses
einzuklinken (zum Beispiel OnWillRenderObject, OnPostRender, ...). Das gleiche gilt auch f{\"{u}}r Inputevents (beispielsweise OnMouseMove) und
die Ereignisse der Physikengine (OnCollisionStart(), ...).

\clearpage
\section{Kommunikation}

\subsection{Interaktionsmuster}
In einem verteilten System (wie hier) ist es n{\"{o}}tig, dass die Kommunikationspartner miteinander kommunizieren.
Dies ist auf verschiedene Arten m{\"{o}}glich, daher werden hier exemplarisch einige dargestellt und daraus eine
Auswahl getroffen um die speziellen Anforderungen dieses Systems zu erf{\"{u}}llen.

\subsubsection{Message Passing}
Beim Kommunizieren miteinander werden Informationen ausgetauscht. Die simpelste Art der Kommunikation zwischen
Programmen ist das \textit{message passing}, hierbei schickt ein Kommunikationspartner eine Nachricht, die an einen bestimmten anderen
Kommunikationspartner adressiert ist.

Bei diesem Interaktionsmuster ist es die Aufgabe der Programmierer, die Nachricht zu interpretieren
(also die {\"{U}}bersetzung in ein verst{\"{a}}ndliches Format, zum Beispiel unter Beachtung der Byte-Reihenfolge) und
basierend auf den ausgetauschten Nachrichten zum Beispiel eine Frage - Antwort Beziehung herzustellen.

\subsubsection{Request-Reply}
Bei einer Kommunikation zwischen zwei Menschen / System, oder allgemeiner: Kommunikationspartnern,
ist meistens so, dass einer redet, dann die andere Person antwortet. Es gibt also \textit{request}
und \textit{reply}.

Diese Form der Kommunikation ist gut geeignet f{\"{u}}r Client-Server Kommunikation wie das
Aufrufen von Websites. Allerdings kann diese Art von Kommunikation zu Problemen f{\"{u}}hren, falls sich die
Informationen schnell {\"{a}}ndern -- die verschiedenen Clients fragen den Server leicht zeitversetzt an und
agieren dann vielleicht basierend auf leicht unterschiedlichen Informationen.

Um das \textit{Request-Reply} Interaktionsmuster einfacher an die verschiedenen Anforderungen verschiedener
Probleme anzupassen, gibt es die \textit{Remote Procedure Calls} als Spezialisierung des \textit{request-
reply}. Dabei wird versucht, die ganze Kommunikation so weit zu verstecken und abstrahieren, das es so wirkt
als k{\"{o}}nnten Funktion \textit{remote} (also in einem anderen Programm auf einem anderen Rechner) wie
lokale Funktionen aufgerufen werden. Dies erleichtert die Arbeit der Programmierer enorm, sie m{\"{u}}ssen
sich (im besten Fall) gar keine Gedanken mehr {\"{u}}ber die Kommunikation machen und es werden nur
ohnehin bekannt Sprachkonstrukte (der Aufruf von Funktionen) genutzt.
\footnote{Gerade bei der Programmierung von fehlertoleranten Systemen kann diese Abstraktion allerdings
kontraproduktiv sein. Beispielsweise wird durch die Abstraktion auf einen Funktionsaufruf die M{\"{o}}glichkeit
einen Timeout vorzugeben "verbaut".}

F{\"{u}}r die Programmierenden hat dieses Modell den Vorteil, dass die Nachricht des anderen Systems schon
interpretiert wurde, es muss also weniger Quellcode selbst geschrieben werden.

\subsubsection{Publish-Subscribe}
Die andere intuitiv verst{\"{a}}ndliche Kommunikationsform ist der Vortrag.

Die Simulation muss mit einer nicht vorab bestimmbaren Anzahl von Votern/Robotern arbeiten.
Die Kommunikationspartner k{\"{o}}nnen auch jederzeit ausfallen oder es k{\"{o}}nnen neue hinzukommen.
Eine erste L{\"{o}}sungm{\"{o}}glichkeit w{\"{a}}re es, die Kommunikation {\"{u}}ber UDP Broadcasts
laufen zu lassen. Dadurch w{\"{u}}rden alle Kommunikationspartner automatisch die Nachricht empfangen, ohne
dass sich der Sender darum k{\"{u}}mmern muss, an wen Nachrichten gesendet werden m{\"{u}}ssen. Da die
Voter aber in unterschiedlichen Subnetzen sein k{\"{o}}nnen, ist dies nicht m{\"{o}}glich.

Abhilfe schafft das Publish-Subscribe Kommunikationspattern\cite{pubsub}. Dort gibt es einen Publisher (in
diesem Fall die Simulation) und Subscriber (die Voter/Roboter). Die Subscriber verbinden sich mit dem
Publisher und teilen ihm mit, welche Art von Nachrichten sie interessiert. Der Publisher verteilt dann die
Nachrichten, die zu nicht vorhersehbaren Zeitpunkten kommen, anhand dieser Informationen an alle Subscriber,
zu denen momentan ein Kontakt m{\"{o}}glich ist.

Dies hat drei gro{\ss}e Vorteile. Es gibt eine r{\"{a}}umliche Trennung, das hei{\ss}t der Publisher muss
nichts {\"{u}}ber die Subscriber (zum Beispiel Anzahl, IP Adresse) wissen. Das gleiche gilt auch andersherum.
Es reicht
wenn beide wissen, wo sie sich "treffen" k{\"{o}}nnen. Als zweites gibt es eine zeitliche Trennung zwischen
den Kommunikationspartner -- der Publisher kann Nachrichten verschicken, ohne sich dar{\"{u}}ber Gedanken
machen zu m{\"{u}}ssen, wann diese beim Subscriber ankommen, w{\"{a}}hrend es f{\"{u}}r den Subscriber
uninteressant ist, wann die Nachricht versendet wurde. Der letze Vorteil ist die Entkoppelung von
Nachrichten{\"{u}}bertragung und den anderen Aufgaben des Programmes.
\todo{git blame hier stand mal was vernuenftiges}
Die Nachrichten werden Daten
von ihrer Herstellung verschickt, das Senden einer Nachricht ist also (aus Sicht des
Publishers) O(1) in Bezug auf die Anzahl der Subscriber. Auch auf Seiten des Subscribers gibt es eine
Asynchronit{\"{a}}t, die empfangen Daten werden mithilfe eines Callback angenommen und blockieren damit
nicht die Abarbeitung dieser Daten.

\begin{figure}
	\centering
	\includevisio{pubsub}
	\caption{Publish-Subscribe}
	\label{fig:pubsub}
\end{figure}

\subsection{Netzwerklibrary}
Um die Kommunikation zwischen den einzelnen Netzwerkteilnehmern nicht komplett selbst zu implementieren, k{\"{o}}nnen bereits bestehende Netzwerklibraries genutzt werden.
Diese bieten, zum Beispiel, die M{\"{o}}glichkeit an einem entfernten Netzwerkteilnehmer Funktionen aufzurufen oder sich f{\"{u}}r Multicasts anzumelden, auch wenn man in einem anderen
\textit{subnet} ist.

\subsubsection{Anforderungen}
Um verteilte Systeme zu programmieren, bietet sich das RPC Modell an, mit dem auf einem anderen Rechner Funktionen 
aufgerufen werden k{\"{o}}nnen. Alle Voter/Roboter m{\"{u}}ssen die gleichen Informationen {\"{u}}ber den Weltstatus empfangen, um
dann gemeinsam sinnvolle Bewegungen auszuf{\"{u}}hren, diese m{\"{u}}ssen also von der Simulation an alle gleichzeitig verteilt werden
muss. Damit bietet sich auch das Publish Subscribe Interaktionsmuster an.

Diese beiden Interaktionsmuster k{\"{o}}nnen kombiniert werden; Es gibt allerdings keine Netzwerklibrary, die gleichzeitig beides anbietet.
Da es einfacher ist, auf Publish-Subscribe ein RPC Modell aufzubauen, als andersherum, wird diese Funktionalit{\"{a}}t bei der Auswahl der
Netzwerkbibliothek vorgezogen. Dies ist das Hauptkriterium.

Netzwerkbibliotheken, die in C++ geschrieben wurden, k{\"{o}}nnen zwar (mit einem Interface) auch unter C benutzt 
werden, f{\"{u}}gen dem Programm dann aber "heimlich" die C++ Runtime hinzu. Da die Netzwerkbibliothek nicht
nur im Simulationsprogramm selbst, sondern auch im Studenteninterface eingesetzt werden soll, m{\"{u}}sste dann
um alle Eventualit{\"{a}}ten wie zum Beispiel \textit{exceptions} abzuhandeln die Studierenden auch C++
Kentnisse haben. Da dies nicht vorausgesetzt werden kann, ist eine rein in C geschriebene Bibliothek vorzuziehen.

Durch die Benutzung von Unity\ref{unity} ist allerdings vorgegeben, dass das Simulationsprogramm selbst weder in C noch
in C\# geschrieben wird, zur Auswahl stehen zum Beispiel C\# und JavaScript. Daher ist es auch wichtig, ob die Netzwerkbibliothek
auch mit anderen Sprachen genutzt werden kann, oder es Alternativimplementierungen gibt.

\begin{table}[h]
\centering
\begin{tabu}{c | c | c | c | p{5cm}}
	\toprule
	Name & RPC & Publish-Subscribe & C++ Runtimeabh{\"{a}}ngigkeit & andere Sprachen \\
	\midrule
	SunRPC & \checkmark & \xmark & \checkmark & C\#, ... \\
	grpc\cite{grpc} & \checkmark & \xmark & \xmark & C\#, ... \\
	ZeroMQ\cite{zeromq} & \xmark & \checkmark & \xmark & C\#, ... \enquote{40+ languages} \\
	nanomsg\cite{nanomsg} & \xmark & \checkmark & \checkmark & C\#, ... (25 Sprachen) \\
	\bottomrule
\end{tabu}
\caption{Anforderungen an die Netzwerkbibliothek}
\end{table}

Aus dieser vergleichenden Darstellung wird klar, dass nanomsg am besten geeignet ist; auf dessen speziellen Eigenschaften wird nun n{\"{a}}her eingegangen.

\subsubsection{nanomsg}
nanomsg ist eine komplett in C geschrieben Netzwerkbibliothek, mit einer API die der POSIX Socket API gleicht (es wird sich also, zum Beispiel mit nn\_connect() verbunden oder mit nn\_send() gesendet).
Der gro{\ss}e Vorteil von nanomsg sind die implementierten {\"{U}}bertragungsarten. Je nach gew{\"{u}}nschter {\"{U}}bertragungsart wird ein unterschiedlicher Socket erstellt, dieser erlaubt die
Interaktion mit den anderen Netzwerkteilnehmern die auch diese {\"{U}}bertragungsart benutzen.

Beispielhaft wird nun auf einige der genutzten Arten eingegangen.

\paragraph{{\"{U}}bertragungsarten}
\subparagraph{Request-Reply} Die intuitiv verst{\"{a}}ndlichste Art zu kommunizieren ist es eine Frage zu stellen, die dann vom Gegen{\"{u}}ber beantwortet wird. In nanomsg ist dies die request-reply
{\"{U}}bertragungsart.
