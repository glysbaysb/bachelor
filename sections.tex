\documentclass[ngerman]{scrartcl}

\usepackage[utf8]{inputenc}
\usepackage[ngerman]{babel}
%%\usepackage[style=alphabetic, backend=biber]{biblatex}
%\addbibresource{../lit.bib}
%\usepackage[nottoc,numbib]{tocbibind}
%\usepackage[toc]{glossaries}
\usepackage[pdfpagelabels=true,linktoc=all,pdfusetitle]{hyperref}

\begin{document}

\tableofcontents

\clearpage



\part{Aufgabestellung}
\section{Einf{\"{u}}hrung}
\subsection{Motivation}
\subsection{Aufgabenstellung} 
\section{Die ausfallsichere Heizung}\label{heizung}
\section{Diskussion}
\section{Anforderungen}\label{anforderung}
\section{Die balancierenden Roboter}
\section{Anforderungen -> Simulationsidee}
\section{Simulationsidee}

\part{Grundlagen}
\section{Bewegungen}
\subsection{Khepera}\label{khepera}
\subsection{Differential Steering}\label{diffs}
\subsection{Wegfindung}
\section{Die Gleichgewichte}
\subsection{Statisches Gleichgewicht}
%\subsection{Kippwinkel} das muss eher in die Implementierung
\subsection{Dynamischen Gleichgewicht}
\section{Interaktionsmuster}
\subsection{Message Passing}
\subsection{Request-Reply}
\subsection{Publish-Subscribe}


\part{Architektur}
% Ein grosses Problem von mir ist das ich nicht verstehe wie ich zwischen Grundlagen
% und Architektur eine Trennung hinkriegen soll, weil sich das mMn gegenseitig bedingt.
%
% Siehe die Grundlagen oben. Also wuerde ich dann die Grundlagen die defintiiv durch
% die Archtitektur beeinflusst werden, hinter den Entwurf stellen. Also diesen \part
% gedanklich in zwei Teilen.
% Hier ist erst noch Entwurf
\section{Systembestandteile}
\subsection{Kommunikation zwischen den Systemteilen} % eigentlich kann an dieser Stelle erst
				% gesagt werden welche Kommuniktion*anforderungen* es gibt richtig?
				% und nicht das z.B. hier pub-sub eingesetzt werden soll.
				% hier das Paintbild "Aufteilung der Netzwerkteilnehme"?
\subsection{Simulationsprogramm}\label{graphics} % z.B. hier das UML? Obwohl da ja schon Unitybestandteile drin sind
\subsection{Voter / Interface}
\subsection{Controller}
\section{Fehlermodell der Simulation}\label{fm}
\subsection{Roboter / Motoren}
\subsection{Weltstatusinformationen}
\subsection{Netwerk}
\subsection{Controller}
\subsection{Auswirkungen}
% Hier fehlt die Fehlerinjektion oder? Die muss j auch entworfen werden
% --------------
% Hier ist Auswahl der "Komponente" basierend auf Anforderungen -> Alternativen -> Auswahl
% -> beschreibung
\section{Game Engine}
\subsection{Anforderungen}
\subsection{Unity}
\subsubsection{Grafikengine}
\subsubsection{Physikengine}
\subsubsection{Scripting bei Unity}
\section{Netzwerklibrary}
\subsection{Anforderungen}
\subsection{nanomsg}
\section{Serialisierung}
\subsection{Anforderungen}
\subsection{MessagePack}


\part{Implementierung}
\section{Fehlerinjektion}
\subsection{Netzwerk}
\subsection{Controller}
\subsection{Simulation}
\subsubsection{Weltstatus}
\subsubsection{Motor}
\subsubsection{Fernsteuerung}
\section{Die Simulation} % oder hier als Einleitung die Stiche von systembestandteile->Simulationsprogramm?
% oder hier UML hin
\subsection{Die Roboter}\label{robot}
\subsection{Die Ladestation}\label{fuelstation}
\subsection{Die Platte}\label{plate}
\subsection{Netzwerkschnittstelle}
\section{Interface f{\"{u}}r die Studenten}\label{interface}% ist das entwurf oder umsetzung?
\section{Beispielimplementation}
\subsection{Fehlermodell} \label{error-model}
\subsubsection{R{\"{a}}umliche Redudanz}
\subsubsection{Netzwerkkommunikation}
\subsection{Voter}
\subsection{Controller}\label{controller}
\section{Zusammenfassung} % was fuer Messungen / Versuche koennte es denn geben? oder was fuer
						% erweiterungen?

\end{document}
