\documentclass[
    12pt,
    bibliography=totoc,
    ngerman,
    enabledeprecatedfontcommands,
	a4paper
]{scrartcl}

\usepackage[utf8]{inputenc}
\usepackage[T1]{fontenc}
\usepackage{lmodern}
\usepackage[ngerman]{babel}
%\setdefaultlanguage{german}
\usepackage{translator}
%\usepackage[decimalsymbol=comma]{siunitx}

\usepackage[style=alphabetic, backend=biber]{biblatex}
\nocite{*} % alle unbenutzen Literatureintraege trotzdem anzeigen
\addbibresource{../lit.bib}
\usepackage{setspace}
\PassOptionsToPackage{hyphens}{url}
\usepackage[ampersand]{easylist}
\usepackage{graphicx}
\usepackage{longtable}
\usepackage{tabu}
\usepackage{booktabs}
\usepackage{mathtools}
\usepackage{csquotes}
\usepackage{lscape}
\usepackage{textcomp}
\usepackage{listings}
\usepackage{fancyhdr}
\usepackage{rotating}
\usepackage{subcaption}
\usepackage{amssymb} % for \
\usepackage{alltt}
\usepackage[nottoc,numbib]{tocbibind}
\usepackage{todonotes}
\usepackage{xcolor}
\usepackage{gensymb}
\usepackage[linktoc=all,pdfusetitle]{hyperref}
\usepackage{bookmark}
\usepackage[toc]{glossaries}

\setstretch{1.433} %entspricht 1,5 in Word 
\graphicspath{{../images/}}
\MakeOuterQuote{"}
\pagestyle{fancy}
\lstset{frame=single,breaklines=true}
\setcounter{biburllcpenalty}{7000}
\setcounter{biburlucpenalty}{8000}

\newcommand{\xmark}{\text{\sffamily X}} % gegensymbol zu \checkmark

\newcommand{\includevisio}[2][]{\includegraphics[#1]{#2}}
% Source code
\colorlet{punct}{red!60!black}
\definecolor{background}{HTML}{EEEEEE}
\definecolor{delim}{RGB}{20,105,176}
\definecolor{darkgreen}{RGB}{34,177,76}
\colorlet{numb}{magenta!60!black}

\lstset{
    basicstyle=\normalfont\ttfamily,
    numbers=left,
    numberstyle=\scriptsize,
    stepnumber=1,
    numbersep=8pt,
    showstringspaces=false,
    breaklines=true,
    frame=lines,
    backgroundcolor=\color{background},
	commentstyle=\color{darkgreen},
	stringstyle=\color{red},
	keywordstyle=\color{blue}
}

% JSON
\lstdefinelanguage{json}{
    literate=
     *{0}{{{\color{numb}0}}}{1}
      {1}{{{\color{numb}1}}}{1}
      {2}{{{\color{numb}2}}}{1}
      {3}{{{\color{numb}3}}}{1}
      {4}{{{\color{numb}4}}}{1}
      {5}{{{\color{numb}5}}}{1}
      {6}{{{\color{numb}6}}}{1}
      {7}{{{\color{numb}7}}}{1}
      {8}{{{\color{numb}8}}}{1}
      {9}{{{\color{numb}9}}}{1}
      {:}{{{\color{punct}{:}}}}{1}
      {,}{{{\color{punct}{,}}}}{1}
      {\{}{{{\color{delim}{\{}}}}{1}
      {\}}{{{\color{delim}{\}}}}}{1}
      {[}{{{\color{delim}{[}}}}{1}
      {]}{{{\color{delim}{]}}}}{1},
}

% Title page
\titlehead{{\large Fachhochschule S{\"{u}}dwestfalen} \\ Fachbereich f{\"{u}}r Elektrotechnik und Informationstechnik \\ Studiengang Technische Informatik}
\subject{\vspace{2cm} Bachelorarbeit}
\title{Simulationswerkzeug f{\"{u}}r das Praktikum Ausfallsichere Systeme}
\author{Geert Martin Ijewski}
\publishers{\vfill Betreuer: Prof. Dr.-Ing. habil. Jan Richling}

\loadglsentries{../glossary}
\makeglossaries

\begin{document}

\maketitle
\todo{Formatierung durchgehen. Bilder, Tabellen, Codestuecke an der richtigen Stelle. Gleichungen sehen gut aus. Alle Verweise (aufs Glossar oder Sektionen) (nach ?? suchen)}
\thispagestyle{empty}

\clearpage
\pagenumbering{roman}
\tableofcontents
\listoffigures
\listoftables

\clearpage
\printglossary[title=Glossar, toctitle=Glossar]

\clearpage
\pagenumbering{arabic}
\part{Problemstellung}
\clearpage
\section{Einf{\"{u}}hrung}
\begin{figure}
	\centering
	\includevisio[width=\textwidth]{simulation}
	\caption{Screenshot der Simulation}
	\label{fig:simulation}
\end{figure}

Diese Bachelorarbeit besteht aus drei Teilen. Im ersten wird das zu l{\"{o}}sende Problem-- die Entwicklung eines
Simulationswerkzeuges, das als {\"{U}}bungsaufgabe fungiert, zum Modul "Ausfallsichere Systeme" -- entfaltet,
im zweiten auf die dazu ben{\"{o}}tigten Grundlagen eingegangen und als letztes wird der L{\"{o}}sungsweg 
skizziert. 

Der erste Teil besteht aus einer Darlegung des Problems, um den Sinn und Zweck dieser Arbeit zu verdeutlichen. 
Danach wird erkl{\"{a}}rt, warum die bislang an der FH S{\"{u}}dwestfalen existierende {\"{U}}bungsaufgabe nicht optimal ist.
Basierend auf dieser Analyse werden die Anforderungen an die neue Aufgabe entwickelt. 

Im zweiten Teil werden als erstes die ben{\"{o}}tigten Grundlagen zur Umsetzung erkl{\"{a}}rt. Dabei geht es einerseits um 
mechanische und physische Konzepte wie das \textit{differential steering} und auch um informationstechnologische Aspekte
wie die Serialisierung von Daten oder verschiedene Netzwerkkommunikationsarten.

Darauf aufbauend wird dann im dritten Teil die Implementierung beschrieben, mit besonderem Fokus auf dem 
Fehlermodell und der Fehlerinjektion. Zum Schluss wird die Simulation anhand der Kriterien evaluiert und ein Fazit 
gezogen.

\subsection{Motivation}
Seit der Einf{\"{u}}hrung der Computer und vor allem seit dem rasanten Wachstums des Internets, werden Computer in immer 
mehr Bereichen eingesetzt. Das betrifft nicht nur die eigentlichen "Personal-Computer" auf dem Schreibtisch, 
sondern mehr und mehr embedded systems, die f{\"{u}}r viele Laien gar nicht als "Computer" wahrgenommen werden. Diese 
als "Digitalisierung" beschriebene Tendenz der Gegenwart hat neben der Berufswelt l{\"{a}}ngst auch weite Bereiche 
der Freizeitaktivit{\"{a}}ten erfasst -- in Bowlingbahnen beispielsweise z{\"{a}}hlt ein Computer f{\"{u}}r jeden 
Spieler mit, wie viele Punkte erreicht wurden und stellt die umgeworfenen Pins automatisch wieder auf. All dies 
bietet enorme, jeden Tag ersichtlich werdende Vorteile, birgt jedoch auch Risiken. 

Denn durch die steigende Komplexit{\"{a}}t der Hard- und Software wird das System fehleranf{\"{a}}lliger. Es gibt 
mehr Komponenten, also kann mehr kaputt gehen. Letztlich ist es nicht m{\"{o}}glich, in hochkomplexen Systemen ein 
absolut perfektes, v{\"{o}}llig fehlerfreies System zu entwickeln. 
Was bei einem Bowlingabend f{\"{u}}r die Mitspieler nicht besonders kritisch ist, wird zu einem Problem,in unmittelbar 
lebenswichtigen Bereichen wie der Strom- und Wasserversorgung, Fahrassistenz und {\"{a}}hnlichem. Auch kleine 
Fehler dort k{\"{o}}nnen enorme Konsequenzen haben. Ein tragisches Beispiel ist der Therac-25, eine Maschine zur 
Strahlentherapie. Durch verschiedene Softwarefehler kam es unter bestimmten Umst{\"{a}}nden zu {\"{u}}berh{\"{o}}hten 
Strahlendosen, die teilweise t{\"{o}}dlich endeten.

Aus dem bislang Festgestellten ergibt sich die zwingende Notwendigkeit, Hard- und Software gr{\"{u}}ndlich zu 
testen. Es gibt verschiedene Arten von Tests und Teststrategien, die das System aus unterschiedlichen Blickwinkeln 
betrachten. Hierbei kann, zum Beispiel, zwischen static und dynamic testing unterschieden werden. 

Bei statischem Testen wird der Quellcode untersucht, das Programm oder  Programmbestandteile aber nicht 
ausgef{\"{u}}hrt. Ein Beispiel daf{\"{u}}r ist static analysis, bei der der Programmcode auf auf{\"{a}}llige
Codekonstrukte untersucht wird, die h{\"{a}}ufig zu Fehlern f{\"{u}}hren -- wie zum Beispiel die Benutzung nicht 
initialisierter Variablen. 

Ein anderer Ansatz sind die code reviews, bei denen der programmierte Code mit weiteren Programmieren besprochen 
wird, um nach dem Vier-Augen-Prinzip weitere Probleme zu identifizieren.

Ein Beispiel f{\"{u}}r das dynamische Testen sind Unit-Tests, bei denen eine logische Einheit des Systems isoliert 
getestet wird, in dem sie mit vorher festgelegten Parametern ausgef{\"{u}}hrt wird und das Ergebnis mit einem 
erwarteten Ergebnis verglichen wird. Zum Beispiel k{\"{o}}nnen bei einem Navigationsger{\"{a}}t die Wegfindung, die 
Anzeige und das Interpretieren der Kartendaten als eigene Module programmiert und getestet werden. Um die 
Wegfindung zu testen, w{\"{u}}rde dann eine k{\"{u}}nstliche Karte genutzt und getestet, ob der 
Wegfindungsalgorithmus f{\"{u}}r vorher definierte Strecken auf dieser Karte die k{\"{u}}rzesten Wege findet. 

Es gibt allerdings auch F{\"{a}}lle, in denen es nicht machbar ist das System komplett durchzutesten. 
Beispielsweise ist es zwar mit spezieller Hardware m{\"{o}}glich, den Befehlssatz einer CPU zu emulieren und zu 
{\"{u}}berpr{\"{u}}fen, ob bestimmte Befehlskombinationen zu Problemen f{\"{u}}hren. Allerdings ist die 
Geschwindigkeit der Tests um mehrere Gr{\"{o}}ssenordnungen kleiner als die der tats{\"{a}}chlichen CPU, so das 
eine einzige x86 CPU in der ersten Sekunde mehr Instruktionen ausf{\"{u}}hrt als w{\"{a}}hrend des gesamten 
Testzyklus \cite{kaplan}. Es ist also unm{\"{o}}glich die CPU komplett durchzutesten, was dazu f{\"{u}}hrt das CPUs 
trotzdem noch Fehler enthalten, ein ber{\"{u}}hmtes Beispiel ist der Pentium F00F Bug.

Da also Testen alleine nicht ausreicht ist es besonders wichtig bei der Designphase angefangen die Ausfallsicherheit immer einzuplanen. Daf{\"{u}}r
muss ein Problembewusstsein geschaffen werden. Um dann wirklich ausfallsichere System zu entwickeln braucht man nat{\"{u}}rlich auch bestimmte Kentnisse 
{\"{u}}ber das Design und die Entwicklung solcher.

\subsection{Aufgabenstellung} 
Diese Kenntnisse zu vermitteln ist die Aufgabe des Moduls "Ausfallsichere Systeme" im Rahmen des Studienganges 
"Technische Informatik". Daf{\"{u}}r gibt es einen theoretischen (die Vorlesung) und einen praktischen Teil (eine 
{\"{U}}bungsaufgabe). Die {\"{U}}bungsaufgabe soll eine M{\"{o}}glichkeit geben, das in der Vorlesung erworbene 
theoretische Wissen anzuwenden. Dabei werden die Studierenden aus den gelernten Techniken die richtigen 
f{\"{u}}r die Aufgabe heraussuchen m{\"{u}}ssen, diese gegen Alternativen abw{\"{a}}gen und sie dann zu 
implementieren. Durch die  praktische Anwendung wird das Gelernte wiederholt und vertieft, was zu einem besseren 
Verst{\"{a}}ndnis f{\"{u}}hrt. 

Die Aufgabe dieser Bachelorarbeit ist die Konzeption einer solchen praktischen {\"{U}}bungsaufgabe. 
Zus{\"{a}}tzlich soll das fertige Produkt auch im Rahmen von Informationsveranstaltungen als Demonstrationsprojekt 
nutzbar sein. Damit soll auch Menschen ohne Fachkenntnisse ein interessanter Einblick in die 
Komplexit{\"{a}}t ausfallsicherer Systeme gegeben werden.

\subsection{Die ausfallsichere Heizung}\label{heizung}
Eine {\"{U}}bungsaufgabe im Module Ausfallsichere System wurde im Sommersemester 2016 an der FH S{\"{u}}dwestfalen entwickelt. Das Ziel war es, mithilfe zwei von Heizk{\"{o}}rpern, zweier
L{\"{u}}fter, einer Klappe und mehrer Temperatursensoren eine Temperatur zu regeln. F{\"{u}}r die Regelung m{\"{u}}ssen sich drei Rechner abstimmen, denn einer kontrolliert
beide Heizungen, einer beide L{\"{u}}fter und der dritte die Klappe. Um sich abzustimmen k{\"{o}}nnen die Rechner {\"{u}}ber UDP und I\textsuperscript{2}C miteinander kommunizieren.

Das Gesamtsystem ist Y-f{\"{o}}rmig aufgebaut; an den Armen des Ypsilon befindet sich jeweils ein Heizk{\"{o}}rper und ein L{\"{u}}fter; an dem Punkt, an dem sich
die Arme treffen, ist die Klappe, die den Luftstrom eines Arms ganz oder teilweise blockieren kann (diese Luft wird dann nach oben abgeleitet).

Um die Temperaturen in den Armen und die Temperatur des Luftstromes zu messen gibt es Temperatursensoren. Jeweils ein Temperatursensor im Luftausgangsstrom ist einem
der Rechner zugeordnet. Zus{\"{a}}tzlich hat der L{\"{u}}ftungsrechner einen Temperatursensor pro Arm zwischen L{\"{u}}fter und Heizung, und der Heizungsrechner jeweils
einen Temperatursensor vor der Heizung.

\begin{figure}
	\centering
	\includevisio[width=\textwidth]{HeizungHW}
	\caption{Aktoren und Sensoren der Heizung}
	\label{fig:heizunghw}
\end{figure}

All diese Bestandteile k{\"{o}}nnen auf verschiedene Arten fehlerhaft sein. Die Aufgabe der Studierenden
ist es, sich f{\"{u}}r die verschiedenen Fehlerkombinationen
geeignete Strategien auszudenken, um die Auswirkungen zu minimieren.

Beispielsweise kann es vorkommen das ein Heizk{\"{o}}rper ausf{\"{a}}llt, beziehungsweise dass dem
Heizk{\"{o}}rper ein Fehler injiziert wird\footnote{In diesem Fall wird die Stromversorgung gekappt},
also sich nicht weiter aufheizen kann. Dies muss erst einmal erkannt werden, zum Beispiel
durch den Vergleich des Sollwertes der Heizungsansteuerung und den Lufttemperaturen vor und hinter der Heizk{\"{o}}rper. Falls diese Temperaturen gleich sind,
wurde die Luft die vom L{\"{u}}fter Richtung Ausgang bewegt wurde, nicht aufgew{\"{a}}rmt, ergo heizt dieser Heizk{\"{o}}rper nicht. Dann k{\"{o}}nnen sich die
Rechner abstimmen, wie sie die Temperatur trotzdem ansteigen lassen k{\"{o}}nnen, zum Beispiel ob der {\"{u}}brig gebliene Heizk{\"{o}}rper st{\"{a}}rker heizen
soll, oder ob dieser weniger gek{\"{u}}hlt werden soll.

{\"{A}}hnlich ist es bei einem Ausfall der L{\"{u}}fter. 

Um die Kommunikation zu testen, k{\"{o}}nnen auch die Kommunikationswege ausfallen\footnote{Hierbei
wird das LAN-Kabel aus dem Switch gezogen, wodurch dieser Rechner keine Verbindung mehr zu den anderen hat.
Das gleiche gilt f{\"{u}}r die I\textsuperscript{2}C Leitungen}.
Was passiert, wenn der Heizungsrechner nicht mehr erreichbar ist, aber eine Heizung weniger stark heizen soll?
Da es zwei Kommunikationswege gibt, kann normalerweise UDP/IP f{\"{u}}r die Kommunikation verwendt werden
und I\textsuperscript{2}C nur als \textit{hot standby}. Alternativ wird ein {\"{U}}bertragungsweg nur
f{\"{u}}r Daten genutzt und der andere um zu signalisieren, dass dieser Rechner noch das
Steuerprogramm ausf{\"{u}}hrt. Der Ausfall einer der beiden Leitung kann anhand der Aktivit{\"{a}}t auf der
anderen erkannt werden.

Als letztes k{\"{o}}nnen auch die Temperatursensoren ausfallen. Im Luftausgangsstrom, wo drei Sensoren 
nebeneinander angeordnet sind, ist es m{\"{o}}glich den
Ausfall eines Sensors anhand der Diskrepanz der Messergebnisse aller Sensoren zu erkennen. Eine m{\"{o}}gliche 
Strategie, um die Ausgangstemperatur zu
bestimmen, w{\"{a}}re also, die \textit{triple modular redudancy} auszunutzen und den Mittelwert aller nicht klar als 
fehlerhaft erkennbaren Sensoren zu bilden.
Es ist auch denkbar, die vorherigen Messwerte zu benutzen um anhand unrealistischer Spr{\"{u}}nge der Messwerte zu 
erkennen, dass ein Sensor einen fehlerhaften Messwert geliefert hat.

Nat{\"{u}}rlich ist es auch denkbar, dass mehrere dieser Fehler auftreten, also beispielsweise f{\"{a}}llt
erst ein Sensor, dann das Netzwerk aus und danach steckt die Klappe fest.
Gerade bei Doppelfehlern, in denen eine L{\"{u}}ftung und ein Heizk{\"{o}}rper in unterschiedlichen {\"{A}}rmen 
ausfallen, gibt es die Gefahr, dass sich ein Arm {\"{u}}berhitzt und besch{\"{a}}digt wird.

\subsection{Diskussion}
Diese bestehende {\"{U}}bungsaufgabe hat allerdings einige Probleme, welche bei Konzeption und Bau einer neuen
{\"{U}}bungsaufgabe vermieden werden sollen.
\todo{was stoert was kann besser sein}

Weil die Heizungen eine beschr{\"{a}}nkte Leistung haben und eine gro{\ss}e Menge Luft 
($ \approx (13*13*55)cm^3$) aufheizen m{\"{u}}ssen, ist die Regelung sehr tr{\"{a}}ge. Damit ist es als
Vorf{\"{u}}hrobjekt zum Beispiel f{\"{u}}r Schulklassen ungeeignet, diese k{\"{o}}nnen nicht 10 Minuten warten
bis die Temperatur eingeregelt wurde, dann ein Fehler injiziert wurde (zum Beispiel eine Heizung
ausf{\"{a}}llt), dieser Ausfall erkannt wird und dann gegengesteuert wurde. Bei einer neuen {\"{U}}bungsaufgabe
soll das System schneller reagieren, am besten sogar instantan.

Desweiteren gibt es dieses System nur einmal. Das bedeutet einerseits, dass es bei gro{\ss}en Gruppen
an Sch{\"{u}}lern unm{\"{o}}glich wird, dass alle etwas f{\"{u}}hlen k{\"{o}}nnen und andererseits, es durch
die lange Regelzeit zu Koordinationsproblemen kommt. W{\"{a}}hrend eine Gruppe ihr Programm testet, m{\"{u}}ssen
alle anderen Gruppen warten. Da fast jeder Test mehrere Minuten dauert, verbringen die Gruppen die
allermeiste Zeit mit warten und nur wenig Zeit mit produktiver Arbeit. Um dies zu vermeiden w{\"{a}}hre es
hilfreich wenn die Ersatz{\"{u}}bungsaufgabe gleichzeitig von mehreren Teams unabh{\"{a}}nig voneinander bearbeitet
werden k{\"{o}}nte.

Das letzes Problem ist, dass die Heizung bei unsachgem{\"{a}}ssem Gebrauch -- also wenn sie sich zu sehr aufheizt --
besch{\"{a}}digt, da sich die Heizwiederst{\"{a}}nde selbst ausl{\"{o}}ten. Danach muss sie erst aufw{\"{a}}ndig
repariert werden und ist in der Zwischenzeit nicht benutzbar. Daher sollte die Alternativaufgabe keine besch{\"{a}}digbaren
Teile beeinhalten.

Auch w{\"{a}}re es sch{\"{o}}n wenn es noch subtilere M{\"{o}}glichkeiten der Fehlerinjektion g{\"{a}}be. Zum Beispiel ist
es nur m{\"{o}}glich die L{\"{u}}fter komplett abzuschalten, aber keine \textit{stuck-at} Fehler zu injezieren.
Gerade bei der Netzwerkkommunikation g{\"{a}}be es viele M{\"{o}}glichkeiten um Zeitfehler und Wertefehler zu injizieren.

Eine bereits existierende Alternative w{\"{a}}re die Wippe. Sie besteht aus einem Computergesteuerten Motor der eine
Wippe kippen kann auf der ein Ball liegt. Die Aufgabe ist es nun den Ball m{\"{o}}glichst in der Mitte zu balancieren.
Dieses System ist intuitiv verst{\"{a}}ndlich und reagiert --- im Gegensatz zur Heizung --- sofort. Allerdings ist
es ungeeignet um Konzepte der Ausfallsicherheit zu verdeutlichen: es gibt nur eine Komponente die ausfallen kann (den
Motor).

Momentan wird im Rahmen einer weiteren Bachelorarbeit ein {\"{a}}hnliches Prinzip im zweidimensionalen Raum entwickelt;
die Kugel soll nun auf einer Achse "Vorne-Hinten" und auf der Achse "Links-Rechts" m{\"{o}}glichst im Mittelpunkt liegen.
Auch die dadurch, wenigstens ansatzweise, vorhandene Redudandanz {\"{a}}ndert aber nichts am gr{\"{o}}{\ss}ten Kritikpunkt:
es gibt nur ein System wodurch nur eine Studentengruppe gleichzeitig arbeiten kann.

\clearpage
\section{Anforderungen}\label{anforderung}
Aus den Schwachstellen der vorhanden L{\"{o}}sung l{\"{a}}sst sich ableiten, was das neue System leisten soll:
\begin{enumerate}
	\item Die Alternativaufgabe soll sich stark von der ausfallsicheren Heizung unterscheiden, damit die Studierenden eine tats{\"{a}}chliche Wahl haben.
	\item Es muss m{\"{o}}glich sein mehrere Instanzen gleichzeitig laufen zu lassen. Damit werden die Koordinierungsprobleme vermieden.
	\item Es soll auch m{\"{o}}glich sein au{\ss}erhalb des MR Labores an der Problemstellung weiterzuarbeiten.
	\item In dem neuen System muss die Fehlerinjektion parametrisierbar sein. Damit ist es m{\"{o}}glich,
		die Aufgabe bei Bedarf zu vereinfachen oder zu verkomplizieren, zum Beispiel in dem bestimmte
		Arten der Fehlerinjektion ganz auschgeschaltet werden.
	\item Um die Aufgabe zu l{\"{o}}sen, muss eine Vielzahl von Konzepten der Ausfallsicherheit genutzt werden.
	\item Es soll visuell ansprechend sein.
	\item Es darf auch durch unsachgem{\"{a}}ssen Gebrauch nicht gesch{\"{a}}digt werden
\end{enumerate}

\clearpage
\section{Die balancierenden Roboter}
Viele dieser Anforderungen k{\"{o}}nnen am einfachsten mit einer Simulation abgedeckt werden. Diese kann gleichzeitig
von mehreren Gruppen genutzt werden und stellt keine Einschr{\"{a}}nkungen an die M{\"{o}}glichkeiten der Fehlerinjektion
oder der ben{\"{o}}tigten Konzepte.

An der Humboldt Universit{\"{a}}t Berlin (in Zusammenarbeit mit Daimler Benz) wurde eine solche
Simulation\cite{Werner00} entwickelt. In dieser soll eine simulierte Platte durch die Bewegung von 
simulierten Robotern ausbalanciert werden. Das Gewicht jedes Roboters wirkt auf die Platte, bringt sie also
aus dem Gleichgewicht.

Die Roboter m{\"{u}}ssen sich nun so bewegen, dass jede durch einen Roboter ausge{\"{u}}bte Kraft von einem
anderen Roboter eliminiert wird. Damit es nicht m{\"{o}}glich ist, einmal eine Stellung einzunehmen,
die zum Gleichgewicht f{\"{u}}hrt und dort zu verharren, haben die Roboter einen Energiespeicher, der
mit der Zeit leerer wird. Da die Roboter sich nur bewegen k{\"{o}}nnen solange sie noch Energiereserven haben, 
m{\"{u}}ssen sie hin und wieder diesen Speicher an einer Ladestation auff{\"{u}}llen.

Das Ziel dieser Simulation war zu demonstrieren, dass auch in Echtzeitsystem eine Konsensfindung m{\"{o}}glich
ist. Fall ein Ergebniss durch \textit{triple modular redudancy} und einen Voter bestimmt werden soll, 
verh{\"{a}}lt sich die naive Implementation im Falle eines \textit{timing faults}, so dass die
Einhaltung der Echtzeitanforderungen nicht mehr garantiert werden kann. 
\todo{raussuchen ob das nicht mit den relative mehrheit geloest wurde}

Die Simulation und die Controller wurden in Objective-C geschrieben und laufen auf dem NeXTStep Betriebssystem, 
w{\"{a}}hrend das Anzeigeprogramm Java basiert ist und z.B. auf
einem Windowsrechner laufen kann. Die Teilnehmern kommunizieren {\"{u}}ber CORBA 
miteinander\cite{predictablecorba}. CORBA ist ein Standard, {\"{u}}ber den Programme auf
mehreren Rechnern miteinander kommunizieren k{\"{o}}nnen, auch falls sie in verschiedenen Programmiersprachen 
geschrieben wurden. Zus{\"{a}}tzlich zur reinen RPC Funktionalit{\"{a}}t abstrahiert es, unter anderem, 
Transkationen und Zeitsynchronisation.

In das System k{\"{o}}nnen Fehler injiziert werden, zum Beispiel kann ein Roboter ferngesteuert bewegt werden. Auch
die Controller k{\"{o}}nnen entweder beendet werden oder ihnen kann die Netzwerkverbindung entzogen werden. Es ist
auch m{\"{o}}glich einen Roboter fernzusteuern und damit direkten Einfluss auf die Simulation zu nehmen.

Die balancieren Robotern w{\"{u}}rden den Anforderungen an vielen Stellen entsprechen, zum Beispiel
unterscheidet sie sich stark von der ausfallsicheren Heizung, es k{\"{o}}nnen mehrere Instanzen
gleichzeitig laufen und die Fehlerinjektion ist parametrisierbar. Gerade die dahinterliegende Idee ist
leicht verst{\"{a}}ndlich und gut geeignet; die Implementation weisst allerdings einige Schwachstellen auf.

Ein erster Kritikpunk ist das Anzeige nicht mehr zeitgem{\"{a}}ss ist, diese m{\"{u}}sste also neu entwickelt werden.

Auch ist das physikalische Modell der Simulation sehr rudiment{\"{a}}r: die Roboter k{\"{o}}nnen sich
holonomisch, also ohne Limitationen in alle Richtungen, bewegen. Dies ist aber bei den meisten Robotern
nicht der Fall, diese haben auf Grund ihres Aufbaus bestimmte Limitationen. Zum Beispiel ist es un{\"{u}}blich, 
dass sich Roboter vertikal frei bewegen k{\"{o}}nen. Des weiteren gibt es in der Simulation keine mechanischen 
Gleichgewichte -- auch dies eine unrealistische Vereinfachung gegen{\"{u}}ber der echten Welt. Die
physikalische Simulation m{\"{u}}sste also erweitert werden.

Das gr{\"{o}}{\ss}ere Problem ist allerdings das Alter der Simulation. Sie wurde in den neunziger Jahren auf 
NeXTStep Rechner entwickelt -- diese existieren an der Fachhochschule S{\"{u}}dwestfalen nicht, das Programm 
m{\"{u}}sste also erst portiert werden. Auch wird CORBA heutzutage nur noch wenig genutzt, also ist der Aufwand,
um sich in den Standard einzuarbeiten und sein Programm basierend auf diesem zu entwickeln, nicht vertretbar. 
Dies gilt gerade auch unter dem Gesichtspunkt, dass die Ausfallsicherheit im Vordergrund stehen sollte und
nicht die Einarbeitung in einen Standard.

Aus all diesen Gr{\"{u}}nden wurde entschieden, eine neue {\"{U}}bungsaufgabe zu entwickeln. Bei dieser wird
allerdings die Idee dieser Aufgabe beibehalten; die neue Aufgabe soll nur die identifizierten Schwachstellen
ausmerzen.

\clearpage
\section{Konkretisierung der Anforderungen}
Um die vorhandene Simulationsidee ohne die vorhanden Schwachstellen neu zu erstellen, so das die Anforderungen
(wie in \ref{anforderung} beschrieben erf{\"{u}}llt werden) ist es n{\"{o}}tig diese eher abstrakten Anforderungen in
technische Merkmale zu "{\"{u}}bersetzen".

Da es nicht m{\"{o}}glich sein soll die neue {\"{U}}bungsaufgabe zu besch{\"{a}}digen, muss es eine reine Softwareaufgabe
sein. Diese k{\"{o}}nnen im schlimmsten Fall einfach noch einmal neu gestartet werden um (tempor{\"{a}}re) Fehler zu
beheben.

Bei einer Softwarel{\"{o}}sung ist es auch m{\"{o}}glich die Ausf{\"{u}}hrung mehrerer Instanzen unabh{\"{a}}nig voneinander
zu erlauben, damit mehrere Studierendengruppen gleichzeitig ihre L{\"{o}}sung entwickeln k{\"{o}}nnen. Daher ist es notwendig
das es keinerlei Resourcen gibt die sich alle Instanzen teilen m{\"{u}}ssen --- das Programm sollte also selbstst{\"{a}}nig und
ohne weitere Komponenten laufen. Bei einem verteilten System das {\"{u}}ber ein IP-Netzwerk per Broadcast kommuniziert ist es
auch wichtig eine Trennung der verschiedenen Instanzen zu erm{\"{o}}glichen (dies kann z.B. durch die Verwendung verschiedener
Ports erreicht werden).

Desweiteren erlaubt der Einsatz einer Simulation ein eigenes Fehlermodell zu definieren. Wenn das Fehlermodell nicht fest
implementiert wird, sondern bei Bedarf ge{\"{a}}ndert werden kann soll es auch m{\"{o}}glich sein einzelne Fehlerquellen an/auszuschalten
oder die Wahrscheinlich mit der Fehler auftreten anzupassen. Damit w{\"{a}}re die Anforderung nach der Parametresierbarkeit
der Fehlerinjektion erf{\"{u}}llt.

Durch den Einsatz einer Simulation ist es auch m{\"{o}}glich die Visualisierung frei zu w{\"{a}}hlen; zum Beispiel kann die 2D
Vogelperspektive wie in der Orginalimplementation der balancierenden Roboter durch eine 3D Ansicht abgel{\"{o}}st werden.

Durch die Simulationsidee, den Ablauf der Steuerung der balancierenden Roboter, ist der Einsatz verschiedener Techniken um
fehlertolerante Systeme --- wie zum Beispiel die Konsensbildung mit einem Voter oder verschiedene Arten der Vorw{\"{a}}rtsfehlerkorrektur ---
n{\"{o}}tig. Dies ist gerade im Lehrkontext wichtig, damit m{\"{o}}glichst viele Konzepte sinnvoll angewandt werden
k{\"{o}}nnen und es im besten Fall sogar n{\"{o}}tig ist das die Studenten zwischen verschiedenen Alternativen w{\"{a}}hlen
m{\"{u}}ssen.

\clearpage
\section{Konzeption}
\subsection{Simulationsidee}
Es wurde sich entschieden die Grundidee der Simulation beizubehalten, diese allerdings neu zu implementieren um die Kritikpunkte am
bestehenden System auszumerzen. Im folgenden wird die Grundidee noch einmal konkretisiert:

Die simulierte Welt besteht aus einer kreisf{\"{o}}rmigen Platte. Auf dieser k{\"{o}}nnen bis zu \gls{N}
Roboter und $1$ F{\"{u}}llstation (\gls{F}) platziert werden. Die Summe alle Gewichte kippt die Platte;
sind die Gewichte zu ungleichm{\"{a}}{\ss}ig verteilt, kippt die Platte um und die Simulation endet.

Die Roboter k{\"{o}}nnen sich frei auf der Platte bewegen, verbrauchen dabei aber Energie und m{\"{u}}ssen diese dann irgendwann
an der F{\"{u}}llstation auff{\"{u}}llen. Auch wenn die Roboter sich nicht bewegen, verbrauchen sie Energie (allerdings weniger
schnell).

Gesteuert werden die Roboter von bis zu \gls{M} $(M \geq N)$ Controllern. Einer oder mehr Controller stimmen sich ab,
so dass:
\begin{itemize}
\item die Roboter ihre Energie immer rechtzeitig auff{\"{u}}llen
\item die Roboter nicht kollidieren
\item die Platte nicht kippt
\end{itemize}

Dieser Abstimmungsvorgang sollte fehlertolerant implementiert werden, so dass auch beim Ausfall eines (oder mehrerer) Controller die Roboter sich immer noch koordiniert bewegen.
(Abh{\"{a}}ngig vom Verh{\"{a}}ltnis der Anzahl der Controller zur Anzahl der Roboter k{\"{o}}nnen verschiedene Stufen der Fehlertoleranz erreicht werden.\cite[s.149]{Werner00})
Daraus folgt das die Studierenden in ihrer L{\"{o}}sung eine Konsensbildung implementieren m{\"{u}}ssen.

Die Voter und Controller laufen auf Virtuellen Maschinen, jeweils eine VM pro Voter- oder Controllerinstanz.
Diese VMs befinden sich in einem Netzwerk, das durch eine Fehlerinjektion UDP Pakete verf{\"{a}}lscht --
dies simuliert ein fehlertr{\"{a}}chtiges Netzwerk, wie es zum Beispiel bei Funknetzwerken der Fall ist.
Es wird vorgegeben, dass die gesamte Kommunikation der Studentenprogramme {\"{u}}ber UDP stattfinden muss.
Dies f{\"{u}}hrt dazu das die Studierenden Gegenmassnahmen gegen die Verf{\"{a}}lschung ergreifen m{\"{u}}ssen, denkbar
w{\"{a}}hre hier zum Beispiel eine Kanalkodierung zur Vorw{\"{a}}rtsfehlerkorrektur.
Die Kommunikation von Simulation zu den Votern ist nicht betroffen, sie soll nicht verf{\"{a}}lscht werden. (Statdessen
kann die Schnittstelle gezieltere Verf{\"{a}}lschungen vornehmen.)

\begin{figure}
	\centering
	\includevisio[width=\textwidth]{Netzwerk}
	\caption{Aufteilung der Netzwerkteilnehmer}
	\label{fig:network}
\end{figure}

Zusammenfassend besteht das ganze System aus diesen Teilen:
\begin{itemize}
	\item Die Simulation berechnet die Bewegungen und implementiert das physikalische Modell. Sie {\"{u}}bernimmt auch die Visualisierung. Per Definition ist sie ausfallsicher.
	\item Der Voter sammelt die Steuerkommandos von den Controllern, bildet daraus das Steuerergebniss und sendet dies an die Simulation. Er ist per Definition ausfallsicher gegen \textit{crash failures}.
	\item Die Controller generieren Steuerkommandos. Sie k{\"{o}}nnen jederzeit ausfallen.
\end{itemize}

\paragraph{Nutzung durch die Studenten} Von den Studierenden muss nicht die gesamte Simulation erstellt werden;
die {\"{U}}bungsaufgabe bezieht sich nur auf die Abstimmung zwischen den Controllern und die Ansteuerung der Roboter.
Daher ist es notwendig, ein Interface ins "Innere" der Simulation bereitzustellen, das dann von den Studenten genutzt wird.

\subsection{Architekturkonzept}
Nach dem ausarbeiten der Anforderungen an das Simulationswerkzeug und die Entscheidung {\"{u}}ber den Simulationsablauf (inklusive
der Trennung zwischen f{\"{u}}r die Studierenden vorgefertigten Teil und den von diesen selbst zu implementierenden) ist es n{\"{o}}tig
die Systembestandteile aufzulisten. Dabei werden auch die Beziehungen zwischen ihnen aufgef{\"{u}}hrt, da diese Einfluss auf die
Architektur haben.

\begin{figure}
	\centering
	\includevisio[width=\textwidth]{Komponenten1}
	\caption{Ben{\"{o}}tigte Systembestandteile und ihre Beziehungen}
	\label{fig:arch_pre}
\end{figure}

Der Hauptbestandteil der Simulation ist die Physiksimulation. Sie ist daf{\"{u}}r zust{\"{a}}ndig
das sich die Roboter bewegen, betankt werden k{\"{o}}nnen, muss Informationen {\"{u}}ber den
Status der Simulation (z.B. Roboterpositionen) bereitstellen und auch die Roboter basierend
auf den Steuerbefehlen der Voter bewegen (unter Beachtung aller physikalischen Eigenschaften
des Roboters und der Fehlerinjektion).

Um die Roboter zu steuern muss a) der Status der Roboter bekannt sein b) es f{\"{u}}r den
Voter m{\"{o}}glich sein die Steuerbefehle an die Simulation weiter zu reichen. Beides
erm{\"{o}}glicht das Interface.

F{\"{u}}r die Fehlerinjektion ist der Fehlerinjektionssmanager zus{\"{a}}ndig. Die Physiksimulation
benutzt diesen um Roboterspezifische Fehler\ref{fm-robot} zu injezieren, w{\"{a}}hrend das
Interface basierend auf dem Fehlerinjektionsmanager die Informationen {\"{u}}ber den
Status der Welt verf{\"{a}}lscht.

Um den Simulationsstatus verfolgen zu k{\"{o}}nnen ist eine visuelle Anzeige hilfreich. Daher
ist ein Bestandteil der Simulation ein Anzeigeprogramm.

Jede Voterinstanz ist daf{\"{u}}r zust{\"{a}}ndig ihren Roboter in der Simulation zu steuern.
Daf{\"{u}}r m{\"{u}}ssen die Steuerkommandos an die Physiksimulation weitergeleitet werden,
dies {\"{u}}bernimmt das Interface. Der Controller ist komplett den Studenten {\"{u}}berlassen-
Es gibt nur die Einschr{\"{a}}nkung das alle Kommunikation zur Simulation hin, {\"{u}}ber den Voter erfolgen muss.

\subsection{Kommunikation zwischen den Systemteilen}
Bei verteilten Systemen (wie diesem) ist einer der wichtigsten Punkte wie die Kommunikation zwischen den Systemteilen abl{\"{a}}uft.
Hierbei kann zwischen zwei Gruppierungen unterschieden werden: einmal die Kommunikation zwischen Simulation und Voter und andererseits
die Kommunikationen zwischen den Controllern und zwischen Controller und Voter. Da die letztere von den Studierenden geplant werden soll,
wird sie hier nicht weiter untersucht.

Da alle Voter in der gleichen Welt miteinander interagieren sollen, ist es am einfachsten wenn die Simulation ein Server ist, zu dem sich alle
verbinden. Dieser Server kann dann intern die Simulation vorantreiben und nur die Ergebnisse dieser weiterversenden. Die Alternative
w{\"{a}}re ein System ohne zentrale Instanz in dem die Simulation auf jedem Robter ausgef{\"{u}}hrt wird. In diesem Fall gibt es
allerdings das Problem das sich der Status der Simulationen voneinander desychnorisieren kann, zum Beispiel wenn ein Voter
eine schw{\"{a}}cheren CPU hat, die l{\"{a}}nger zur Berechnung eines Simulationstaktes braucht als die anderen Voter. Daher ist
das Modell mit zentrale Instanz vorzuziehen; es stellt automatisch sicher das alle Voter (im Normalfall, also ohne Fehler)
die gleichen Informationen kriegen und anhand dieser dann reagieren.

Durch die Verwendung einer zentralen Instanz ist die eine H{\"{a}}lfte der Isolation zwischen verschiedenen Instanzen realisiert:
Die Voter jeder Instanz verbinden sich mit ihrer Instanz, diese sendet ihre Informationen nur an die verbundenen Voter.
\footnote{Der andere Teil --- die Isolation zwischen verschiedenen Instanzen in Bezug auf Controller und Voter --- kann entweder
durch die Benutzung verschiedener Ports oder den Gebrauch mehrere \textit{subnets} erreicht werden.}

Ein Nachteil dieser L{\"{o}}sung ist das sie einen \textit{single point of failure} erschafft. Fehler in dem Simulationsprogramm
wirken sich direkt auf das ganze System aus; daher wird definiert das dieses keine Fehler enth{\"{a}}hlt.

F{\"{u}}r den zeitlichen Ablauf in der Simulation sind zwei M{\"{o}}glichkeiten denkbar: eine rundenbasierte Abfolge oder eine
Abfolge in Echtzeit. Bei der rundenbasieren Simulationstrategie wird der Status der Simulation immer nach festgelegten Intervallen
ver{\"{a}}ndert. In diesem Fall w{\"{a}}re es denkbar das die Controller und Voter 250ms Zeit haben ihre Bewegungsvorschl{\"{a}}ge
zu berechnen und diese an die Simulation schicken. Diese sammelt sie, wendet sie alle auf einmal an und versendet den neuen Status der
Welt. Damit beginnt die n{\"{a}}chste Runde. Die Alternative ist es das Bewegungsvorschl{\"{a}}ge direkt angewendet werden, falls
der Abstimmungsvorgang zwischen den Controllern also schneller vonstatten geht bewegen sich die Roboter auch {\"{o}}fter.

Beide M{\"{o}}glichkeiten w{\"{a}}hren realisierbar, allerdings wird die Echtzeitmethode vorgezogen. Einerseits da, gerade bei einer
3D-Ansicht, es unangehnem auff{\"{a}}llt wenn auch nur wenige Millisekunden nichts passiert --- es wirkt als w{\"{u}}rde die
Simulation stocken. Desweiteren ist es auch eine unrealistische Vereinfachung gegen{\"{u}}ber der realen Welt w{\"{a}}hre,
die sich nun einmal konstant ver{\"{a}}ndert.

Dies hei{\ss} zwangsl{\"{a}}ufig das eine Event-gesteuerten Kommunikation eingesetzt werden muss. Sender und Empf{\"{a}}nger
sollen zeitlich entkoppelt sein, haupts{\"{a}}chlich ihre eigenen Aufgaben durchf{\"{u}}hren und erst zu "passenden" Zeitpunkten
mit den jeweils anderen Kommunikationspartnern interagieren. 

Damit die Studenten sich nicht um die Kommunikation zur Simulation k{\"{u}}mmern m{\"{u}}ssen brauchen sie eine Bibliothek welche
diese Funktionalit{\"{a}}t bereitstellt. Sie kann benutzt werden um einen neuen Roboter zu erstellen und diesen zu steuern.
Im Fehlermodell gibt es auch eine Anforderung das einzelene Voter andere Informationen {\"{u}}ber den Status der Welt
bekommen sollen als andere. Daher ist die Bibliothek nicht nur f{\"{u}}r die reine Kommunikation zust{\"{a}}ndig, sondern wird auch
benutzt um diese Art von Fehler zu injizieren.



\clearpage
\part{Konzept}
\clearpage
\section{Simulationsidee}
Es wurde sich entschieden die Grundidee der Simulation beizubehalten, diese allerdings neu zu implementieren um die Kritikpunkte am
bestehenden System auszumerzen. Im folgenden wird die Grundidee noch einmal konkretisiert:

Die simulierte Welt besteht aus einer kreisf{\"{o}}rmigen Platte. Auf dieser k{\"{o}}nnen bis zu \gls{N}
Roboter und $1$ F{\"{u}}llstation (\gls{F}) platziert werden. Die Summe alle Gewichte kippt die Platte;
sind die Gewichte zu ungleichm{\"{a}}{\ss}ig verteilt, kippt die Platte um und die Simulation endet.

Die Roboter k{\"{o}}nnen sich frei auf der Platte bewegen, verbrauchen dabei aber Energie und m{\"{u}}ssen diese dann irgendwann
an der F{\"{u}}llstation auff{\"{u}}llen. Auch wenn die Roboter sich nicht bewegen, verbrauchen sie Energie (allerdings weniger
schnell).

Gesteuert werden die Roboter von bis zu \gls{M} $(M \geq N)$ Controllern. Einer oder mehr Controller stimmen sich ab,
so dass:
\begin{itemize}
\item die Roboter ihre Energie immer rechtzeitig auff{\"{u}}llen
\item die Roboter nicht kollidieren
\item die Platte nicht kippt
\end{itemize}

Dieser Abstimmungsvorgang sollte fehlertolerant implementiert werden, so dass auch beim Ausfall eines (oder mehrerer) Controller die Roboter sich immer noch koordiniert bewegen.
(Abh{\"{a}}ngig vom Verh{\"{a}}ltnis der Anzahl der Controller zur Anzahl der Roboter k{\"{o}}nnen verschiedene Stufen der Fehlertoleranz erreicht werden.\cite[s.149]{Werner00})
Daraus folgt das die Studierenden in ihrer L{\"{o}}sung eine Konsensbildung implementieren m{\"{u}}ssen.

Die Voter und Controller laufen auf Virtuellen Maschinen, jeweils eine VM pro Voter- oder Controllerinstanz.
Diese VMs befinden sich in einem Netzwerk, das durch eine Fehlerinjektion UDP Pakete verf{\"{a}}lscht --
dies simuliert ein fehlertr{\"{a}}chtiges Netzwerk, wie es zum Beispiel bei Funknetzwerken der Fall ist.
Es wird vorgegeben, dass die gesamte Kommunikation der Studentenprogramme {\"{u}}ber UDP stattfinden muss.
Dies f{\"{u}}hrt dazu das die Studierenden Gegenmassnahmen gegen die Verf{\"{a}}lschung ergreifen m{\"{u}}ssen, denkbar
w{\"{a}}hre hier zum Beispiel eine Kanalkodierung zur Vorw{\"{a}}rtsfehlerkorrektur.
Die Kommunikation von Simulation zu den Votern ist nicht betroffen, sie soll nicht verf{\"{a}}lscht werden. (Statdessen
kann die Schnittstelle gezieltere Verf{\"{a}}lschungen vornehmen.)

\begin{figure}
	\centering
	\includevisio[width=\textwidth]{Netzwerk}
	\caption{Aufteilung der Netzwerkteilnehmer}
	\label{fig:network}
\end{figure}

Zusammenfassend besteht das ganze System aus diesen Teilen:
\begin{itemize}
	\item Die Simulation berechnet die Bewegungen und implementiert das physikalische Modell. Sie {\"{u}}bernimmt auch die Visualisierung. Per Definition ist sie ausfallsicher.
	\item Der Voter sammelt die Steuerkommandos von den Controllern, bildet daraus das Steuerergebniss und sendet dies an die Simulation. Er ist per Definition ausfallsicher gegen \textit{crash failures}.
	\item Die Controller generieren Steuerkommandos. Sie k{\"{o}}nnen jederzeit ausfallen.
\end{itemize}

\paragraph{Nutzung durch die Studenten} Von den Studierenden muss nicht die gesamte Simulation erstellt werden;
die {\"{U}}bungsaufgabe bezieht sich nur auf die Abstimmung zwischen den Controllern und die Ansteuerung der Roboter.
Daher ist es notwendig, ein Interface ins "Innere" der Simulation bereitzustellen, das dann von den Studenten genutzt wird.

\clearpage
\section{Architekturkonzept}
Nach dem ausarbeiten der Anforderungen an das Simulationswerkzeug und die Entscheidung {\"{u}}ber den Simulationsablauf (inklusive
der Trennung zwischen f{\"{u}}r die Studierenden vorgefertigten Teil und den von diesen selbst zu implementierenden) ist es n{\"{o}}tig
die Systembestandteile aufzulisten. Dabei werden auch die Beziehungen zwischen ihnen aufgef{\"{u}}hrt, da diese Einfluss auf die
Architektur haben.

\begin{figure}
	\centering
	\includevisio[width=\textwidth]{Komponenten1}
	\caption{Ben{\"{o}}tigte Systembestandteile und ihre Beziehungen}
	\label{fig:arch_pre}
\end{figure}
\footnote{Hierbei sind die Bestandteile der Simulation an sich in Blau und die von den Studierenden zu erstellenden Teile in Grau
dargestellt}

Der Hauptbestandteil der Simulation ist die Physiksimulation. Sie ist daf{\"{u}}r zust{\"{a}}ndig
das sich die Roboter bewegen, betankt werden k{\"{o}}nnen, muss Informationen {\"{u}}ber den
Status der Simulation (z.B. Roboterpositionen) bereitstellen und auch die Roboter basierend
auf den Steuerbefehlen der Voter bewegen (unter Beachtung aller physikalischen Eigenschaften
des Roboters und der Fehlerinjektion).

Um die Roboter zu steuern muss a) der Status der Roboter bekannt sein b) es f{\"{u}}r den
Voter m{\"{o}}glich sein die Steuerbefehle an die Simulation weiter zu reichen. Beides
erm{\"{o}}glicht das Interface.

F{\"{u}}r die Fehlerinjektion ist der Fehlerinjektionssmanager zus{\"{a}}ndig. Die Physiksimulation
benutzt diesen um Roboterspezifische Fehler\ref{fm-robot} zu injezieren, w{\"{a}}hrend das
Interface basierend auf dem Fehlerinjektionsmanager die Informationen {\"{u}}ber den
Status der Welt verf{\"{a}}lscht.

Um den Simulationsstatus verfolgen zu k{\"{o}}nnen ist eine visuelle Anzeige hilfreich. Daher
ist ein Bestandteil der Simulation ein Anzeigeprogramm.

Jede Voterinstanz ist daf{\"{u}}r zust{\"{a}}ndig ihren Roboter in der Simulation zu steuern.
Daf{\"{u}}r m{\"{u}}ssen die Steuerkommandos an die Physiksimulation weitergeleitet werden,
dies {\"{u}}bernimmt das Interface. Der Controller ist komplett den Studenten {\"{u}}berlassen ---
Es gibt nur die Einschr{\"{a}}nkung das alle Kommunikation zur Simulation hin, {\"{u}}ber den Voter erfolgen muss.

\clearpage
\section{Kommunikation zwischen den Systemteilen}
Bei verteilten Systemen (wie diesem) ist einer der wichtigsten Punkte wie die Kommunikation zwischen den Systemteilen abl{\"{a}}uft.
Hierbei kann zwischen zwei Gruppierungen unterschieden werden: einmal die Kommunikation zwischen Simulation und Voter und andererseits
die Kommunikationen zwischen den Controllern und zwischen Controller und Voter. Da die letztere von den Studierenden geplant werden soll,
wird sie hier nicht weiter untersucht.

Da alle Voter in der gleichen Welt miteinander interagieren sollen, ist es am einfachsten wenn die Simulation ein Server ist, zu dem sich alle
verbinden. Dieser Server kann dann intern die Simulation vorantreiben und nur die Ergebnisse dieser weiterversenden. Die Alternative
w{\"{a}}re ein System ohne zentrale Instanz in dem die Simulation auf jedem Robter ausgef{\"{u}}hrt wird. In diesem Fall gibt es
allerdings das Problem das sich der Status der Simulationen voneinander desychnorisieren kann, zum Beispiel wenn ein Voter
eine schw{\"{a}}cheren CPU hat, die l{\"{a}}nger zur Berechnung eines Simulationstaktes braucht als die anderen Voter. Daher ist
das Modell mit zentrale Instanz vorzuziehen; es stellt automatisch sicher das alle Voter (im Normalfall, also ohne Fehler)
die gleichen Informationen kriegen und anhand dieser dann reagieren.

Durch die Verwendung einer zentralen Instanz ist die eine H{\"{a}}lfte der Isolation zwischen verschiedenen Instanzen realisiert:
Die Voter jeder Instanz verbinden sich mit ihrer Instanz, diese sendet ihre Informationen nur an die verbundenen Voter.
\footnote{Der andere Teil --- die Isolation zwischen verschiedenen Instanzen in Bezug auf Controller und Voter --- kann entweder
durch die Benutzung verschiedener Ports oder den Gebrauch mehrere \textit{subnets} erreicht werden.}

Ein Nachteil dieser L{\"{o}}sung ist das sie einen \textit{single point of failure} erschafft. Fehler in dem Simulationsprogramm
wirken sich direkt auf das ganze System aus; daher wird definiert das dieses keine Fehler enth{\"{a}}hlt.

F{\"{u}}r den zeitlichen Ablauf in der Simulation sind zwei M{\"{o}}glichkeiten denkbar: eine rundenbasierte Abfolge oder eine
Abfolge in Echtzeit. Bei der rundenbasieren Simulationstrategie wird der Status der Simulation immer nach festgelegten Intervallen
ver{\"{a}}ndert. In diesem Fall w{\"{a}}re es denkbar das die Controller und Voter 250ms Zeit haben ihre Bewegungsvorschl{\"{a}}ge
zu berechnen und diese an die Simulation schicken. Diese sammelt sie, wendet sie alle auf einmal an und versendet den neuen Status der
Welt. Damit beginnt die n{\"{a}}chste Runde. Die Alternative ist es das Bewegungsvorschl{\"{a}}ge direkt angewendet werden, falls
der Abstimmungsvorgang zwischen den Controllern also schneller vonstatten geht bewegen sich die Roboter auch {\"{o}}fter.

Beide M{\"{o}}glichkeiten w{\"{a}}hren realisierbar, allerdings wird die Echtzeitmethode vorgezogen. Einerseits da, gerade bei einer
3D-Ansicht, es unangehnem auff{\"{a}}llt wenn auch nur wenige Millisekunden nichts passiert --- es wirkt als w{\"{u}}rde die
Simulation stocken. Desweiteren ist es auch eine unrealistische Vereinfachung gegen{\"{u}}ber der realen Welt w{\"{a}}hre,
die sich nun einmal kontinuierlich ver{\"{a}}ndert.

Dies hei{\ss} zwangsl{\"{a}}ufig das eine Event-gesteuerten Kommunikation eingesetzt werden muss. Sender und Empf{\"{a}}nger
sollen zeitlich entkoppelt sein, haupts{\"{a}}chlich ihre eigenen Aufgaben durchf{\"{u}}hren und erst zu "passenden" Zeitpunkten
mit den jeweils anderen Kommunikationspartnern interagieren. 

Damit die Studenten sich nicht um die Kommunikation zur Simulation k{\"{u}}mmern m{\"{u}}ssen brauchen sie eine Bibliothek welche
diese Funktionalit{\"{a}}t bereitstellt. Sie kann benutzt werden um einen neuen Roboter zu erstellen und diesen zu steuern.
Im Fehlermodell gibt es auch eine Anforderung das einzelene Voter andere Informationen {\"{u}}ber den Status der Welt
bekommen sollen als andere. Daher ist die Bibliothek nicht nur f{\"{u}}r die reine Kommunikation zust{\"{a}}ndig, sondern wird auch
benutzt um diese Art von Fehler zu injizieren.

\clearpage
\section{Das zu simulierende Fehlermodell}\label{fm}
Das Ziel der Aufgabe ist es, Studierenden etwas {\"{u}}ber ausfallsichere Systeme beizubringen. Daf{\"{u}}r
muss auch etwas ausfallen, denn nur dann wird getestet ob die Auswirkungen des Ausfalls abgemindert werden 
konnten.

Im Fehlermodell wird festgelegt, was alles -- und auf welche Weise -- ausfallen kann. Dabei wird f{\"{u}}r jeden Bestandteil der Simulation ein eigenes Fehlermodell aufgestellt. Die Bestandteile
der Simulation sind hier die simulierte Welt mit ihren Robotern (und den dazugeh{\"{o}}rigen Motoren/Aktoren
und Sensoren), die Kommunikationswege zwischen Voter und Controller und die Controller selbst.

\subsection{Roboter / Motoren}\label{fm-robot}
Die simulierten Roboter {\"{a}}hneln dem Khepera Roboter und haben daher 2 Motoren. Diese beiden Motoren k{\"{o}}nnen unabh{\"{a}}ngig von einander fehlerhaft sein. Fehlerhaft
hei{\ss}t hier:
\begin{itemize}
	\item Die H{\"{a}}ufigkeit von Hardwarefehler wird oft mithilfe der Badewannenkurve modelliert. Sie besagt,
		dass Hardware entweder kurz nach Inbetriebnahme oder nach langer Betriebszeit mit hoher
		Wahrscheinlichkeit fehlerhaft ist, und in der Zwischenzeit nur mit niedriger Wahrscheinlichkeit.
		F{\"{u}}r die Motoren der Roboter wird ein vereinfachtes Modell genutzt, in dem die Wahrscheinlichkeit 
		f{\"{u}}r einen Ausfall bis zu einer bestimmten Betriebslaufzeit einstellbar ist und danach auf
		Null sinkt. Falls der Motor in dieser Zeit Fehler aufweist, bleibt diese Fehler auch.
	\item Desweiteren kann ein Motor tempor{\"{a}}r ein \textit{stuck-at}-Verhalten aufweisen. Dieses
		kann jederzeit passieren und entweder tempor{\"{a}}rer oder permament sein.
		Die Wahrscheinlichkeiten f{\"{u}}r das Auftreten eines permamenten oder tempor{\"{a}}ren
		\textit{stuck-at}-Fehlers sind unabh{\"{a}}ngig voneinander einstellbar.
\end{itemize}

Da jeder der \gls{N} Roboter 2 Motoren hat, k{\"{o}}nnten theoretisch auch 2 Motoren kaputt gehen. Ein Roboter mit einem fehlerhaften Motor kann allerdings immernoch sinnvoll genutzt werden,
bei einem Roboter mit 2 fehlerhaften Motoren ist dies nicht der Fall. Es soll m{\"{o}}glich sein verschiedene 
Schwierigkeitsstufen der Simulation einzustellen. Daher ist es m{\"{o}}glich
dass nur ein Motor pro Roboter gleichzeitig Fehler aufweisen kann.

Aus dem gleichen Grund ist es parametresierbar, wie viele Roboter gleichzeitig irgendeine Art von Fehler aufweisen k{\"{o}}nnen.

Zus{\"{a}}tzlich zu diesen Fehler soll es auch m{\"{o}}glich sein, einen Roboter fernzusteuern. W{\"{a}}hrend
der Fernsteuerung darf dieser Roboter keine Steuerbefehle seines Voters ausf{\"{u}}hrend sondern wird rein
vom Nuter gesteuert. Diese Art der Fehlerinjektion w{\"{u}}rde in Richtung byzantinische Fehler gehen,
denn dies w{\"{u}}rde es erm{\"{o}}glichen, gezielt die Platte zu destabilisieren oder Roboter zu
blockieren.

\subsection{Weltstatusinformationen}
Die Roboter haben ihre Aktoren, um sich in der Welt zu bewegen. Damit sie sich sinnvoll bewegen und die
Platte balanciert k{\"{o}}nnen, brauchen sie Informationen {\"{u}}ber den Status der Welt und ihre
(und die der anderen Roboter) Position in dieser. Diese Informationen sind direkt verf{\"{u}}gbar
und m{\"{u}}ssen nicht, zum Beispiel odeometrisch, erst herausgefunden werden. In der realen Welt sind
Sensoren, wie sie hier vielleicht h{\"{a}}tten simuliert werden k{\"{o}}nnen, nat{\"{u}}rlich manchmal 
fehlerbehaftet. Um dies nachzubilden k{\"{o}}nnen, die Informationen {\"{u}}ber den Weltstatus auf
verschiedene Arten fehlerhaft an den Roboter weitergeleitet werden.

Als erstes ist es m{\"{o}}glich, dass der Roboter einfach keinerlei Informationen mehr bekommt, also ein \textit{omission failure} vorliegt. Ein \textit{stuck-at} Fehler liegt
vor wenn veralterte Informationen wieder an den Roboter weitergegeben werden, so als w{\"{u}}rde ein Sensor nur noch seine letzen g{\"{u}}ltigen Daten liefern. Der dritte
Fall liegt vor, wenn komplett falsche Daten geliefert werden. Komplett falsch hei{\ss}t, dass die Anzahl und die Positionen, Gewichte etc. der Roboter potenziell fehlerhaft an
den Roboter weitergegeben werden. Die Wahrscheinlichkeiten f{\"{u}}r die jeweiligen F{\"{a}}lle lassen sich unabh{\"{a}}ngig von einander einstellen.

\subsection{Netwerk}
Da die simulierten Roboter den Khepera Robotern {\"{a}}hneln sollen, sollen auch ihre 
Kommunikationsm{\"{o}}glichkeiten denen der Khepera Roboter {\"{a}}hneln. Bei beweglichen
Objekten sind Ethernetkabel suboptimal, weswegen diese Funkverbindungen -- im Falle der Khepera
Roboter ist dies WLAN -- bevorzugen. Funkverbindungen haben aber, im Gegensatz zu gut abgeschirmten
Ethernetkabeln, Probleme mit Interferenzen durch andere Funk{\"{u}}bertragungen.

Dies f{\"{u}}hrt dazu das w{\"{a}}hrend der {\"{U}}bertragung Bits, oder sogar ganze Bytes, verf{\"{a}}lscht werden. Auch die Wahrscheinlichkeit f{\"{u}}r diesen Fall
soll einstellbar sein. Vereinfachend wird hier angenommen das jeweils nur ein Byte pro Paket verf{\"{a}}lscht wird, es gibt also keine \textit{burst errors}.
Es wird angenommen das die meiste Kommunikation zwischen Robotern "in Sichtweite" geschieht, ohne Zwischenstationen und komplexes Routing. In diesem Fall ist
es ist un{\"{u}}blich das Pakete verloren gehen oder zeitverz{\"{o}}gert weitergeleitet werden, daher ist dies nicht Teil des Fehlermodells.

\subsection{Controller}
Es wird davon ausgegangen, dass die Controllerprogramme auf unzuverl{\"{a}}ssiger Hardware, unzuverl{\"{a}}ssigen 
Betriebssystem laufen und fehlerhaft programmiert sind. Daher muss davon ausgegangen werden, dass die
Programme jederzeit abst{\"{u}}rzen k{\"{o}}nnen. Desweiteren kann auch nicht davon ausgegangen werden,
dass die Controllerprogramme das einzige Programm ist, das l{\"{a}}uft; es kann also auch zu Problemen bei
der Resourceverteilung (RAM, CPU Zeit) kommen.

\subsection{Auswirkungen}
Nachdem Fehlermodelle f{\"{u}}r die verschiedenen Elemente des Systems entwickelt wurden, kann
betrachtet werden, welche Auswirkungen die verschiedenen Fehler haben k{\"{o}}nnen. Hierbei werden nur
die Auswirkungen in Bezug auf die Platte betrachtet.

\paragraph{Unkontrollierte Bewegung} Falls sich ein Roboter unkontrolliert bewegt (also ein oder
mehrere Motoren einen \textit{stuck-at}-Fehler und die verbliebenden Motoren gar nicht mehr funktionieren)
kann dieser Roboter nicht mehr genutzt werden um die Platte auszubalancieren.
\begin{figure}
	\centering
	\includevisio[width=\textwidth]{fm_robot}
	\caption{Fehlerbaum: Unkontrollierte Bewegung eines Roboters}
	\label{fig:fault-tree-robot}
\end{figure}
\clearpage

\paragraph{Inkorrekte Bewegungen} Ein Roboter bewegt sich "falsch", wenn seine Bewegung weder die Platte 
ausbalanciert noch dazu gedacht, ist den Energiespeicher aufzuf{\"{u}}llen. Dies kann verschiedene Gr{\"{u}}nde 
haben: Entweder bewegt der Roboter sich unkontrolliert oder er wird falsch angesteuert. Beide F{\"{a}}lle
passieren falls die Controller falsche Steuerkommandos senden, oder theoretisch richtige Steuerkommandos
falsch ausgewertet werden.

Die tats{\"{a}}chliche Auswirkung dieses Zustandes h{\"{a}}ngt von der momentanen Stellung der Roboter (und ihrer
Stati) auf der Platte ab und kann nicht allgemein bestimmt werden. Generell l{\"{a}}sst sich nur sagen das dieser
Roboter nicht mehr genutzt werden kann um die Platete auszubalancieren und die verbleibenden Roboter nun auch noch
die Bewegungen dieses Roboters ausgleichen m{\"{u}}ssen.

\begin{figure}
	\centering
	\includevisio[width=\textwidth]{fm_robot2}
	\caption{Fehlerbaum: Inkorrekte Bewegung eines Roboter}
	\label{fig:fault-tree-robot2}
\end{figure}



\clearpage
\part{Grundlagen}
\clearpage
\section{3D Engine}
\subsection{Anforderungen}
Die Simulation, vor allem die Bewegung der Roboter, kann besonders gut visuell dargestellt werden. Auch wenn sich das Geschehen rein auf einer zweidimensionalen Ebene stattfindet, ist eine dreidimensionale
Darstellung ansprechender und wird deshalb vorgezogen.

Diese Darstellung geschieht mithilfe einer \textit{3D Engine}; sie bietet die M{\"{o}}glichkeit Objekte und oft auch Lichtquellen und Schattenspiel darzustellen.
Anstatt solch eine Engine selbst zu implementieren, wird eine vorhandene genommen.

Um die sehr gro{\ss}e Auswahl einzuschr{\"{a}}nken, wurden die nachfolgenden Kriterien benutzt.

\paragraph{Kosten} Die 3D Engine muss f{\"{u}}r den nicht-kommerziellen edukativen Gebrauch kostenlos sein;
dies schlie{\ss}t zum Beispiel die CryEngine\cite{cryengine} automatisch aus.

\paragraph{Plattformunabh{\"{a}}ngigkeit} Die mit der \textit{game engine} entwickelte Simulation muss auf
ohne gro{\ss}en Aufwand auf mehrere System (hier Windows und Linux) portierbar sein.

\paragraph{Funktionalit{\"{a}}t} Eine Untermenge der 3D Engines sind \textit{game engines}, diese beeinhalten 
zus{\"{a}}tzlich zur reinen 3D Engine auch eine Physikengine und eine Audioengine.
Durch die Art der Simulation ist ersichtlich, dass eine Physikengine viel Arbeit ersparen k{\"{o}}nnte, gerade im Hinblick auf Beschleunigung, Bewegung oder Kollisionserkennung. Daher werden \textit{game engines} reinen 3D Engines (wie Blender) vorgezogen.

\paragraph{Einfache Benutzung} Das Hauptkriterium ist die einfache Benutzung. Dies beeinhaltet zum einen die
tats{\"{a}}chliche Nutzung (also: gibt es einen graphischen Leveleditor? Wie kann die Interaktionen zwischen
Objekten gesteuert werden?), zum anderen die Qualit{\"{a}}t der Dokumentation. Und schließlich wie verbreitet diese
\textit{engine} ist, und damit auch wie einfach es ist bei Problemen Hilfestellungen zu bekommen.

Gerade in diesem Punkt stach Unity hervor und wurde deshalb als Basis f{\"{u}}r diese Arbeit ausgew{\"{a}}hlt.

\subsection{Unity}
Unity ist eine von Unity Technologies entwickelte \textit{game engine}, die 2005 f{\"{u}}r Macintosh entwickelt wurde und in der Zwischenzeit auf 27 Plattformen portiert wurde.

In dieser Arbeit wird es benutzt, um die Welt und die Interaktionen zwischen den Objekten zu simulieren und diese auch anzuzeigen. Mithilfe einer Scriptingschnittstelle kann das Geschehen modifiziert werden, z.B. kann die Welt
abh{\"{a}}ngig von der Position und den Gewichten der Objekt gekippt werden oder die Kamera durch Tastatureingaben bewegt werden.

\subsection{Unitys Grafikengine}
Die in Unity enthaltene Grafikengine ist eine vollwertige 3D-Grafikengine, die es erm{\"{o}}glicht professionelle Videospiele zu programmieren. In dieser Arbeit wird nur
ein kleiner Teil der Funktionalit{\"{a}}t benutzt; komplexere Features wie die Anzeige von Rauch oder realistischem Wasser sind nicht in der Simulation enthalten.

Notwendig ist es aber, dass die verschiedenen Objekte in der simulierten Welt in 3D zu sehen sind und sich die Kamera bewegen l{\"{a}}sst, um das simulierte Geschehen zu
verfolgen. Dies geh{\"{o}}rt zur Grundfunktionalit{\"{a}}t und muss nicht extra implementiert werden.

\subsection{Unitys Physikengine}
Ein integraler Teil moderner Videospiele ist eine realistische physikalische Umgebung. Dies wird durch die Physikengine von Unity erm{\"{o}}glicht, die in dieser Arbeit wichtigen Funktionalit{\"{a}}ten sind:

\begin{itemize}
\item die Kollisionserkennung, einerseits um zu bestrafen wenn Roboter in einander fahren,
	und andererseits zur Erkennung des Kontaktes mit der F{\"{u}}llstation
\item die Bewegung der Objekte in der Welt zu simulieren, abh{\"{a}}ngig von ihrem Gewicht und ihrer Geschwindigkeit
\end{itemize}

\subsection{Scripting bei Unity}
Alle Elemente der Unity Engine k{\"{o}}nnen vom Programmierer gesteuert werden, daf{\"{u}}r k{\"{o}}nnen
die Skriptsprachen C\#, JavaScript und Boo genutzt werden.\cite{wiki:unity}
Um dies zu erm{\"{o}}glichen ist es aus dem Skript heraus m{\"{o}}glich, alle Eigenschaften
der Objekte \footnote{Dies beinhaltet nicht nur die Roboter und F{\"{u}}llstationen sondern auch die Kamera
oder statische Objekte wie Mauern.} auszulesen und zu manipulieren.

Beispielsweise wurde der F{\"{u}}llstation ein Skript zugeordnet, dass solange ein Objekt mit dieser kollidiert, testet, ob dies Objekt ein Roboter ist, und falls dies der Fall ist,
den Roboter auftankt. Ein anderes Skript bewegt die Kamera abh{\"{a}}ngig von den Tastatureingaben des Benutzers.

Um die Skripte in Unity zu integrieren, m{\"{u}}ssen sie bestimmte Eigenschaften haben. Alle Klassen die von UnityEngine.MonoBehaviour
abstammen und registriert sind werden w{\"{a}}hrend der Berechnung jedes Frames aufgerufen. Der genau Zeitpunkt dieses Aufrufes h{\"{a}}ngt von
der Funktion ab\footnote{Vgl. \cite{unity-exec-order}}, beispielsweise wird Update() genau einmal pro Frame aufgerufen nachdem die Physikengine fertig ist
und Inputevents behandelt wurden. Im Gegensatz dazu kann FixedUpdate() mehrmals oder sogar gar nicht w{\"{a}}hrend der Renderung eines Frames aufgerufen
werden, abh{\"{a}}ngig von der Framerate. Daf{\"{u}}r wird FixedUpdate() vor der Physikengine aufgerufen, {\"{A}}nderungen die in dieser
Funktion vorgenommen werden, haben also schon Auswirkungen auf den momentanen Frame. Desweiteren muss der Klassenname innerhalb einer Datei so
hei{\ss}en wie die Datei selbst.

Da Konstruktor und Dekonstrktor nicht {\"{u}}berschrieben werden k{\"{o}}nnen, gibt es die virtual functions Start() und OnDestroy(). Diese
{\"{u}}bernehmen die Aufgaben des Konstruktor und Dekonstrktors.

Es gibt auch weitere {\"{u}}berschreibare Funktionen, mit denen es m{\"{o}}glich ist sich in verschiedene Zwischenschritte des Renderprozesses
einzuklinken (zum Beispiel OnWillRenderObject, OnPostRender, ...). Das gleiche gilt auch f{\"{u}}r Inputevents (beispielsweise OnMouseMove) und
die Ereginisse der Physikengine (OnCollisionStart(), ...).


\clearpage
\section{RPC}
Der Grundgedanke von verteilten System ist es, verschiedene Systeme verschiedene Aufgaben
erledigen zu lassen. Oft gibt es aber Abh{\"{a}}ngigkeiten zwischen Aufgaben (Aufgabe A muss beendet werden
bevor Aufgabe B angefangen werden kann, zum Beispiel weil das Ergebniss von A f{\"{u}}r B wichtig
ist) oder die Ergebnisse von mehreren System m{\"{u}}ssen zusammengef{\"{u}}hrt werden. Zwischen
den Systemen muss also eine Synchronisation und Kommunikation m{\"{o}}glich sein.

Die Kommunikation
an sich kann {\"{u}}ber verschiedene {\"{U}}bertragungsmedien erfolgen, allerdings muss die
Kommunikation an die spezifischen Anforderungen der Applikation angepasst werden. Zum Beispiel
ist die Kommunikation zwischen Systemen die komplexe Wettermodelle berechnen, eine andere als die
Kommunikation, um Konsistenz zwischen verteilten Datenbankinstanzen herzustellen.\todo{kann ich das einfach
behaupten?}

Abhilfe schaffen \textit{remote procedure call}, mit denen es f{\"{u}}r die verschiedenen Netzwerkteilnehmer
so wirkt, als k{\"{o}}nnten sie auf den anderen Netzwerkteilnehmern Funktionen direkt aufrufen. Welche
Funktionen dies sind, h{\"{a}}ngt von den jeweiligen Anforderungen der Anwendung oder des Systems ab.

Viele RPC Bibliotheken abstrahieren den Aufruf einer Funktion {\"{u}}ber das Netzwerk so gut das es auch
m{\"{o}}glich ist in verschiedenen Sprachen zu programmieren. Dies kann die Programmierung erleichtern, in dem
die St{\"{a}}rken verschiedener Sprachen (zum Beispiel die Abw{\"{a}}gung zwischen Programmieraufwand und
Programmeffizenz) geschickt kombiniert werden.

\clearpage
\section{Interaktionsmuster}
In einem verteilten System m{\"{u}}ssen die verschiedenen Rechner miteinander kommunizieren um
ihr Ziel zu erreichen. Die Kommunikation kann verschiedene Arten annehmen. Um dies zu verdeutlichen
werden diese Mustern mit menschlicher Kommunikation verglichen.

\subsection{Request-Reply}
Bei einer Kommunikation zwischen zwei Menschen / System, oder allgemeiner: Kommunikationspartnern,
ist meistens so das einer redet, dann die andere Person antwortet. Es gibt also \textit{request}
und \textit{reply}.

Diese Form der Kommunikation ist gut geeignet f{\"{u}} Client-Server Kommunikation wie das
aufrufen von Websites. Diese Art von Kommunikation f{\"{u}}hrt zu Problemen falls sich die
Informationen schnell {\"{a}}ndern -- die verschiedenen Clients agieren dann vielleicht basierend
auf leicht unterschiedlichen Informationen, was zu Problemen f{\"{u}}hren kann.

\subsection{Publish-Subscribe}
Die andere intuitiv verst{\"{a}}ndliche Kommunikationsform ist der Votrag.

Die Simulation muss mit einer nicht vorab bestimmbaren Anzahl von Votern/Robotern arbeiten. Die Kommunikationspartner k{\"{o}}nnen auch jederzeit ausfallen oder es k{\"{o}}nnen neue hinzukommen.
Eine erste L{\"{o}}sungm{\"{o}}glichkeit w{\"{a}}hre es, die Kommunikation {\"{u}}ber UDP Broadcasts
laufen zu lassen. Dadurch w{\"{u}}rden alle Kommunikationspartner automatisch die Nachricht empfangen, ohne
dass sich der Sender darum k{\"{u}}mmern muss an wen Nachrichten gesendet werden m{\"{u}}ssen. Da die
Voter aber in unterschiedlichen Subnetzen sein k{\"{o}}nnen, ist dies nicht m{\"{o}}glich.

Abhilfe schafft das Publish-Subscribe Kommunikationspattern\cite{pubsub}. Dort gibt es einen Publisher (in
diesem Fall die Simulation) und Subscriber (die Voter/Roboter). Die Subscriber verbinden sich mit dem
Publisher und teilen ihm mit, welche Art von Nachrichten sie interessiert. Der Publisher verteilt dann die
Nachrichten, die zu nicht vorhersehbaren Zeitpunkten kommen, anhand dieser Informationen an alle Subscriber zu denen momentan ein Kontakt m{\"{o}}glich ist.

Dies hat drei gro{\ss}e Vorteile. Es gibt eine r{\"{a}}umliche Trennung, das hei{\ss}t der Publisher muss
nichts {\"{u}}ber die Subscriber (zum Beispiel Anzahl, IP Adresse) und andersrum wissen. Es reicht
wenn beide wissen wo sie sich "treffen" k{\"{o}}nnen. Als zweites gibt es eine zeitliche Trennung zwischen
den Kommunikationspartner -- der Publisher kann Nachrichten verschicken, ohne sich dar{\"{u}}ber Gedanken
machen zu m{\"{u}}ssen wann diese beim Subscriber ankommen, w{\"{a}}hrend es f{\"{u}}r den Subscriber
uninteressant ist wann die Nachricht versendet wurde. Der letze Vorteil ist die Entkoppelung von
Nachrichten{\"{u}}bertragung und den anderen Aufgaben des Programmes. Die Nachrichten werden Daten
von ihrer Herstellung verschickt, das Senden einer Nachricht ist also (aus Sicht des
Publishers) O(1) in Bezug auf die Anzahl der Subscriber. Auch auf Seiten des Subscribers gibt es eine
Asynchronit{\"{a}}t, die empfangen Daten werden mithilfe eines Callback angenommen und blockieren damit
nicht die Abarbeitung dieser Daten.

\begin{figure}
	\centering
	\includevisio{pubsub}
	\caption{Publish-Subscribe}
	\label{fig:pubsub}
\end{figure}
\todo{sieht kacke aus. kleiner und andere Auflösung?}


\clearpage
\section{Netzwerklibrary}
Um die Kommunikation zwischen den einzelnen Netzwerkteilnehmern nicht komplett selbst zu implementieren, k{\"{o}}nnen bereits bestehende Netzwerklibraries genutzt werden.
Diese bieten, zum Beispiel, die M{\"{o}}glichkeit an einem entfernten Netzwerkteilnehmer Funktionen aufzurufen oder sich f{\"{u}}r Multicasts anzumelden, auch wenn man in einem anderen
\textit{subnet} ist.

\subsection{Anforderungen}
Um verteilte Systeme zu programmieren, bietet sich das RPC Modell an, mit dem auf einem anderen Rechner Funktionen aufrufen kann.

Es gibt allerdings keine Netzwerklibrary, die dieses Pattern und RPC anbietet; da es einfacher ist, auf Publish-Subscribe ein RPC Modell aufzubauen, als dieses Pattern, zum Beispiel, grpc beizubringen, wird
diese Funktionalit{\"{a}}t vorgezogen.

Netzwerkbibliotheken, die in C++ geschrieben wurden, k{\"{o}}nnen zwar (mit einem Interface) auch unter C benutzt werden, f{\"{u}}gen dem Programm
dann aber "heimlich" die C++ Runtime hinzu. Eine rein in C geschriebene Bibliothek ist daher vorzuziehen.

\clearpage % make sure the table is, at least, in the right section
\begin{table}[h]
\centering
\begin{tabu}{c | c | c | c | p{5cm}}
	\toprule
	Name & RPC & Publish-Subscribe & C++ Runtimeabh{\"{a}}ngigkeit & andere Sprachen \\
	\midrule
	SunRPC & \checkmark & \xmark & \checkmark & C\#, Java, ... \\
	grpc\cite{grpc} & \checkmark & \xmark & \xmark & C\#, Python, Java, ... \\
	ZeroMQ\cite{zeromq} & \xmark & \checkmark & \xmark & C\#, Python, Java, ... \enquote{40+ languages} \\
	nanomsg\cite{nanomsg} & \xmark & \checkmark & \checkmark & C\#, Python, Java, ... (25 Sprachen) \\
	\bottomrule
\end{tabu}
\caption{Anforderungen an die Netzwerkbibliothek}
\end{table}

Aus dieser vergleichenden Darstellung wird klar, dass nanomsg am besten geeignet ist; auf dessen speziellen Eigenschaften wird nun n{\"{a}}her eingegangen.

\subsection{nanomsg}
nanomsg ist eine komplett in C geschrieben Netzwerkbibliothek, mit einer API die der POSIX Socket API gleicht (es wird sich also, zum Beispiel mit nn\_connect() verbunden oder mit nn\_send() gesendet).
Der gro{\ss}e Vorteil von nanomsg sind die implementierten {\"{U}}bertragungsarten. Je nach gew{\"{u}}nschter {\"{U}}bertragungsart wird ein unterschiedlicher Socket erstellt, dieser erlaubt die
Interaktion mit den anderen Netzwerkteilnehmern die auch diese {\"{U}}bertragunsart benutzen.

Beispielhaft wird nun auf einige der genutzten Arten eingegangen.

\paragraph{{\"{U}}bertragungsarten}
\subparagraph{Request-Reply} Die intuitiv verst{\"{a}}ndlichste Art zu kommunizieren ist es eine Frage zu stellen, die dann vom Gegen{\"{u}}ber beantwortet wird. In nanomsg ist dies die request-reply
{\"{U}}bertragungsart.

\clearpage
\section{Serialisierung}
Durch die Benutzung der Unity 3D Engine ist die Benutzung von C\# vorgeschrieben, der Rest der Arbeit soll in C geschreiben werden. Die Verwendung verschiedener Sprachen erschwert die Benutzung eines
Bin{\"{a}}rprotokoll -- aus diesem Grund werden die Daten serialisiert.

\subsection{Anforderungen}
Das wichtigste Auswahlkriterium der Serialsierungslibrary ist, dass es eine C (f{\"{u}}r das Studenteninterface) und eine C\# Version (f{\"{u}}r Unity) gibt.
Desweiteren sollen die serialisierten Nachrichten m{\"{o}}glichst klein sein, um auch nach der Anwendung
eines fehlerkorrigierenden Codes so wenig Redudanz wie m{\"{o}}glich aufzuweisen und damit nur wenig f{\"{u}}r Paketverf{\"{a}}lschungen anf{\"{a}}llig zu sein.

\begin{table}[h]
\centering
\begin{tabu}{c | c | c | c | p{5cm}}
	\toprule
	Name & Wenig Redudanz? & C Version & C\# Version \\
	\midrule
	JSON & \xmark & \checkmark & \checkmark \\
	XML & \xmark & \checkmark & \checkmark \\
	ProtoBuf & \checkmark & \xmark\footnotemark & \checkmark \\
	Msgpack\cite{msgpack} & \checkmark & \checkmark & \checkmark \\
	BSON & \checkmark & \checkmark & \checkmark \\
	\bottomrule
\end{tabu}
\caption{Anforderungen an die Serialisierungsbibliothek}
\end{table}
\footnotetext{Es existieren unoffizielle, nicht mehr weiterentwickelte Implementation} 

Bekannte Serialisierungsformate wie JSON und XML enthalten zu viel Redudanzen und k{\"{o}}nnen ausgeschlossen werden. Zur Auswahl bleiben damit beispielsweise BSON und MsgPack.
Bei BSON werden die Daten an sich effizient serialisiert, allerdings wird immer der Feldname mitgesendet. Dies erleichtert zwar die Versionierbarkeit, vergr{\"{o}}{\ss}ert
die Nachricht allerdings. Aus diesem Grund wird MsgPack als Serialisierungsformat genutzt.

\subsection{MessagePack}
MessagePack ist ein in {\"{u}}ber 50 Sprachen verf{\"{u}}gbares Serialisierungsframework, dessen Nachrichten besonders wenig Redudanzen enthalten sollen. Beispielhaft wird hier auf die Serialisierung von
Integern und Zeichenketten mit fester L{\"{a}}nge eingegangen.

Die Grundpidee von MessagePack ist es Datentypen nicht mit fester Breite sondern nur mit der minmal n{\"{o}}tigen Breite (oder wenigstens einer Ann{\"{a}}hrung an diese) abzuspeichern.
\paragraph{Integer} Da Datens{\"{a}}tze oftmals aus vielen kleinen Integerwerte bestehen, macht es Sinn gerade die Serialsierung dieser zu optimieren. Im Gegensatz zur "normalen" Verarbeitung als immer gleich breite (zum Beispiel 64bit) Zahlen, werden vorzeichenlose Integerwerte kleiner als 128 in einem Byte abgespeichert; der Wert des Bytes entspricht dabei dem Wert des Integers. Dahingegen signalisiert ein gesetztes MSB das es kein (kleiner) Integerwert sondern ein anderer Datentyp ist.
\paragraph{Fixed Size String} Um kleinere Zeichenketten zu encodieren, gibt es erst ein Signalzeichen,
in dem die obersten 3 bits gesetzt sind. Die restlichen 5bits geben die L{\"{a}}nge der Zeichenkette an --
also maximal $2^5 - 1 = 31$ Zeichen. Dies f{\"{u}}hrt zu einem durchschnittlichen Overhead von
$ \frac{1}{\frac{2^5-1}{2}} \approx 6\% $.

L{\"{a}}ngere Zeichenketten werden mit anderen Signalzeichen kodiert, beispielsweise 0xDB f{\"{u}}r eine Zeichenketten mit der Maximall{\"{a}}nge von $(2^{32})-1$ Zeichen.
In diesem Fall folgt die L{\"{a}}ng als 32bit big endian
\footnote{L{\"{a}}ngen werden immer in big-endian, kodiert um den Datenaustausch auch zwischen Rechnern unterschiedlicher Architekturen gew{\"{a}}hrleisten zu k{\"{o}}nen.}
Wert mit fester Breite. Also ein verschwindend geringer durchschnittlicher Overhead von $ \frac{1}{\frac{2^{32}-1}{2}} = 0.00000004\% $.

\clearpage
\section{Bewegungen}
Die simulierten Roboter sind an den Kepheraroboter angelehnt, einen gerade im unitversit{\"{a}}ren Bereich weit verbreiteten Roboter mit 2 Motoren und 2 starren R{\"{a}}dern.
Dies gibt auch das Bewegungsmodell vor: \hyperref[diffs]{\textit{differential steering}}, auch \textit{differential drive} genannt. Ausf{\"{u}}hrlicher wird dies im zugeh{\"{o}}rigen Unterkapitel erkl{\"{a}}rt.

\subsection{Khepera}\label{khepera}
Die Khepera-Roboterserie existiert seit Mitte der 90er Jahre, das neuste Modell ist der Khepera IV
\footnote{Vergleiche: \cite{Soares2016}}.
Dieser Roboter hat eine zylindrische Grundform mit einem Durchmesser von 140mm und einer
H{\"{o}}he von 58mm. Der Roboter wiegt 540g und kann zus{\"{a}}tzlich bis zu 2kg Nutzlast tragen.

Auf der Unterseite sind 2 starre R{\"{a}}der, angetrieben von jeweils einem elektrischen Servomotor.
Damit erreicht der Roboter Geschwindigkeiten von bis zu einem Meter pro Sekunde. Um die Position des
Roboters odeometrisch bestimmen zu k{\"{o}}nnen, geben die Motoren auch aus, wie oft sich das
jeweilige Rad schon gedreht hat. Zus{\"{a}}tzlich dazu hat der Roboter auch Infrarot-, Ultraschall-
und Lichtsensoren, mit denen die Umgebung wahrgenommen werden kann. Dies kann z.B. genutzt werden
um Kollisionen zu vermeiden. Mithilfe der integrierten Beschleunigungs- und Gyrosensoren kann
der Roboter auch seine Geschwindigkeit und seine Neigungswinkel feststellen.

Gesteuert wird der Khepera IV von einer ARM CPU, auf der ein vollwertiges Linux l{\"{a}}uft. Des weiteren gibt es auch ein WLAN Modul, eine Kamera und Lautsprecher plus Mikrophon.

\begin{figure}
	\centering
	\includevisio{k4-detail1.png}
	\caption{Der Khepera IV Roboter}
	\label{fig:kheperaiv}
\end{figure}
\clearpage

\subsection{Differential Steering}\label{diffs}
Beim \textit{differential steering} k{\"{o}}nnen die R{\"{a}}der nicht, wie z.B. beim Auto, geschwenkt werden -- sie sind fest. Es kann zwischen drei Bewegungsarten unterschieden werden:
\begin{itemize}
\item Falls die R{\"{a}}der gleich schnell drehen, f{\"{a}}hrt der Roboter gerade nach vorne oder hinten.
\item Falls die R{\"{a}}der unterschiedlich schnell, aber in die gleiche Richtung rotieren, wird eine Kurve gefahren. Dabei dreht sich der Roboter in Richtung des langsamer rotierenden Rades.
\item Falls die R{\"{a}}der gleisch schneller, aber in unterschiedliche Richtunge rotieren, wendet der Roboter auf der Stelle.
\end{itemize}

Diese Bewegungen k{\"{o}}nnen auch mathematisch ausgedr{\"{u}}ckt werden. Dabei gibt $V_r(t)$ die Rotationsgeschwindigkeit des rechten Motors zum Zeitpunkt t an, w{\"{a}}hrend $V_l(t)$ den
linken Motor beschreibt. \gls{R} ist der Radius eines Rades, und \gls{L} die Distanz zwischen den R{\"{a}}dern.
Folgende Gleichungen aus \cite{Dudek2010, Egerstedt}:
\begin{subequations}\label{eq:diffs}
\begin{align}
	\Theta(t) &= \Theta(0) + \frac{R}{L} \int_0^t \mathrm{v_R(t) - v_L(t)}\mathrm{d}t \label{eq:diffs-theta}\\
\intertext{Mithilfe der Orientierung zum Zeitpunkt $t$ ist es m{\"{o}}glich die "Fahrtrichtung" zum
	Zeitpunkt $t$ zu bestimmen. Eine Multiplikation dieser mit den Geschwindigkeitsvektoren ergibt den
	Bewegungsvektor zum Zeitpunkt $t$ - daraus l{\"{a}}sst sich die Position ermitteln:
}
	x(t) &= x(0) + \frac{1}{2} \int_0^t \mathrm{\big(v_R(t) + v_L(t)\big) \times \cos\big(\Theta(t)\big)}\mathrm{d}t \label{eq:diffs-x}\\
	y(t) &= y(0) + \frac{1}{2} \int_0^t \mathrm{\big(v_R(t) - v_L(t)\big) \times \cos\big(\Theta(t)\big)}\mathrm{d}t \label{eq:diffs-y}
\end{align}
\end{subequations}

\subsection{Wegfindung}
Da ein \textit{differential steering} Roboter sich nicht-holonomisch (ohne Beschr{\"{a}}nung auf allen Achsen) bewegen kann, wird die Wegfindung
erschwert. Zur L{\"{o}}sung dieses Problems (auch bekannt als inverses kinematisches Problems) k{\"{o}}nnte der Roboter sich erst auf der Stelle
drehen, bis er das Ziel "anguckt" und sich dann gradlienig auf dieses zubewegen.

Falls einer der beiden Robotermotoren allerdings nicht vollst{\"{a}}ndig funktionst{\"{u}}chtig ist, ist dies nicht m{\"{o}}glich. Eine m{\"{o}}gliche
Alternative w{\"{a}}hre es eine Kurve zu fahren, dies muss allerdings -- abh{\"{a}}ngig von der zur{\"{u}}ckzulegenden Strecke -- nicht immer m{\"{o}}glich
sein.

\clearpage
\section{Die Gleichgewichte}
Die Platte soll sich im Gleichgewicht befinden, also sich nicht bewegen und nicht gekippt sein. 

\subsection{Statisches Gleichgewicht}
Die Platte ist im statischen Gleichgewicht wenn alle Kr{\"{a}}fte/Gewichte (der Objekte, also der F{\"{u}}llstation und aller Roboter) die auf sie wirken, sich zu Null aufaddieren.
Dabei werden die Objekte als Vektoren bestehend aus Gewicht und Abstand zum Mittelpunkt betrachtet.

Um das Kippen der Platte auf der links-rechts und vorne-hinten Achse getrennt angeben zu k{\"{o}}nnen, werden diese getrennt voneinander ausgerechnet. Dabei wird das Hebelgesetz
im eindimensionalen Fall (wie auf einer Wippe) angewandt, also Distanz mal Masse ergibt die Kraft in dieser Richtung
\footnote{Da die Positionsbestandteile (x und y) einzeln betrachtet werden, muss der Winkel nicht in die Rechnung einflie{\ss}en}.

F{\"{u}}r die in Links-Rechts wirkende Kr{\"{a}}fte gilt Gleichung ~\ref{eq:forces:x}, f{\"{u}}r die Kr{\"{a}}fte die beschreiben wie die Platte nach vorne-hinten gekippt
gilt Gleichung ~\ref{eq:forces:y}.
\begin{subequations}\label{eq:forces}
\begin{align}
	V(x) = x(F) \times w(F) + \sum_{i=0}^{|\gls{N}|} ( x(N_i) \times w(N_i) ) \label{eq:forces:x}\\
	V(y) = y(F) \times w(F) + \sum_{i=0}^{|N|} ( y(N_i) \times w(N_i) ) \label{eq:forces:y}
\end{align}
\end{subequations}

Das Ziel ist es die Platte auszubalancieren, also soll gelten:
\begin{equation}\label{eq:gleichgewicht}
	V(x) = V(y) = 0
\end{equation}

\subsection{Kippwinkel}
Diese Kr{\"{a}}fte f{\"{u}}hren dazu das die Platte kippt. Es ist definiert das der maximale Kippwinkel von 30\textdegree erreicht wird, falls sich alle
Objekte (also vollgetankte Roboter und Ladestation) am {\"{a}}ussersten Rand der Platte befinden. Um den momentanen Kippwinkel zu ermittelen werden nun
die momentan wirkenden Kr{\"{a}}fte mit den maximal m{\"{o}}glichen Kr{\"{a}}ften verrechnet; dies geschieht mit einem simplen Dreisatz:

\begin{equation}\label{eq:tilt-from-maxtilt}
30\degree = Max, X\degree = Curr \\
\Rightarrow \frac{Max}{Curr} = \frac{30}{X} \\
\Rightarrow \frac{1}{X} = \frac{Max}{30 \times Curr} \\
\Rightarrow X = \frac{30 \times Curr}{Max}
\end{equation}
\todo{nicht alles in eine zeile}

\subsection{Dynamischen Gleichgewicht}
Eine Geschwindigkeits{\"{a}}nderung der Roboter erzeugt einen Impuls, auch dieser wirkt auf die Platte. Die Energie des Impulses ist proportional zur
{\"{A}}nderung und schwingt dann langsam ab. Da ein geschlossenes System simuliert wird, ist dem Impuls eine gleich gro{\ss}e Kraft entgegen gerichtet.
Diese beiden Kr{\"{a}}fte f{\"{u}}hren zu einer (abklingenden) Schwingung. Mathematisch ausgedr{\"{u}}ckt:
\begin{equation}\label{eq:schwingung}
	s(t) = e^\mathrm{-\delta t} \times sin(\omega_d t) \times s(0)
\end{equation}
\cite{wiki:schwingung}
\footnote{$\delta$ und $\omega_d$ sind in der Simulation frei einstellbar}

Hierbei ist $s(t)$ ein Vektor, in dem die Geschwindigkeits{\"{a}}nderung in X-, Y- und Z-Richtung die jeweiligen Elemente sind. Um die gesamte Schwindung der
Platte zu bestimmen werden die Ergebnissvektoren der einzelnen Roboter aufaddiert.



\clearpage
\part{Implementierung}
\clearpage
\section{Architektur}
Das ganze System besteht aus verschiedenen Teilen (zum Beispiel der Simulationsserver, der Voter und die Controller) welche
auf bestimmten Arten miteinander kommunizieren und alle eine Funktion erf{\"{u}}llen.

\begin{figure}
	\centering
	\includevisio[width=\textwidth]{Komponenten2}
	\caption{Implementierte Systembestandteile und ihre Beziehungen}
	\label{fig:arch}
\end{figure}


Das Simulationsprogramm besteht aus mehreren Komponenten; deren Funktionalit{\"{a}}t wurde noch weiter in Klassen untergliedert. Allerdings
ist es auf dieser feineren Ebene teilweise sinnvoll Klassen zu bilden die logisch Teile von zweien der Komponenten abbilden. Eine andere
m{\"{o}}gliche Sicht auf das Simulationsprogramm ist daher die Einteilung in zwei Kategorien: als erstes die Klassen die f{\"{u}}r die
Simulation und das Netzwerk zust{\"{a}}ndig sind, als zweites die Klassen die den Benutzer mit dem System interagieren lassen.
\begin{figure}
	\centering
	\includevisio[width=\textwidth]{ClassDiagram}
	\caption{Klassendiagramm}
	\label{fig:uml}
\end{figure}

\subsection{Physiksimulation}
Der wichtigste Teil dieser Bachelorarbeit ist die Simulation. Sie soll nicht nur ein physikalisch glaubw{\"{u}}rdiges Modell
f{\"{u}}r Roboterbewegungen darstellen, den Votern die ben{\"{o}}tigten Informationen zur Steuerung bereitstellen und die
es erm{\"{o}}glichen das empfange Steuerkommandos ausf{\"{u}}hren werden. Dar{\"{u}}ber hinaus soll das Geschehen auch graphisch
dargestellt werden und dem Benutzer die M{\"{o}}glichkeit gegeben werden mit der Simulation zu interagieren, zum einen
durch Bewegen der Kamera, zum anderen durch eine Fernsteuerung eines Roboters (zwecks Fehlerinjektion). Auch diese
Anforderungen wirken sich auf die Implementierung der Physiksimulation aus. 

F{\"{u}}r das Kippen der Platte ist die Klasse PlatePhysics zust{\"{a}}ndig. Sie ist von MonoBehaviour abgeleitet und ist daher
in das Unity Eventsystem eingebunden --- dies erlaubt es innerhalb der Funktion LateUpdate() f{\"{u}}r jeden \textit{\gls{Frame}} den momentanen
Stand der Physiksimulation abzufragen und die Platte entsprechend zu rotieren / kippen.

Der Kippwinkel der Platte wird durch zwei Faktoren bestimmt, das statische und das dynamische Ungleichgewicht. Die Informationen {\"{u}}ber
das statische Gleichgewicht werden von DrawPressureVector() bestimmt. Das dynamische Gleichgewicht wird innerhalb PlatePhysics berechnet,
jeder RobotController {\"{u}}bergibt beim Ausf{\"{u}}hren einer Bewegung diese auch an PlatePhysics. Diese werden dann aufsummiert, ged{\"{a}}mpft
und schlie{\ss}lich zusammen mit dem statischen Gleichgewicht als Plattenrotation / -kippwinkel gesetzt.

Auch f{\"{u}} die Ladestation gibt es eine Klasse, diese erkennt Kollisionen mit anderen Objekten (also den Robotern) und l{\"{a}}dt diese auf
solange der Roboter sich an der Ladestation bedinet.

Eine weitere wichtige Klasse ist RobotController. Um die Roboter zu steueren (ob {\"{u}}ber die manuelle Fehlerinjektion oder mithilfe des
Controllers) und dabei a) das Fehlermodell und b) den physikalischen Status (Ladestatus, momentane Geschwindigkeit, ...) zu beachten. In der
Funktion FixedUpdate, die einmal pro \textit{Frame} aufgef{\"{u}}hrt wird, wird der Roboter bewegt. Dabei n{\"{a}}hrt sich der Roboter den
gew{\"{u}}nschten Beschleunigungen und Rotationen an.

\subsection{Anzeige}\label{graphics}
F{\"{u}}r interessierte Sch{\"{u}}ler und auch f{\"{u}}r die Studenten, die diese Aufgabe l{\"{o}}sen
sollen muss es m{\"{o}}glich sein einzusch{\"{a}}tzen wie der momentane Status der Simulation ist.
Also: wo befinden sich Roboter, wie voll sind sie, wie stark kippt die Platte? All dies soll auch, wie
in \hyperref[anforderung]{Anforderung 5} festgelegt, visuell ansprechend sein.

Da die Unity Engine auch eine vollwertige Grafikengine ist (dies war ja schliesslich ein Kriterium),
kann diese Funktionalit{\"{a}}t genutzt werden. Das Simulationsprogramm ist nicht nur f{\"{u}}r die
reine Simulation sondern auch f{\"{u}}r die Anzeige verantwortlich.

Angezeigt wird eine kreisf{\"{o}}rmige Platte auf der sich die verschiedenen Roboter bewegen. Auch die
Ladestation wird angezeigt.

\paragraph{CameraController} Die Sicht auf die simulierte Welt geschieht durch das Unity Objekt "Camera". An diese
werden verschiedene Skripte angegeh{\"{a}}ngt, um z.B. dem Benutzer die M{\"{o}}glichkeit zu
geben die Kamera zu bewegen und damit einen anderen Teil der Simulation genauer zu betrachten.

Die Kamera kann {\"{u}}ber die Pfeiltasten bewegt werden, ein Kippen der Kamera ist nicht implementiert --
dies w{\"{u}}rde auch verwirren, da damit eventuell der Kippwinkel der Platte visuell ausgeglichen
werden k{\"{o}}nnte. Um immer einen Teil der Simulation im Blick haben zu k{\"{o}}nnen schwebt die Kamera
{\"{u}}ber der Platte und schaut in einem 20\textdegree Winkel schr{\"{a}}g nach unten.

\paragraph{Lichtquelle} Damit etwas sichtbar wird, muss es von Licht getroffen werden. In der Simulation
gibt es daher ein Spotlight. Dies befindet sich weit von der Platte entfernt und scheint in einem steilen Winkel
von links oben auf diese.

\paragraph{Radar} Um einzusch{\"{a}}tzen wie stark die Platte gekippt wurde, k{\"{o}}nnen sich die Studenten einmal
innerhalb der Simulation die Platte anschauen. Um etwas genauer sehen zu k{\"{o}}nen wie die Roboter stehen m{\"{u}}ssten
m{\"{u}}ssten um die Platte ins statische Gleichgewicht zu bringen, wird der momentane Schwerpunkt der Platte visuell
dargestellt.

Dies geschieht durch die Klasse Radar. Sie zeichnet mehrere {\"{u}}berlappende Kreise in die linke, obere Bildschirmecke;
drei der Kreise (Gr{\"{u}}n, Gelb, Rot) symbolisieren die Platte. Auf diesen wird ein schwarzer Kreis gezeichnet, der den
momentanen Schwerpunkt angibt. Anhand der Farbe des Kreises auf welchen der momentane Schwerpunkt liegt ist es sehr einfach
abzulesen wie kritisch der Kippwinkel der Platte, und damit wie gut die Regelung, ist. Und anhand der zweidimensionalen Position
des Schwerpunktindikators k{\"{o}}nnen die Roboter bewegt werden.

\paragraph{FaultInjector\_Player} F{\"{u}}r die Studierenden soll es m{\"{o}}glich sein einen Roboter fernzusteuern um damit
gew{\"{u}}nschte Roboterkonstellationen gezielt herbeizuf{\"{u}}hren. Um anzuzeigen ob die manuelle Fehlerinktion gerade aktiv
oder inaktiv ist wird dem Benutzer, in der Mitte des Bildschirms, dies textuell Text angezeigt.

\subsection{Fehlerinjektion - Physik}
Da es in einer Simulation keine mechanischen (o.{\"{a}}.) Fehler geben kann, das Ziel des Simulationswerkzeuges aber gerade ist
die Studierenden ein fehlertoleranter Systeme entwickelen zu lassen braucht es eine Fehlerinjektion.

Die RobotController konsultieren bevor sie eine neue Zielbeschleunigung und -rotation setzen die Fehlerinjektion {\"{u}}ber den
Status ihres Roboters. Falls beispielsweise einer der Motoren nicht funktioniert hat das Auswirkungen auf die Zielbeschleunigung
(die Richtung und die L{\"{a}}nge des Beschleunigungsvektores werden ge{\"{a}}ndert). Die Informationen {\"{u}}ber den Status
jedes Motoren werden also von der Fehlerinjektionsklasse gespeichert und bereitgestellt.

\subsection{Interface - Simulationsprogramm}
Die Kommunikation zwischen Simulationsprogramm und Voter macht es n{\"{o}}tig das auf beiden Seiten eine Schnittstelle existiert
welche die Kommunikation abstrahiert. Auf der Seite des Simulationsprogrammes wird dies haupts{\"{a}}chlich durch die Klasse
HandleRPCRequests abgehandelt.

An den Voter muss in periodischen Abst{\"{a}}nden der Status der Platte und der sich auf ihr befindenen Roboter gesendet werden.
Daf{\"{u}}r gibt es den pubSock member, eine Instanz der Klasse PublishSocket aus der NNanoMsg Bibliothek. Die andere Aufgabe
dieser Klasse ist es Steuerbefehle der Controller anzunehmen und diese dann an die Roboter weiterzugeben. Daf{\"{u}}r gibt es
den repSock. Beide Aufgaben m{\"{u}}ssen immer wieder ausgef{\"{u}}hrt werden, anstatt einen neuen Thread zu starten werden
nicht-blockierende Funktionen benutzt und sich in das Unityevensystem eingeh{\"{a}}ngt.

\subsection{Interface - Voter}


\clearpage
\section{Fehlerinjektion}
Da es in der Simulation, anders in einer realen Welt, keine normalen Fehlerquellen gibt, wird eine
Fehlerinjektion benutzt, um die Fehlertoleranz zu testen. Die hier simulierten Fehlerarten 
sind im \hyperref[fm]{Fehlermodell} beschrieben.

In diesem Fall wird simuliert, dass das Netzwerk Daten fehlerhaft weiterleitet
(wie es z.B. ein Funknetzwerk tuen w{\"{u}}rde) und das die Controllerprogramme unvorhersehbar
ausfallen - was nachbildet, dass z.B. der Controller seine Batterie entladen hat. Auf der Roboterseite
kann die Funktionsf{\"{a}}higkeit der Motoren eingeschr{\"{a}}nkt werden oder sie k{\"{o}}nnen
falsche Informationen {\"{u}}ber den Status der Welt empfangen.

Wie in \hyperref[anforderung]{Anforderung 3} spezifiert, sind alle Fehlerwahrscheinlichkeiten einstellbar.

\subsection{Netzwerk}
Die Kommunikation zwischen Voter und Controller ist nur {\"{u}}ber ein IP Netzwerk m{\"{o}}glich;
f{\"{u}}r dieses Modul kann ein eigenes Netzwerk aufgebaut werden, in das eine Fehlerinjektionskomponente 
intergiert werden kann. Diese Komponente soll nur die Kommunikation, die von den Studentenprogrammen ausgeht, 
verf{\"{a}}lschen und alle andere normal behandeln. Die St{\"{a}}rke der Fehlerinjektion soll parametresierbar 
sein, zum Beispiel wie wahrscheinlich es ist, das, einzelne Pakete verf{\"{a}}lscht werden oder wie stark
die Verf{\"{a}}lschung sein soll.

Es wird vorgeschrieben, dass die ganze Kommunikation der Studentenprogramme {\"{u}}ber UDP laufen muss. Dadurch reicht es, wenn der Fehlerinjektor nur UDP verf{\"{a}}lscht, was dazu
f{\"{u}}hrt, dass die normalen Administrationstat{\"{a}}tigkeiten nicht beeintr{\"{a}}chtigt werden. Der Fehlerinjektor muss in der Lage sein, mit einstellbaren Wahrscheinlichkeiten
Pakete zu verf{\"{a}}lschen: bei den Arten der Verf{\"{a}}lschung ist es ausreichend, wenn einzelne Bytes gekippt werden.

Daf{\"{u}}r zust{\"{a}}ndig ist Net Inject\cite{kubertzki}. Dabei werden auf den fraglichen Rechnern
Standartrouten definiert, die den Netzwerkverkehr {\"{u}}ber andere Rechner leiten, auf denen Net Inject
installiert ist. Dort werden die Pakete vom Modul NetMod angenommen und es wird {\"{u}}berpr{\"{u}}ft ob
f{\"{u}}r diese Kommunikation (spezifiziert durch Protkoll, Quell- und Zielport) Verf{\"{a}}lschungsregeln
existieren. Anhand dieser Regeln wird das Paket dann ausgwertet und bei Bedarf verf{\"{a}}lscht.

Standardm{\"{a}}ssig ist dieser Fehlerinjektor so eingestellt dass durchschnittlich in $ ^1/_{20} $ aller Pakete ein Byte verf{\"{a}}lscht wird,
ohne dass das Betriebssystem diese Modifikation erkennt. Diese Erkennung muss also von den Studenten selbst implementiert werden.

\subsection{Controller}
Im Fehlermodell wurde spezifiziert, dass die Controller zu beliebigen Zeitpunkten einen
\textit{crash failure} erleiden k{\"{o}}nnen.

Um dies zu erm{\"{o}}glichen, gibt es ein Shellskript, das die Controllerprogramme zu zuf{\"{a}}ligen Zeitpunkten 
beendet und wieder startet - dabei wird das Programm einfach mit SIGKILL beendet und nicht vorgewarnt.
Mit einer Wahrscheinlichkeit von 33\% werden dem Programm auch nur eingeschr{\"{a}}nkte Ressourcen zugeteilt,
z.B. darf es nur eine bestimmte Anzahl an Dateideskriptoren gleichzeitig offen haben oder nur eine 
eingeschr{\"{a}}nkte Menge Speicher benutzen.
\lstinputlisting[language=Bash]{../fault_injector.sh}

\subsection{Simulation}
Auch innerhalb der Simulation soll es Fehler geben, zum Beispiel an den Robotermotoren. Desweiteren soll,
laut Fehlermodell, auch die Fernsteuerung eines Roboters m{\"{o}}glich sein.

Laut \hyperref[fm]{Fehlermodell} k{\"{o}}nnen die Robotermotoren auf verschiedene Arten fehlerbehaftet
sein und die Voter bekommen eventuell fehlerhafte, oder sogar gar keine, Informationen {\"{u}}ber
den Status der Welt. Die Wahrscheinlichkeiten f{\"{u}}r diese verschiedenen F{\"{a}}lle
werden in einer JSON Konfigurationsdatei angegeben. Die Simulation liest diese beim Start aus und 
l{\"{a}}sst dann w{\"{a}}hrend des laufenden Betriebes die Robotermotoren ausfallen.
Ein Teil der Informationen muss an den Voter weitergegeben werden, daher fragt dieser beim Start
automatisch die ben{\"{o}}tigten Daten ab.


\subsubsection{Weltstatus}
Die echten \hyperref[khepera]{Kheperaroboter} haben verschiedene Sensoren (Infrarot, Ultraschall,
odeometrische Motoren, eine Kamera, etc). In der echten Welt k{\"{o}}nnen auch diese Senoren fehlerhaft
sein. Allerdings werden in der Simulation diese Sensoren nicht simuliert, die Roboter kriegen alle
ben{\"{o}}tigten Informationen als Weltstatuspaket. Um die fehlerhafte Sensorik nachzuahmen, soll diese
Informationen abge{\"{a}}ndert werden.

Der Status der Welt der von der Simulation zum Voter geschickt wird kann, wie im \hyperref[fm]{Fehlermodell} angegeben, verf{\"{a}}scht werden, kann zeitweise gar
nicht ankommen oder es kann ein alter Status erneut versendet werden. Die Wahrscheinlichkeiten f{\"{u}}r diese drei F{\"{a}}lle werden direkt angegeben.
\begin{lstlisting}[frame=single, language=json] 
{
	"network" : {
		"dropWorldStatus": 0.001,
		"fakeWorldStatus": 0.001,
		"dupWorldStatus": 0.001
	}
}
\end{lstlisting}

\paragraph{dropWorldStatus} Der Status der Welt kann, aus Sicht des Roboters, einfach nicht ankommen. Dies
	ist ein \textit{omission fault}.
\paragraph{dupWorldStatus} Des weiteren kann ein veralteter Weltstatus ein weiteres Mal empfangen werden.
    Hierbei handelt es sich allerdings immer nur um den jeweils letzten und nicht um noch weiter in
	der Vergangenheit liegende.
\paragraph{fakeWorldStatus} Es ist auch m{\"{o}}glich, dass Daten zur richtigen Zeit ankommen, aber
    verf{\"{a}}lscht wurden. Verf{\"{a}}lscht werden kann hierbei:
	\begin{itemize}
		\item X- und Y-Kippwinkel. Dabei wird einer der Werte um maximal $\pm 2$\textdegree verf{\"{a}}lscht.
			Diese Art der Verf{\"{a}}lschung geschieht in jeweils 40\% der F{\"{a}}lle.
		\item Mit einer Wahrscheinlichkeit von 10\% werden die Daten maximal eines Roboters verf{\"{a}}lscht.
			Namentlich kann hierbei eine der folgenden Daten ver{\"{a}}ndert werden:
			\begin{itemize}
				\item X- oder Y-Koordinaten, um maximal $\pm 5$ Einheiten
				\item Die Rotation, um maximal $\pm 7$\textdegree
				\item Das Gewicht um maximal $\pm 5$ Einheiten
				\item Der Ladezustand um maximal $\pm 50$ Einheiten
			\end{itemize}
			Jede dieser Verf{\"{a}}lschungsarten ist gleich wahrscheinlich, sie liegen also bei 20\%.
		\item Die Daten der Ladestation, nach den gleichen Regeln wie f{\"{u}}r den Roboter
	\end{itemize}

\subsubsection{Motor}
Die zweite Kategorie von Fehlern sind die Ausf{\"{a}}lle der Motoren. Direkt angegeben werden die maximale
Anzahl an fehlerhaften Robotern und die maximale Anzahl an fehlerhaften Motoren pro Roboter.
Die einzelnen Motoren k{\"{o}}nnen entweder tempor{\"{a}}r\footnote{Tempor{\"{a}}r bedeutet
in diesem Fall bis zum n{\"{a}}chsten Steuerbefehl. Dieser wird nat{\"{u}}rlich wieder mit
gleicher Fehlerwahrscheinlich "fehlschlagen"} oder permament fehlerbehaftet (\textit{stuck-at}
oder keinerlei Leitung) sein. Die Wahrscheinlichkeiten f{\"{u}}r diese beiden Fehlerarten werden getrennt
angegeben und k{\"{o}}nnen entweder einer zeitunabh{\"{a}}nigen festen Wahrscheinlichkeit oder der
Badewannenkurve folgen. Bei der Badewannenkurve wird zus{\"{a}}tzlich zur Ausfallwahrscheinlichkeit auch
noch die Dauer (in s) des ersten Bereiches angegeben.
\begin{lstlisting}[frame=single, language=json] 
{
	"robot" : {
		"breakEngineA" : {
			"perm": {"after": 120, "chance" : 0.01}
		},
		"stuckAtEngineA" : {
			"temp": 0.00001
		},
		"maxEnginesBroken": 3,
		"maxEnginesBrokenPerRobot": 1
	}
}
\end{lstlisting}


\subsubsection{Fernsteuerung}
Wie im \hyperref[fm]{Fehlermodell} festgelegt, soll der Benutzer einen Roboter fernsteuern k{\"{o}}nnen
und damit direkten Einfluss auf die simulierte Welt haben. W{\"{a}}hrend ein Roboter ferngesteuert wird, 
f{\"{u}}hrt dieser eine Roboter keine Befehle seines Voters mehr aus. Abgesehen davon verh{\"{a}}lt sich der
Roboter normal, es wird also die gleiche Menge Energie verbraucht, es gelten die gleichen
Einschr{\"{a}}nkungen bei Geschwindigkeit und Rotation, und so weiter.

Dies wird durch ein Script in Unity erm{\"{o}}glicht. Im deaktivierten Zustand zeigt es einen Hilfetext
auf dem Bildschirm an. Mit der Escapetaste wird die Fernsteuerung aktiviert; um dies zu signalisieren wird
der Hilfetext ge{\"{a}}ndert. Dann kann ein Roboter {\"{u}}ber die Pfeiltasten gesteuert werden
(Vorw{\"{a}}rts beschleunigt, Links und Rechts drehen den Roboter). Um anzuzeigen, welcher Roboter
gerade gesteuert wird wird dieser von oben angeleuchtet. Mit der Tabulatortaste kann zwischen den
Robotern umgeschaltet werden. Bei einem weiteren Druck auf die Escapetaste wird die Fernsteuerung deaktiviert.


\clearpage
\section{Die Simulation}
Die simulierte Welt besteht aus den Robotern, die von den Studierenden gesteuert werden sollen, einer Ladestation, an der die Robter Energie tanken k{\"{o}}nnen, und
der Welt, einer kreisf{\"{o}}rmigen, kippbaren Platten, auf der diese Objekte platziert werden und sich bewegen k{\"{o}}nnen.

Simuliert wird die Welt mit der Unity \textit{game engine}. Diese erm{\"{o}}glicht es plattformunabh{\"{a}}nige 
Spiele oder, in diesem Fall, Simulationen zu schreiben. Dabei stellt sie, unter anderem eine Physikengine,
eine Grafikengine und eine Schnitstelle zum Scripten dieser bereit. Desweiteren gibt es auch M{\"{o}}glichkeiten Benutzereingaben
abzufragen, um dem Benutzer die M{\"{o}}glichkeit zu geben mit der Simulation zu interagieren.

Hier werden die M{\"{o}}glichkeiten der objektorierntierten Programmierung genutzt. Jeder Bestandteil der Simulation wird durch
ein Skript abgebildet, diese Skripte k{\"{o}}nnen auch miteinander interagieren. Zum Beispiel benutzt die Robotersteuerungsklasse
die RPC Klasse, so wie dies auch die Fehlerinjektion tut. Genauso wurde die Vererbung genutzt, denn die Roboter und Ladestation haben
einige Eigenschaften gemeinsam (zum Beispiel Position und Gewicht), die eine Vererbungshierachie erm{\"{o}}glichen.

\subsection{Die Roboter}\label{robot}
In der simulierten Welt k{\"{o}}nnen sich bis zu \gls{N} Roboter bewegen. Diese bewegen sich aber nicht selbstst{\"{a}}ndig, sondern werden von den Controllern ferngesteurt.
Wie sie in der Simulation dargestellt werden, wird durch das grafische Modell bestimmt. Anhand dessen bestimmen sich auch die Dimensionen, diese werden f{\"{u}}r die Kollisionerkennung
gebraucht. Die Dimensionen, zusammen mit der Masse, ergeben das physische Modell; dieses hat Auswirkungen auf die Simulation.

\paragraph{Grafisches Modell} Mithilfe von Blender, einem 3D Designprogramm, wurde ein Robotermodell designt, das dem Kepheraroboter\hyperref{khepera} entspricht. Die Grundform des Roboters ist
eine S{\"{a}}ule. Auf dieser befindet sich eine Lampe die den Energielevel angezeigt; daf{\"{u}} {\"{a}}ndert sich ihre Farbe von Gr{\"{u}}n (voll), {\"{u}}ber Gelb bis Rot (leer) .
\todo{Bild}


\paragraph{Physikalisches Modell}
Ein Roboter \gls{Ni} wird dabei beschrieben durch seine Position und Gewicht
$ N_i = \bigl(\begin{smallmatrix} x(i) \\ y(i) \\ w(i) \end{smallmatrix}\bigr)$, eine
Geschwindigkeit $ V_i = \Delta v $ und den momentanen Drehwinkel
$ R_i = r_y(i)$. \todo{Ausdehnung}

Das Gewicht des Roboters ist abh{\"{a}}ngig vom Grundgewicht des Roboters und seinem momentanen F{\"{u}}llstatus: 
\begin{equation}\label{eq:w}
 w(N_i) = 1 + e(N_i) \times 0.03
\end{equation}

Die Roboter haben einen Energiespeicher, der mit maximal 1000 Energieeinheiten
aufgeladen werden kann, und verbrauchen diese Energie, ob beim Fahren oder
Stillstand. Dabei verbrauchen sie pro Sekunde immer eine Energieeinheit und zus{\"{a}}tzlich, abh{\"{a}}ngig von der Geschwindigkeit, Energie f{\"{u}}r die Bewegung:
\begin{equation}\label{eq:entladen}
	e(N_i, t + 1) = e(N_i, t) - 1 - |V_i|
\end{equation}
Falls nicht gen{\"{u}}gend Energie f{\"{u}}r Bewegung und Rotation vorhanden ist, bewegt sich der Roboter nicht.

Die Bewegung des Roboters wird vorgegeben durch $V_l(t)$ und $ V_r(t)$. Diese werden vom Controller, {\"{u}}ber den Voter an die Simulation
weitergegeben und dann von Unity verarbeitet. Die neue Position und der neue Rotationswinkel des Roboters werden errechnet und,
falls gen{\"{u}}gen Energie vorhanden, auch eingenommen - dazu wird der \textit{rigidbody} des Roboters manipuliert.
Um die Bewegung fl{\"{u}}{\ss}iger darzustellen wird zwischen den momentanen und gew{\"{u}}nschten Positionen / Winkel linear interpoliert.

Hierbei muss auch die Fehlerinjektion beachtet werden. Es ist m{\"{o}}glich das die gew{\"{u}}nschte Drehzahl eines Motors nicht eingenommen werden
kann, entweder weil dieser momentan ein \textit{stuck-at}-Verhalten aufweist, oder kaputt (definiert als: Drehzahl = 0) ist. Daf{\"{u}}r wird bevor
die Bewegung durchgef{\"{u}}hrt wird {\"{u}}berpr{\"{u}}ft ob einer der beiden Motoren momentan nicht korrekt funktioniert; ist dies der Fall so wird,
statt des gew{\"{u}}nschten Sollwertes, diesem Motor der Sollwert der Fehlerinjektion zugewiesen. Mit diesem Wert wird nun normal weitergerechnet, also
erst die {\"{U}}berpr{\"{u}}fung ob der Roboter sich von der verbleibenden Energiemenge her so bewegen kann, und -- falls dies der Fall ist -- wird nun
nach und nach diese Drehzahl der Motoren eingenommen was zu einer Rotation und Translation des Roboters f{\"{u}}hrt.

\subsection{Die Ladestation}\label{fuelstation}
Innerhalb der Welt muss eine Ladestation platziert werden, um den Roboter die M{\"{o}}glichkeit zu geben sich aufzuladen. Auch diese wird durch ihren Vektor $ F = \bigl(\begin{smallmatrix} x \\ y \\ w \end{smallmatrix}\bigr)$ beschrieben. Eine Ladestation hat dabei ein festes Gewicht: $ w(F) = 5 $.

Diese wird vor Simulationsbeginn platziert und bewegt sich im weiteren Verlauf nicht.
Falls sich ein Roboter an die Ladestation heranbewegt, also gilt: 
\begin{equation}\label{eq:dist}
 |\bigl(\begin{smallmatrix} x(i) \\ y(i) \end{smallmatrix}\bigr) - \bigl(\begin{smallmatrix} x(F) \\ y(F) \end{smallmatrix}\bigr)| \leq |\bigl(\begin{smallmatrix} 1 \\ 1 \end{smallmatrix}\bigr)|
\end{equation}
wird dieser Roboter aufgeladen. Die Ladefunktion ~\ref{eq:laden} ist hier eine einfache Gerade:
\begin{equation}
    \label{eq:laden}
	e(N_i, t + 1) = max((e(N_i, t) + 10, 1000) 
\end{equation}

\subsection{Die Platte}\label{plate}
Die simulierte Welt besteht aus einer 100 Einheiten gro{\ss}en kreisf{\"{o}}rmigen Platte, die, basierend auf den Gewichten welche sich auf ihr befinden kippt.

Die Kr{\"{a}}fte die auf die Platte wirken berechnen sich nach den Gleichungen \ref{eq:diffs-x}, \ref{eq:diffs-y}. Es wird definiert das die Platte um
maximal 10\textdegree kippen kann; dies sei der Fall wenn die Ladestation und alle Roboter (vollgetankt) am {\"{a}}u{\ss}ersten Rand stehen. Diese theoretisch
m{\"{o}}gliche Maximalkraft entspricht also den 10\textdegree; mithilfe des Dreisatzes ist es nun m{\"{o}}glich die tats{\"{a}}chlichen momentanen Kippwinkel
zu errechnen.

Allerdings wird diese Rechnung erst ausgef{\"{u}}hrt wenn sich 2 (oder mehr) Roboter auf der Platte befinden. Vorher wirkt das Gewicht der Ladestation zu
stark und die Platte ist schon beim Start so stark gekippt das sie nicht mehr ausbalanciert werden kann.

Die Impulse welche auf die Platte wirken (beschrieben in \ref{eq:schwingung}) f{\"{u}}hren zu einer zus{\"{a}}tzlichen Kippung und einer Rotation der Platte.
Immer wenn ein Roboter sich bewegt, wird dieser Bewegungsimpuls gespeichert. Um die momentan auf die Platte wirkenden Kr{\"{a}}fte zu errechnen, werden alle
Impulse aufaddiert. Da der Impuls mit der Zeit abschw{\"{a}}cht wird bei jedem Frame aus dem Anfangsimpuls und der seit dem vergangen Zeit der momentane
Impuls errechnet -- wenn ein Impuls irgendwann keinen sp{\"{u}}rbaren Einfluss auf die Rechnung hat ($ |s(t)| < 0.01$) f{\"{a}}llt dieser Impuls aus der
Rechnung heraus.

Um die Platte dann tats{\"{a}}chlich zu kippen wird die Unity Funktion `AddTorque()` benutzt. Dieser wird ein Kraftvektor {\"{u}}bergeben, um sich also einem
bestimmtem Kippwinkel anzun{\"{a}}hren wird erst die Differenz zwischen den momentanen Winkeln und den gew{\"{u}}nschten errechnet und dann eventuell auf
einen Maximalwert heruntergesetzt (damit die Platte sich langsam zu den gew{\"{u}}nschten Winkeln hinbewegt und nicht direkt "springt"); Dieser Wert kann dann
auf die Platte wirken und rotiert sie in die gew{\"{u}}nschte Richtung.

\subsection{Netzwerkschnittstelle}
Zur Kommunikation mit den Studentenprogrammen gibt es die Netzwerkschnittstelle. Hierr{\"{u}}ber bekommen die Voter die Informationen {\"{u}}ber den momentanen
Status der Welt und k{\"{o}}nnen die Roboter bewegen (nachdem sie einen Roboter erstellt haben). Die Kommunikation geschieht {\"{u}}ber ein selbstentwickeltes
RPC Protokoll. Dieses erlaubt es verschiedene Funktion (aufgef{\"{u}}hrt in \ref{interface}) aufzurufen und die Weltstatusinformationen zu empfangen.

Haupts{\"{a}}chlich verantwortlich hierf{\"{u}}r ist die Klasse HandleRPCRequest. Bei der Initalisierung erstellt sie zwei \textit{sockets}, einen der f{\"{u}}r
die Weltstatusinformationen nach dem \textit{publish-suscribe} Muster sendet (TCP, Port 8001) und einen der die RPC Anfragen bearbeitet (TCP, Port 8000).

In der Update() Funktion werden die Funktion f{\"{u}}r das abarbeiten der RPC Anfragen und das senden der Weltstatusinformationen aufgerufen. In handleRPC()
wird versucht vom socket zu lesen. Dies geschieht nicht-blockierend, so dass, falls keine Daten gelesen werden konnten die Funktion direkt einen Fehler
zur{\"{u}}ckgibt. Ist dies der Fall, passiert in handleRPC() nichts weiter. Sind allerdings doch Daten empfangen worden, werden sie zu allererst deserialisiert.
Danach kann das RPC Paket ausgewertet werden; je nach dem welche Funktion aufgerufen werden soll, verzweigt sich die Ausf{\"{u}}hrung.

Wenn sich ein Voter mit der Simulation verbindet werden als aller erstes die Fehlerinjektionskonfigurationen abgefragt -- dies geschieht durch einen RPC Aufruf
von GET\_FAULT\_INJECTOR\_NETWORK\_CFG. In diesem Fall serialisiert die Simulation diese Daten und schickt sie an den Voter zur{\"{u}}ck.

Als n{\"{a}}chstes erstellt ein Voter einen Roboter, durch Aufruf von CREATE\_ROBOT. In diesem Fall wird aus dem Roboter \textit{prefab} ein Roboter an einer
zuf{\"{a}}lligen Positionen erstellt. Von diesem wird dann die Objektid genommen und an den Voter gesendet. Dieser kann diese nun benutzen um den Roboter zu
steuern.

Die Steuerung eines Roboters geschieht durch einen Aufruf von MOVE\_ROBOT. Dann wird im RoboterController die Funktion MoveRobot() aufgerufen, in der die
Bewegung (unter Beachtung des Ladestands und der Fehlerinjektion) ausgef{\"{u}}hrt wird. Diese RPC Funktion gibt nichts an den Aufrufer zur{\"{u}}ck, Erfolg
oder Misserfolg des Steuerbefehls muss anhand der Weltstatusinformationen erkannt werden.

Wie oft die Weltstatusinformationen (von der Funktion sendWorldStatus()) an alle Subscriber gesendet wird, ist {\"{u}}ber die Konfigurationdateien
einstellbar. Eine genauere Beschreibung des Weltstatus findet sich in \ref{worldstatus}.

\clearpage
\section{Interface f{\"{u}}r die Studenten}\label{interface}
Damit die Studenten sich auf die Implementierung der Fehlertoleranz konzentrieren k{\"{o}}nnen, gibt es Schnittstellen.
Im ganzen System gibt es zwei Schnittstellen:
\begin{itemize}
\item Die Schnittstelle zwischen Controller und Voter
\item Die Schnittstelle zwischen Voter und Roboter/Simulation
\end{itemize}

\paragraph{Die Schnittstelle zwischen Controller und Voter} Den Studenten wird nicht vorgegeben wie die Kommunikation zwischen den Controllern und Votern aussehen soll - denn gerade hier soll ja die Fehlertoleranz implementiert werden.

\paragraph{Die Schnittstelle zwischen Voter und Roboter} Diese Schnittstelle besteht aus den Funktionen:
\begin{lstlisting}[frame=single, language=c] 
void* connectToWorld();
void detachFromWorld(void* ctx);
int createRobot(void* ctx);
typedef void (*TypeGetWorldStatusCallback)(WorldStatus ws, void* optional);
int startProcessingWorldEvents(void* ctx, TypeGetWorldStatusCallback cb, void* optional);
void moveRobot(void* ctx, int id, float vL, float vR);
\end{lstlisting}

Diese wird den Studierenden als kompilierte Library mit einem detailliert kommentierten Headerfile zur Verf{\"{u}}gung gestellt und kann
dann vom Studentencode einfach aufgerufen werden.

Die Weltstatusinformation beschreibt die beiden Kippwinkel der Platte und die Objekte auf dieser. Die Objekte (also Roboter und Ladestation) werden hierbei beschrieben
durch ihre Position, ihre Rotation und Masse. Desweiteren gibt es auch eine eindeutigen Bezeichner und eine Ladestandsanzeige (diese ist allerdings nur bei Robotern
g{\"{u}}ltig).
\label{worldstatus} 
\begin{lstlisting}[frame=single, language=c]
enum SimulationObjectType {
	ROBOT,
	FUEL_STATION
};
typedef struct SimulationObject {
	enum SimulationObjectType type; //! Either "ROBOT" or "FUELSTATION"

	float x; //! x position in the world
	float y; //! y position
	float rotation; //! rotation. 0 < rotation < 360
	float m; //! mass of object

	int id; //! id of this object. Use that for MoveRobot() calls

	int fuel; //! how much fuel that robot has left. invalid for a
		  //! fuel station
} SimulationObject;
\end{lstlisting}

\clearpage
\section{Beispielimplementation}

\todo{20 ist okay, man könnte am Anfang aber nochmal den Bogen zum großen
Bild/Ziel schliessen}

Um das Prinzip dieser Simulation, ob den Studierenden des Moduls Ausfallsichere Systeme oder Besuchern, zu verdeutlichen, ist ein Teil der Bachelorarbeit die Implementierung einer
Beispielimplementation.

\paragraph{Ablauf} Jede Viertelsekunde sendet die Welt Statusinformationen aus. Diese Informationen werden von den Votern empfangen, k{\"{o}}nnen
aber auf dem {\"{U}}bertragungsweg zum Controller potenziel verf{\"{a}}lscht worden sein. Also wird f{\"{u}}r jedes Objekt der Welt der Konsensalgorithmus ausgef{\"{u}}hrt. Wenn ein Konsens {\"{u}ber den Status der Welt hergestellt wurde, berechnet 
jeder Controller f{\"{u}}r jeden Roboter die n{\"{a}}chste Bewegung. Diese wird dann an den Voter geschickt, der ein einfaches Mehrheitsvotum durchf{\"{u}}hrt und diese Bewegung an die Welt weitergibt.

\subsection{Fehlermodell} \label{error-model}
Bei der Planung eines ausfallsicheren Systems ist es besonders wichtig zu definieren, welche Art von Fehlern
{\"{u}}berhaupt korrigiert / abgefangen werden soll. F{\"{u}}r die Beispielimplementation ist es nicht
n{\"{o}}tig eine Regelung zur Ausbalancierung zu entwickelen. Daher werden nur f{\"{u}}r bestimmte Fehlerklassen
die m{\"{o}}glichen Fehler aufgelistet und beschrieben, ob und wie sie gel{\"{o}}st werden.

\paragraph{Crash failure} Die Controller k{\"{o}}nnen jederzeit ausfallen. Im Extremfall k{\"{o}}nen alle Controller ausfallen, die Voter sind, per Definition, gegen diese Art von Fehlern unempfindlich.

\paragraph{Value error} Das Netzwerk kann Pakete verf{\"{a}}lschen, es wird angenommen, dass bei bis zu
$^1/_{20}$ aller Pakete Verf{\"{a}}lschungen geben kann, diese sich allerdings auf
\textit{Single Byte Errors} beschr{\"{a}}nkt. Dar{\"{u}}ber hinausgehende Verf{\"{a}}lschungen werden nicht
erkannt und f{\"{u}}hren zu einem \textit{silent failure}; Dadurch wird der Roboter fehlerhaft angesteuert,
dies ist allerdings nicht mehr Teil des Fehlermodells.

\subsubsection{R{\"{a}}umliche Redudanz}
Es wird davon ausgegangen, dass die Controller sehr fehleranf{\"{a}}llig sind und leicht ausfallen --
daraus folgt das eine gro{\ss}e Anzahl ($Y$) an Controllern f{\"{u}}r jeden Roboter kontrollieren muss;
nur falls weniger als $X$ Controller dieses Roboter noch aktiv sind kann eine ordnungsgem{\"{a}}{\ss}e
Steuerung nicht mehr garantiert werden. Zu bestimmen wie gro{\ss} $X$ sein muss ist nicht Teil der
Beispielimplementation.

\subsubsection{Netzwerkkommunikation}
Da durch die Fehlerinjektion das Netzwerk UDP Pakete verf{\"{a}}lscht, m{\"{u}}ssen alle Daten mit einer 
Kanalkodierung versehen werden. Hier wird ein (255, 240) Reed-Solomon Code benutzt, also ein Code, der 15 parity 
bits pro 240 Datenbits benutzt. Durch die Benutzung dieser Kodierung k{\"{o}}nnen alle Einzelfehler und 
Doppelfehler erkannt und korrigiert werden. Erst ab 7 Fehlern ist eine Korrektur nicht mehr m{\"{o}}glich, ab 15 
Fehlern versagt auch eine Fehlererkennung.

Diese Kodierungsart wurde aus zwei Gr{\"{u}}nden gew{\"{a}}hlt:
\begin{itemize}
\item Durch expermientelle Verifikation wurde klar, dass im Netzwerk nur 1 Byte pro Paket verf{\"{a}}lscht wird.
	Da das durchschnittliche Paket um mindestens Faktor 100 gr{\"{o}}{\ss}er ist, ist nicht erforderlich,
	eine sehr kompakte Kodierung zu benutzen, es ist ausreichend, die Redudanz zu reduzieren. 
\item Es ist ein systematischer Code, der es erlaubt, w{\"{a}}hrend des laufenden Betriebes die Pakete 
	mitzuschneiden und sich die Daten anzugucken. Dies vereinfacht die Fehlersuche -- beispielsweise bei
	einem Viterbicode w{\"{a}}re dies nicht m{\"{o}}glich, dort sind Nutz- und Kodierungsdaten nicht klar 
	unterscheidbar.
\end{itemize}

\subsection{Voter}
\label{voter}
Der Roboter kann nur direkt durch den Voter gesteuert werden. Der Voter ist dabei nur daf{\"{u}}r
zust{\"{a}}ndig, aus den vielen Steuerkommandos, die ankommen, die Mehrheit zu bilden (\textit{N-modular reduancy}).
Da nicht klar ist, wie viele Controller momentan {\"{u}}berhaupt Steuerkommandos senden k{\"{o}}nnten,
kann nicht gewartet werden bis eine bestimmte Anzahl von Kommandos empfangen wurde. Deswegen wird beim Empfangen
eines neuen Weltstatus (also zu periodisch wiederkehrenden Zeitpunkten) das Steuerergebniss gebildet,
weggesendet und der Vorgang wird neu gestartet.

Um das Steuerergebniss zu finden werden die empfangenen Steuerkommandos sortiert und der Medianwert genommen 
(\textit{Mid-Value Selection}). Anstatt des Mittelwerts wird der Medians genutzt weil dadurch weniger "extreme"
Bewegungen bevorzugt werden.
\noindent\begin{minipage}{.30\textwidth}
	\begin{lstlisting}[caption=Sammeln, frame=tlrb, language={[11]C++}]
votes.res[id].push_back(Vector{x, y});
\end{lstlisting}
\end{minipage}\hfill
\begin{minipage}{.60\textwidth}
\begin{lstlisting}[caption=Auswahl, frame=tlrb, language={[11]C++}]
/* go through all robots */
for(auto&& votes : votes.res) {
  /* vote */
  std::sort(std::begin(votes.second), std::end(votes.second));
  auto x = votes.second[votes.second.size() / 2];

  /* send */
  int r = 0;
  if((r = moveRobot(info->worldCtx, votes.first, x.x_, x.y_)) < 0) {
  	fprintf(stderr, "can't move robot: %d", r);
  }
}
\end{lstlisting}
\end{minipage}


\begin{figure}
	\centering
	\includevisio[width=\textwidth]{seqvoter}
	\caption{Ablaufdiagramm Voter}
	\label{fig:sequence-voter}
\end{figure}
\clearpage % make sure the table is, at least, in the right section

\subsection{Controller}\label{controller}
Die Roboter m{\"{u}}ssen sich so bewegen, dass die Platte m{\"{o}}glichst gut ausbalanciert
ist und gleichzeitig die Roboter nicht ihre ganze Tankf{\"{u}}llung verbrauchen. Gleichzeitig sollten
die Roboter auch nicht miteinander kollidieren.

\paragraph{Algorithmus zur Bestimmung der Bewegung} Der implementierte Algorithmus unterschiedet zwischen 3 Situation: ist der Zustand des Roboters kritisch,
unkritisch oder dazwischen? Dabei ist das Kriterium die momentane Tankf{\"{u}}llung geteilt durch die Entfernung zur Ladestation. Roboter mit fast leerem Tank
versuchen also sich aufzuf{\"{u}}llen, Roboter mit fast vollem Tank versuchen auszubalancieren.

Jeder Controller berechnet f{\"{u}}r jeden Roboter wie dieser sich bewegen soll. Dabei wird mit dem Roboter
im kritischtesten Zustand anfangen, denn dieser muss ja unbedingt Richtung Ladestation gesteuert werden 
und die weniger kritischen k{\"{o}}nnen/m{\"{u}}ssen diese Bewegung dann ausgleichen. Daf{\"{u}}r werden nach
jeder berechneten Roboterbewegung die Weltkippwinkel neu berechnet, so dass die vollsten Roboter die Bewegungen
der vorher berechneten mit in ihre Entscheidung einbeziehen.


Um nun zu einem Punkt zu fahren wird sich anhand des Drehwinkels gedreht und entsprechend des Distanzvektores 
bewegt. Rotation und Translation werden nacheinander ausgef{\"{u}}hrt. Dies ist zwar langsamer, erleichtert die
Implementierung aber ungemein.

\paragraph{Ausbalancieren} Auch das ausbalancieren kann als Bewegung zu einem Punkt aufgefasst werden.
Die Platte ist ja durch die Gewichte der Roboter aus dem Gleichgewicht gebracht worden, um sie wieder ins
Gleichgewicht zu bringen k{\"{o}}nnten sich die einzelnen Roboter ja an den jeweiligen Punkt bewegen der
die Platte ins Gleichgewicht bringen w{\"{u}}rde.

Dazu wird zuerst der Kippwinkel der Platte ohne den momentan betrachteten Roboter berechnet. Anhand dessen
kann bestimmt werden an welcher Position der momentan betrachtete Roboter sein m{\"{u}}sste um die Platte
ins Lot zu bringen. Dorthin bewegt der Roboter sich nun, stoppt allerdings am Rande der Platte um nicht
herunterzufallen -- schlie{\ss}lich kann es sein das dieser Roboter sich von der Platte bewegen m{\"{u}}sste
um die Platte auszubalancieren.

\clearpage
\section{Zusammenfassung}
Das Ziel dieser Bachelorarbeit war die Entwicklung einer {\"{U}}bungsaufgabe um Studierenden der Fachhochschule
S{\"{u}}dwestfalen die Grundz{\"{u}}ge der Ausfallsicheren Systeme beizubringen. Dabei soll die schon bestehende
{\"{U}}bungsaufgabe\ref{heizung} erg{\"{a}}nzt werden.

Dabei besteht das System aus mehreren Teilen. Zum einen das Simulationsprogramm das gleichzeitig auch als
Server f{\"{u}}r die, von den Studenten zu programmierenden, Roboter fungiert. Die Roboter bestehen aus
einem von den Studierenden programmierten Teil; dieser benutzt die libworld um sich mit der Simulation zu
verbinden und die Roboter in ihr zu steuern.

Da das Hauptziel des Simulationswerkzeuges war den Studenten eine weitere {\"{U}}bungsaufgabe zur Verf{\"{u}}gung
zu stellen, ist das Hauptbewertungskriterium wie gut diese Simulation f{\"{u}}r diesen Zweck geeignet ist.

Die anderen Kriterien sind: \todo{}

\begin{enumerate}
	\item Die Alternativaufgabe soll sich stark von der ausfallsicheren Heizung unterscheiden, damit die Studierenden eine tats{\"{a}}chliche Wahl haben.
	\item Um die Aufgabe zu l{\"{o}}sen, muss eine Vielzahl von Konzepten der Ausfallsicherheit genutzt werden.
	\item Es soll visuell ansprechend sein.
	\item Es darf auch durch unsachgem{\"{a}}ssen Gebrauch nicht gesch{\"{a}}digt werden
\end{enumerate}

Es ist allerdings m{\"{o}}glich auszuwerten ob die anderen Anforderungen erf{\"{u}}llt worden sind. Als erstes
soll hier die M{\"{o}}glichkeit betrachtet werden (im Gegensatz zur Heizung) das mehrere Gruppen gleichzeitig
an ihrer L{\"{o}}sung arbeiten k{\"{o}}nnen. Durch die Nutzung eines Servers zu dem sich die Voter/Roboter
verbinden zusammen mit ein wenig Absprache der Gruppe kann dieses Ziel erreicht werden. Die Simulation welche
auf diesem Server l{\"{a}}uft liest beim Start eine Konfigurationdatei ein, mit der verschiedene Einstellungen
der Fehlerinjektion frei gew{\"{a}}hlt werden k{\"{o}}nnen.

Mithilfe der Beispielimplementation (und einigen speziellen Testprogrammen) ist es m{\"{o}}glich einen Systemtest
durchzuf{\"{u}}hren. Bei einem Systemtest werden die Komponenten nicht seperat voneinander getestet, sondern das
gesamte System wird gleichzeitig getestet. Diese Art von Tests verifiziert das Zusammenspiel der Komponenten.

Die Aufgabe welche von den Studierenden zu bew{\"{a}}ltigen ist, ist es die Platte so gut wie m{\"{o}}glich zu
stabilisieren und dabei m{\"{o}}gliche Fehler zu erkennen und so gut wie m{\"{o}}glich zu beheben. Es muss also
ein physikalisches und ein informationstechnologisches Problem gel{\"{o}}st werden. Daher sollte das System in
Bezug auf die Anforderungen im Hinblick auf die Korrektheit der physikalischen Simulation und der korrekten
Funktionalit{\"{a}}t der Fehlerinjektion getestet werden.

\subsection{Evaluierung in Bezug auf Lernziel}
Wie sehr diese Bachelorarbeit als Lehrmittel zeigt, kann noch nicht evauliert werden. Dies wird erst w{\"{a}}hrend
der Benutzung durch die Studenten ersichtlich.

\subsection{Evaluierung in Bezug auf Simulation - Physik}
Die Physiksimulation simuliert Roboterbewegungen, das Entladen der Roboter und das Kippen der Platte. Diese k{\"{o}}nnen
seperat getestet werden.

Ein Testfall ist ob die Bewegungen des Roboters zu einem Entladen f{\"{u}}hren. Daf{\"{u}}r wurde ein Testprogramm geschrieben
das den Roboter mit einer festen Geschwindigkeit im Kreis fahren l{\"{a}}sst und sich beendet sobald der Roboter sich entladen
hat. Anhand der Durchlaufzeit des Programmes und der vorgegebenen Geschwindigkeit (und damit des erwarteten Verbrauches) ist
es m{\"{o}}glich den tats{\"{a}}chlichen Verbrauch herauszufinden.
\todo{durchfuehrung. am coolsten waere ein balkendiagramm, x achse sind 3 Geschwindigkeit, dann jeweils ein Balken fuer errechnet
und gemessen}

Da in der Simulation die Roboter sich nicht nur entladen sollen, sondern m{\"{o}}glichst lange fahren sollen gibt es eine
Ladestation welche die Roboter aufl{\"{a}}dt falls diese nah genug an dieser sind. Es ist m{\"{o}}glich ein Testprogramm zu
schreiben welches den Roboter an die Ladestation heranf{\"{a}}hrt und dort stehen bleibt. Die Roboter haben einen bestimmten
Grundumsatz der nach \todo{} dazu f{\"{u}}hren w{\"{u}}rde das der Roboter seine Energie verbraucht hat. Falls der Roboter allerdings
konstant aufgeladen wird, sollte dieser Roboter sich auch nach viel l{\"{a}}ngere Zeitspannen nicht entleeren.
\begin{subequations}\label{eq:eval:laden}
\begin{align}
 e_i(t = 0) = 1000 - 1 * 0
 \intertext{Bei einem Grundumsatz von 1 pro Sekunde}
	e_i(t = X) &= 1000 - 1 * X  = 0
	\implies e_i(t = 1000) &= 0
\end{align}
\end{subequations}


\todo{durchfuerung, vlt einfach diagramm mit fuellstatus ueber zeit?}

\subsection{Evaluierung in Bezug auf Simulation - Fehlerinjektion}
Da die Fehlerinjektion die Fehler zuf{\"{a}}llig injeziert ist eine statistische Auswertung notwendig. Das hei{\ss}t praktisch das
der gleiche Versuch mehrmals ausgef{\"{u}}hrt werden muss und der Anteil der "gegl{\"{u}}ckten" Versuche mit dem Erwartungswert
verglichen wird.

F{\"{u}}r den ersten Test wird die Fehlerinjektionswahrscheinlichkeit f{\"{u}}r den Motor A einmal auf ann{\"{a}}hrend null und einmal
auf ann{\"{a}}hrend eins gesetzt. Dann wird die durchschnittliche Zeit verglichen die der Roboter braucht um eine Strecke mit fester
L{\"{a}}nge abzufahren --- falls die Fehlerinjektion wirkt, also der Motor immer wieder mal tempor{\"{a}}r ausf{\"{a}}llt sollte die
Zeit um die Strecke abzufahren proportional mit der Fehlerinjektionswahrscheinlichkeit steigen.


\clearpage
\printbibliography

\clearpage
\thispagestyle{empty}
\section*{Eidesstattliche Erklärung}

\vspace*{1cm}

\begin{LARGE}
    Eidesstattliche Erklärung zur Bachelorarbeit
\end{LARGE}

\vspace*{1cm}

\noindent
Ich versichere, die Bachelorarbeit selbstständig und lediglich unter Benutzung der angegebenen Quellen und Hilfsmittel verfasst zu haben. \\

\noindent
Ich erkläre weiterhin, dass die vorliegende Arbeit noch nicht im Rahmen eines anderen Prüfungsverfahrens eingereicht wurde.

\begin{displaymath}
\begin{array}{ll}
Unterschrift:~~~~~~~~~~~~~~~~~~~~~~~~~~~~~~~~~~~~~~~~~~
& Ort, Datum:~~~~~~~~~~~~~~~~~~~~~~~~~~~~~~~~~~~~~~~~~~
\end{array}
\end{displaymath}


\end{document}
